% file: .../doc/snacc.tex

% $Header: /usr/app/odstb/CVS/snacc/doc/snacc.tex,v 1.1 1997/01/01 22:47:25 rj Exp $
% $Log: snacc.tex,v $
% Revision 1.1  1997/01/01 22:47:25  rj
% first check-in
%

% file: .../doc/misc-hyph.tex

% $Header: /usr/app/odstb/CVS/snacc/doc/misc-hyph.tex,v 1.1 1997/01/01 22:47:27 rj Exp $
% $Log: misc-hyph.tex,v $
% Revision 1.1  1997/01/01 22:47:27  rj
% first check-in
%

\hyphenation{
  snacc-path
  Asn-Type
  Asn-Type-Desc
  ostream
  nmdescs
  mdescs
  desc
}


% different methods get a backslash:
%   \verb_\_ -> Courier
%   \char`\\ -> Times-Roman
%   $\backslash$ -> math font

% for 2up (2x A5 on A4):
%\def\fmtsize{12pt}
%% for A4:
\def\fmtsize{10pt}

\documentclass[\fmtsize,a4paper,twoside]{report}
\usepackage{t1enc}
\usepackage{isolatin1}
\usepackage{times}
%\usepackage{latexsym}
\usepackage{amssymb}
\usepackage[dvips]{graphicx}

\sloppypar

\newlength{\spacing}
\setlength{\spacing}{\baselineskip}
\newcommand{\nspace}[1]{\setlength{\baselineskip}{#1\spacing}}
\newenvironment{linespacing}[1]{\nspace{#1}}{}

%% myitemize environment---less space between items
%\newenvironment{myitemize}[0]{\begin{itemize}\nspace{0.5}} {\end{itemize}}

%\setlength{\topmargin}{-.50in}
%\setlength{\textwidth}{6.0in}
%\setlength{\textheight}{8.5in}
%\setlength{\oddsidemargin}{.25in}
\def\myparindent{0in}
\setlength{\parindent}{\myparindent}
\def\myparskip{8pt}
\setlength{\parskip}{\myparskip}
%\def\myparsep{\parsep}
%\def\myitemsep{\itemsep}

\input{misc-defs}

\def\snaccversion{1.3b3}


\begin{document}
\thispagestyle{empty}
\title{Snacc 1.2rj: A High Performance ASN.1 to C/C++/IDL Compiler}
\author{Michael Sample\\
        msample@cs.ubc.ca\\\\
        Department of Computer Science\\
        University of British Columbia\\
        6356 Agricultural Rd.\\
        Vancouver, British Columbia\\
        Canada, V6T 1Z2\\\\
	\\
	augmented by:\\
	Robert Joop\\
	$<$rj@rainbow.in-berlin.de$>$}
%\date{February 1993, updated July 1993,
%augmented by Robert Joop $<$rj@rainbow.in-berlin.de$>$ 1994/1995}
\date{msample: February 1993, updated July 1993\\
rj: 1994/1995\\
\quad\\
\emph{Preliminary documentation as of \today\\for Snacc \snaccversion}
}
\maketitle

% copyright notice on software and documentation

This work was made possible by grants from the Canadian Institute for
Telecommunications Research (CITR) and Natural Sciences and
Engineering Research Council of Canada (NSERC).

Copyright (C) 1990, 1991, 1992, 1993 Michael Sample
          and the University of British Columbia

Copyright \copyright 1994, 1995 Robert Joop
	and GMD FOKUS

This program, Snacc, is free software; you can redistribute it and/or
modify it under the terms of the GNU General Public License as
published by the Free Software Foundation; either version 2 of the
License, or (at your option) any later version.

The runtime libraries are copyright to the University of British
Columbia and Michael Sample.  They are free software; you can
redistribute them and/or modify them as long as the original,
unmodified copyright information with/in them.  The GNU Library
Public License has been removed as of version 1.1.

What we're trying to say is: you can't sell the compiler but you can
sell products that use the code generated by the compiler and the
runtime libraries.

This program and the associated libraries are distributed in the hope
that they will be useful, but WITHOUT ANY WARRANTY; without even the
implied warranty of MERCHANTABILITY or FITNESS FOR A PARTICULAR
PURPOSE\@.  See the GNU General Public License for more details.

You should have received a copy of the GNU General Public License and
the GNU Library General Public License along with this program; if
not, write to the Free Software Foundation, Inc., 675 Mass Ave,
Cambridge, MA 02139, USA\@.

Permission is granted to make and distribute verbatim copies of
this manual provided the copyright notice and this permission notice
are preserved on all copies.

Permission is granted to copy and distribute modified versions of this
manual under the conditions for verbatim copying, provided that the
entire resulting derived work is distributed under the terms of a
permission notice identical to this one.

% table of contents
%\setlength{\parskip}{0pt}
\tableofcontents
%\setlength{\parskip}{\myparskip}

% list of figures
\listoffigures

% list of tables
%listoftables

% file: .../doc/intro-1.2.tex

% $Header: /usr/app/odstb/CVS/snacc/doc/intro-1.2.tex,v 1.2 1997/02/16 16:49:30 rj Exp $
% $Log: intro-1.2.tex,v $
% Revision 1.2  1997/02/16 16:49:30  rj
% made return *this after calling abort()'' a compile time option.
%
% Revision 1.1  1997/01/01 22:47:30  rj
% first check-in
%

\chapter{\label{intro-1.2}Introduction}

Snacc compiles ASN.1 \cite{X.208} (Abstract Syntax Notation One) modules into C, C++, CORBA IDL \cite{corba} or type tables.
The generated C or C++ code contains equivalent data structures and
routines to convert values between the internal (C or C++)
representation and the corresponding BER \cite{X.209} (Basic Encoding Rules) format.
The name ``snacc'' is an acronym for ``Sample Neufeld ASN.1 to C/C++ Compiler''.

This is release 1.2rj\footnote{\emph{1.2} since it is the successor of 1.1 and \emph{rj} as i don't think that I'm the only one who worked on Snacc.} of Snacc.
This chapter lists only the differences to the original author's last release, Snacc~1.1.
The list in this chapter is incomplete---for a more thorough enumeration, see the file {\ufn .../ChangeLog}.

New features include:
\begin{itemize}
\item
  The output files generated get names derived from their input file's name, with only the suffix replaced.
  This eases makefile writing, as now you can use simple suffix rules or other forms of filename pattern matching.
  The old behaviour, where the output files got their name from the ASN.1 module name, can be retained by using the {\ufn -mm} command line switch to {\ufn snacc}.
\item
  The C++ backend generates code with a much more complete set of constructors, destructors and assignment operators.
\item
  The C++ backend can supply the generated C++ classes with \emph{meta} information about their own structure.
  This information can be used to build interpreted interfaces; the Snacc 1.2rj distribution contains a Tcl interface that uses this meta information as well as a Tcl script (that uses the Tcl interface) for a simple editor.
\item
  Snacc has successfully been ported to Linux and Alpha OSF/1, and should be both byte order and 64 bit clean.
\item
  A new backend that generates CORBA IDL (work-in-progress, not even alpha)
\end{itemize}

The makefiles have been rewritten.
The old ones removed the {\ufn .o} files after successful compilation, and thus, for every tiny code change, a full recompilation took place!
With the new set of makefiles, only those files that need to be remade are.
Following usual conventions, the phony targets depend, check, install, clean and clobber have been added.
\newline
NOTE: the makefiles still are not safe for parallel making.
\\
See Appendix~\ref{makefile-sect} for some explanations of some the makefile tricks.
\\
If you've got problems with the makefiles, please refer to the appendix!

The 1.1 release used five config.h files, and three almost identical copies of the ASN.1 C library.
They have all been merged in the file {\ufn \dots/snacc.h}.
A very small number (currently three) of compilation switches has been put into {\ufn \dots/policy.h}.

The previous release produced huge virtual inline functions.
Due to their size, these inlines wouldn't get inlined anyway.
Virtual functions don't get inlined (they get referenced via pointer in the virtual function table).
Due to their size they wouldn't offer any speed advantage (the function call overhead diminishes).
Instead, the compiler generated static functions in every {\ufn .C} file were the {\ufn .h} file is included!
This inflates the {\ufn .o} files and executables real quick (I'm speaking of MBytes per executable).
These functions have been turned into normal functions.

\section{\label{install-section}Configuring and Installing Snacc}

First of all, if you haven't already done so, un-archive Snacc to
produce the directory {\ufn snacc-1.2rj.\textit{patchlevel}} and its contents.
The directory {\ufn snacc-1.2rj.\textit{patchlevel}} will henceforth be referred to as ``{\ufn \dots}''.
The following tools are required to compile Snacc:
\begin{itemize}
  \item {\ufn make} (GNU {\ufn make} is recommended)
  \item {\ufn patch} (for a tiny patch in {\ufn \dots/c-lib/})
  \item {\ufn makedepend} or a look-alike (all of them have their advantages and disadvantages, it is hard to recommend any of them, see below)
  \item {\ufn lex} or GNU's {\ufn flex} ({\ufn flex} is recommended)
  \item {\ufn yacc} or GNU's {\ufn bison} ({\ufn bison} is recommended)
  \item a C compiler (it doesn't have to support ANSI, K\&R will do)
\end{itemize}

Some versions of {\ufn yacc} may choke due to the large size of the
{\ufn parse-asn1.y} file, however, we have had no problems with {\ufn bison}.
Our {\ufn yacc} grammar for ASN.1 has 61 shift/reduce errors and 2
reduce/reduce errors.  Most of these errors were introduced when
certain macros were added to the compiler.  Some of the shift/reduce
errors will require you to follow the offending macro in the ASN.1
module with a semi-colon.  The reduce/reduce errors were introduced by
macros that have ``Type or Value Lists'' because the NULL Type and
NULL values use the same symbol, ``NULL''.  This is not a problem
since no real processing is done with the macros in question at the
present.

{\ufn Lex} will work for the {\ufn lex-asn1.l} file but {\ufn flex} will typically
produce a smaller executable.  Most versions of {\ufn lex} have a small
maximum token size that will cause problems for long tokens in the
ASN.1 source files, such as quoted strings.  To avoid this problem,
increase the {\C YYLMAX} value in the generated {\ufn lex-asn1.c} file to at least
2048.  {\ufn Flex} does not seem to have this problem.

The configuration process has been simplified (at least for the installer of Snacc ;-) by the use of GNU autoconf.

The only file has may have to be edited is {\ufn \dots/policy.h}.
It contains a few compilation switches you may want to toggle.

The behaviour of makedepend has been changed from X11R5 to R6.
The new version keeps the source files' dirname and replaces the suffix only, the old version removed the dirname.
The makefiles expect the new behaviour.
If you've still got the R5 makedepend, the compiler's dependencies will be useless.
(If you only install the code and don't make any source code changes, this won't hurt you.)
If you haven't got {\ufn makedepend}, you can use any of the look-alikes, which often are {\ufn sh}-scripts calling the compiler with the {\ufn -M}-switch.
If you don't plan to make any source code changes, you can replace {\ufn makedepend} with {\ufn /bin/true}.
\newline
Warning: MIT X11's makedepend is broken, in both R5 and R6.
It silently does not produce any output for many of Snacc's C++ files (in {\ufn \dots/c++-lib/}).
\newline
The C compiler called with the {\ufn -M}-switch gives much better results, but is \emph{much} slower.

The Snacc compiler and library C code has been written to support ANSI or non-ANSI C\@.
The configuration script tries to find out whether your C compiler understands ANSI C\@.

The configuration script generates two files:
\begin{description}
  \item[{\ufn \dots/makehead}] gets included by all makefiles.
    It contains a lot of definitions used by make.
  \item[{\ufn \dots/config.h}] contains all the machine, operating system, compiler and environment dependent settings.
    It is included by {\ufn \dots/snacc.h}.
\end{description}

The C++ runtime library is known to compile with both {\ufn gcc 2.5.8} and {\ufn gcc 2.6.3}.
The latter has the {\C bool} type built-in (which the configuration script automatically detects).

Finally, to compile {\ufn snacc} and the C and C++ runtime libraries,
type the following at the shell prompt:

\begin{verbatim}
%1 cd snacc-1.2rj.*
%2 ./configure
%3 make
\end{verbatim}

If you wish to install only the C (including type tables) or only the
C++ versions of the library, type {\ufn make c} or {\ufn make c++},
respectively, instead of {\ufn make}.  If the make succeeds, the
{\ufn snacc} binary should be present as {\ufn \dots/compiler/snacc},
the C runtime libraries, {\ufn libasn1csbuf.a},
{\ufn libasn1cebuf.a}, {\ufn libasn1cmbuf.a} and {\ufn libasn1ctbl.a}, should be in
{\ufn \dots/c-lib/} and the C++ runtime library, {\ufn libasn1c++.a}
(and, if you compiled with the Tcl option enabled,
another runtime library, {\ufn libasn1tcl.a}),
should be in {\ufn \dots/c++-lib/}.
The type table tools,
{\ufn ptbl}, {\ufn pval} and {\ufn mkchdr}, will be in their respective directories under {\ufn \dots/tbl-tools/}.

After compiling the libraries, you can test the library routines by calling {\ufn make check}
(or by calling {\ufn make c-check} or {\ufn make c++-check} to test the C or C++ library routines only, respectively).

Manual pages that contain information on running {\ufn snacc} and the type table tools can be found in {\ufn \dots/doc/}.  

To install Snacc, you can call {\ufn make install} (or {\ufn make c-install} or {\ufn make c++-install}, respectively).
This installs the snacc compiler binary, the libraries, the {\ufn .h} and {\ufn .asn1} files, the type table tools, as well as the manual pages into the usual directories.

To remove the {\ufn .o} and other intermediate files, you can call {\ufn make clean}.
To remove the binaries, libraries and all other generated files as well, call {\ufn make clobber}.

\section{\label{run-section}Running Snacc}

Snacc is typically invoked from the shell command line and has the synopsis:
\begin{verbatim}
snacc [-h] [-P] [-t] [-e] [-d] [-p] [-f]
      [ -c | -C | -idl | -T <table output file>]
      [-meta <module>.<type>[,...]] [-mA | -mC]
      [-tcl <module>.<type>[,...]]
      [-u <useful types ASN.1 file>]
      [-mm] [-mf <max file name len>]
      [-l <neg number>]
      [-novolat]
      <ASN.1 file list>
\end{verbatim}

Snacc generates C or C++ source code for BER encode and decode
routines as well as print and free routines for each type in the given
ASN.1 modules.  Alternatively, snacc can produce type tables that can
be used for table based/interpreted encoding and decoding.  The type
table based methods tend to be slower than their C or C++ counterparts
but they tend use less memory (table size vs. C/C++ object code).

Snacc may also be used to generate CORBA IDL\@.
This part of Snacc is very new and I would rate it as pre-alpha.

The {\ufn -meta}, {\ufn -mA}, {\ufn -mC} and {\ufn -tcl} options are only present when the Tcl and Tk libraries where found at configuration time.

Most of the 1990 ASN.1 features are parsed although some do not affect
the generated code.  Fairly rigourous error checking is performed on
the ASN.1 source; any errors detected will be reported (printed to
{\C stderr}).

Each file in the ASN.1 file list should contain a complete ASN.1
module.  ASN.1 modules that use the IMPORTS feature must be compiled
together (specify all necessary modules in the ASN.1 file list).  The
generated source files will include each module's header file in the
command line order.  This makes it important to order the modules from
least dependent to most dependent on the command line to avoid type
ordering problems. Currently, snacc assumes that each ASN.1 file
given on the command line depends on all of the others on the command
line.  No attempt is made to only include the header files from
modules referenced in the import list for that module.

If the target language is C, snacc will generate a {\ufn .h} and {\ufn .c} file for each specified ASN.1 module.
If the target language is C++, snacc will generate a {\ufn .h} and {\ufn .C} file for each module.
If the target language is CORBA IDL, snacc will generate an {\ufn .idl} file for each module.
The generated file names will be derived from the module's filenames, or from the
module names if the {\ufn -mm} command line switch has been given.

The command line options are:

\begin{description}
  \item[--h    ] {short for ``help'', prints a synopsis of snacc and exits.}

  \item[--c    ] {causes snacc to generate C source code.
  This is the default behaviour of snacc if neither of the {\ufn -c} or {\ufn -C} options are given.
  Only one of the {\ufn -c}, {\ufn -C}, {\ufn -idl} or {\ufn -T} options should be specified.}

  \item[--C    ] {causes snacc to generate C++ source code.}

  \item[--novolat] {causes snacc to generate C++ ``{\C return *this}''
  after calling {\C abort()}. (Some broken compilers don't know about
  volatile functions, or their abort() isn't correctly typed.)}

  \item[--idl  ] {causes snacc to generate CORBA IDL source code.}

  \item[--T \emph{file}] {causes snacc to generate type tables and write them to the given file \emph{file}.}

  \item[--meta \emph{types}]
    causes snacc to generate C++ classes with type meta information.
    Requires C++ functionality and therefore implies {\ufn -C} (C++ code generation).

  The \emph{types} denote the PDUs and have the following syntax: a comma-separated list of pairs of: module name, a dot, and a type name from that module. (Example: {\ufn snacc -tcl M1.T-a,M-2.Tb mod1.asn1 m2.asn1})

  \item[--mA \textnormal{and} --mC]
    causes the metacode to use identifiers as defined in the ASN.1 source files or as used in the generated C++ code, respectively.
    (It defaults to {\ufn -mC}.)

  \item[--tcl \emph{types}]
    causes snacc to generate functions for a Tcl interface.
    Needs the type meta information and thus implies {\ufn -meta} (see above).
    The {\ufn -meta} option can and should be omitted, the \emph{types} are as for the {\ufn -meta} option (the \emph{types} arguments are additive, in case you specify both options).

  \item[--P    ] {causes snacc to print the parsed ASN.1 modules to {\C stdout} after the types have been linked, sorted, and processed.
  This option is useful for debugging snacc and observing the modifications snacc performs on the types to make code generation simpler.}
\end{description}

The options, {\ufn -t, -v, -e, -d, -p,} and {\ufn -f} affect
what types and routines go into the generated source code.
These options do not affect type table generation. If none of
them are given on the command line, snacc assumes that all of them are
in effect.  For example, if you do not need the Free or Print
routines, you should give the {\ufn -t -v -e -d} options to snacc.
This lets you trim the size of the generated code by removing
unnecessary routines; the code generated from large ASN.1
specifications can produce very large binaries.

\begin{description}
\item[--t    ]
  causes snacc to generate type definitions in the target language for each ASN.1 type.

\item[--v    ]
  causes snacc to generate value definitions in the target language for each ASN.1 value.
  Currently value definitions are limited to INTEGERs, BOOLEANs and OBJECT IDENTIFIERs.

\item[--e    ]
  causes snacc to generate encode routines in the target language for each ASN.1 type.

\item[--d    ]
  causes snacc to generate decode routines in the target language for each ASN.1 type.

\item[--p    ]
  causes snacc to generate print routines in the target language for each ASN.1 type.

\item[--f    ]
  causes snacc to generate free routines in the target language for each ASN.1 type.
  This option only works when the target language is C\@.
  The free routines hierarchically free C values.
  A more efficient approach is to use the provided nibble-memory system.
  The nibble memory permits freeing an entire decoded value without traversing the decoded value.
  This is the default memory allocator used by snacc generated decoders.
  See file {\ufn \dots/c-lib/inc/asn-config.h} to change the default memory system.
  For more information on the memory management see Section~\ref{lib-mem-C-section}.

\item[--u \emph{file}]
  causes snacc to read the useful types definitions from the ASN.1 module in file \emph{file} for linking purposes.
  For some ASN.1 specifications, such as SNMP, the useful types are not needed.
  The types in the given useful types file are globally available to all modules; a useful type definition is overridden by a local or explicitly imported type with the same name.
  The useful type module can be found in {\ufn \dots/asn1specs/asn-useful.asn1} and contains:
  \begin{itemize}
    \setlength{\itemsep}{0pt}
    \setlength{\parsep}{0pt}
    \nspace{0}
    \item ObjectDesccriptor
    \item NumericString
    \item PrintableString
    \item TeletexString
    \item T61String
    \item VideoTexString
    \item IA5String
    \item GraphicString
    \item ISO646String
    \item GeneralString
    \item UTCTime
    \item GeneralizedTime
    \item EXTERNAL
  \end{itemize}

\item[--mm]
  This switch is supplied for backwards compatibility.
  Snacc versions 1.0 and 1.1 produced files with names generated from the ASN.1 module name contained in the input file.
  Snacc 1.2rj by default retains the input file name and replaces the suffix only.
  The new behaviour makes {\ufn makefile} writing easier, as with modern {\ufn make}s, pattern matching can be used.

\item[--mf \emph{number}]
  causes the names of the generated source files to have a maximum length of \emph{number} characters, including their suffix.
  The \emph{number} argument must be at least 3.
  This option is useful for supporting operating systems that only support short file names.
  A better solution is to shorten the module name of each ASN.1 module.

\item[--l \emph{number}]
  this is fairly obscure but may be useful.
  Each error that the decoders can report is given an id number.
  The number \emph{number} is where the error ids start decreasing from as they are assigned to errors.
  The default is -100 if this option is not given.
  Avoid using a number in the range -100 to 0 since they may conflict with the library routines' error ids.
  If you are re-compiling the useful types for the library use -50.
  Another use of this option is to integrate newly generated code with older code; if done correctly, the error ids will not conflict.

\end{description}

Since ASN.1 has different scoping rules than C and C++, some name munging
is done for types, named-numbers etc. to eliminate conflicts.
Some capitalization schemes were chosen to fit common C programming
style.  For all names, dashes in the ASN.1 source are converted to
underscores.  See Sections \ref{naming-C-section} and \ref{naming-C++-section}
for more naming information.

If the {\ufn -mm} switch has been given, the module name is used as a base name for the generated source file
names.  It will be put into lowercase and dashes will be replaced with
underscores.  Module names that result in file names longer than
specified with the {\ufn -mf} option will be truncated.  If the
{\ufn -mf} option was not given, file names will be truncated if they
are too long for the target file system. You may want to shorten long
module names to meaningful abbreviations. This will avoid file name
conflicts for module names that are truncated to the same substring.
Any module name and file name conflicts will be reported.

If your ASN.1 modules have syntactic or semantic errors, each error
will be printed to {\C stderr} along with the file name and line number of
where it occurred. These errors are usable by GNU emacs compiling
tools.  See the next chapter for more information on the types of
errors snacc can detect.

More errors can be detected and reported in a single compile if type
and value definitions are separated by semi-colons.  Separating type
and value definitions with semi-colons is not required, and if used,
need not be used to separate all type and value definitions.
Semi-colons are necessary after some macros that introduce ambiguity.
In general, if you get a parse error you can't figure out, try
separating the surrounding type/value definitions with semicolons.


\subsection{Known Bugs}

\begin{itemize}
  \item
    Snacc has problems with the following case:
    \begin{ASNcode}
      Foo ::= SEQUENCE\\
      \{\+\\
	id IdType,\\
	val ANY DEFINED BY id\-\\
      \}\\
      \\
      IdType ::= CHOICE\\
      \{\+\\
	a INTEGER,\\
	b OBJECT IDENTIFIER\-\\
      \}
    \end{ASNcode}

    The error checking pass will print an error to the effect that the id
    type must be INTEGER or OBJECT IDENTIFER\@.  To fix this you must modify
    the error checking pass as well as the code generation pass.  To be
    cheap about it, disable/fix the error checking and hand modify the
    generated code.

  \item
    The hashing code used for handling ANY DEFINED BY id to type mappings
    will encounter problems if the hash table goes more than four levels
    deep (I think this is unlikely).  To fix this just add linear chaining
    at fourth level.

  \item 
    The {\ufn \dots/configure} script should check whether the machine's floating point format is IEEE or whether the IEEE library exists.

  \item
    The C++ library severly lacks a convenient buffer management class that automatically expands like the C libraries' ExpBuf.
    What use is an efficient buffer management when you have got to build a loop a round snacc's encoding routine that reallocates larger buffers until the result fits?

  \item
    Where this document describes personal experiences, it is usually unclear to which author `I' refers.
    (One way to find out is to look at snacc~1.1's documentation.)

\end{itemize}

\section{\label{bug-section}Reporting Bugs and Your Own Improvements}

Snacc 1.1 was Michael Sample's final release.
While he is watching Snacc's development, he isn't actively developing it himself.

Since there are quite a number of changes from release 1.1 to 1.2rj, bug reports and new features are best sent to me.
I can be reached as \texttt{Robert Joop <rj@rainbow.in-berlin.de>} or \texttt{<rj@fokus.gmd.de>}.

% file: .../doc/intro-1.1.tex

% $Header: /usr/app/odstb/CVS/snacc/doc/intro-1.1.tex,v 1.1 1997/01/01 22:47:28 rj Exp $
% $Log: intro-1.1.tex,v $
% Revision 1.1  1997/01/01 22:47:28  rj
% first check-in
%

\chapter{\label{intro-1.1}Introduction to Snacc Release~1.1}

Snacc compiles ASN.1 \cite{X.208} (Abstract Syntax Notation One)
modules into C, C++ or type tables.  The generated C or C++ code
contains equivalent data structures and routines to convert values
between the internal (C or C++) representation and the corresponding
BER \cite{X.209} (Basic Encoding Rules) format.  The name ``snacc'' is
an acronym for ``Sample Neufeld ASN.1 to C/C++ Compiler''.

This compiler was written so I could do some encoding performance
research for my M.Sc.  See \cite{Sample93-1}, or \cite{Sample93-2} for
the results of that research.  A techreport will soo be available
via ftp from UBC, in the same directory as snacc.

The ASN.1 data structure language can specify complex types such as
lists and recursively defined types. BER data values are defined
independently of any computer architecture, providing a universal data
value representation that is useful for sharing data in heterogeneous
networks.

The process of converting an ASN.1 value from its C or C++
representation into an equivalent BER data value is called encoding
and the reverse process is called decoding.  This document was written
assuming that the reader is familiar ASN.1 and BER.  Further
information on ASN.1 and BER can be found in \cite{ASN.1Book},
\cite{ASN.1Overview}, \cite{X.208} and \cite{X.209}.

Compiling ASN.1 into C is not a new idea but many other tools such as
UBC's CASN1 \cite{CASN1}, ISODE's PEPY/POSY \cite{ISODE}, and
commercial tools either do not parse ASN.1 '90, produce slow encoders
and decoders or are outrageously expensive. The aim of this tool is to
provide an ASN.1 compiler that parses ASN.1 '90, produces efficient
encoding and decoding routines and is freely available.  Effort has
been made to make the generated encoders and decoders relatively easy
to fit into different software environments.

The table driven encoders are useful for certain applications such as
protocol testing.  They are also useful if you need to dynamically
load new ASN.1 definitions.  It is also fairly simple to write your
own special ASN.1 tools based on tables (e.g. a protocol tester that
verifies that values conform to a given ASN.1 type definition).  The
price of the flexibility is speed; they are slower (~4 times) than the
compiled C and C++ versions.

Some of snacc's features include:
\begin {itemize}
\item {parses CCITT ASN.1 '90 including  subtype notation}
\item {can compile and link inter-dependent ASN.1 modules (IMPORTS/EXPORTS)}
\item {some X.400 and SNMP macros are parsed}

\item {macro {\em definitions} do not generate syntax errors but are
are not processed. The macro definitions are retained as a string
internally (if you want to modify the compiler to process them).}

\item {value notation is parsed. OBJECT IDENTIFIERs, INTEGERs
and BOOLEANs are translated to C/C++ values. Any other value in \{\}'s
is kept as a string internally (if you want to modify the compiler to
process them).}

\item {optionally supports ``;'' separated type or value definitions
      in the ASN.1 source. This is useful for dealing with some macros and
      other language ambiguities that introduce parsing problems.}
\item {ANY DEFINED BY types are supported via the SNMP OBJECT-TYPE macro}
\end{itemize}

\section{\label{old-install-section}Installing snacc}

First of all, if you haven't already done so, un-archive snacc to
produce the directory {\em snacc} and its contents.  The following tools
are required to compile snacc:
\begin{itemize}
\item {\verb$lex$ or GNU's \verb$flex$ (\verb$flex$ is recommended)}
\item {\verb$yacc$ or GNU's \verb$bison$ (\verb$bison$ is recommended)}
\item {a C compiler and \verb$make$}
\end{itemize}

Some versions of \verb$yacc$ will choke due to the large size of the
\verb$asn1.yacc$ file, however, I have had no problems with \verb$bison$.
Our \verb$yacc$ grammar for ASN.1 has 61 shift/reduce errors and 2
reduce/reduce errors.  Most of these errors were introduced when
certain macros were added to the compiler.  Some of the shift/reduce
errors will require you to follow the offending macro in the ASN.1
module with a semi-colon.  The reduce/reduce errors were introduced by
macros that have ``Type or Value Lists'' because the NULL Type and
NULL values use the same symbol, ``NULL''.  This is not a problem
since no real processing is done with the macros in question at the
present.

\verb$Lex$ will work for the \verb$asn1.lex$ file but \verb$flex$ will typically
produce a smaller executable.  Most versions of \verb$lex$ have a small
maximum token size that will cause problems for long tokens in the
ASN.1 source files, such as quoted strings.  To avoid this problem,
increase the \verb$YYLMAX$ value in the generated {\em lex.yy.c} file to at least
2048.  \verb$Flex$ does not seem to have this problem.

The compiler and library C code has been written to support ANSI or
non-ANSI C\@.  ANSI C is used by default; this can be configured in
\verb$snacc/c_include/asn_config.h$.

By default, the compiler's makefiles use \verb$flex$, \verb$bison$ and
\verb$gcc$.  If you wish to change these, edit the following files:

\begin{verbatim}
snacc/src/makefile
snacc/src/c_lib/makefile
snacc/src/back_ends/c_gen/makefile
snacc/src/back_ends/c++_gen/makefile
\end{verbatim}

The C runtime library uses \verb$gcc$, and its makefile is
\verb$snacc/c_lib/makefile$.  The C++ runtime library uses \verb$g++$
(\verb$gcc-2.2.3$) and its makefile is \verb$snacc/c++_lib/makefile$.
The type table library makefile uses \verb$gcc$ and is
\verb$snacc/tbl_lib/makefile$.

Finally, to compile \verb$snacc$ and the C and C++ runtime libraries,
type the following at the shell prompt:

\begin{verbatim}
%1 cd snacc
%2 make all
\end{verbatim}

If you wish to install only the C (including type tables) or only the
C++ versions of the library, type \verb$make c$ or \verb$make c++$,
respectively, instead of \verb$make all$.  If the make succeeds, the
snacc binary, {\em snacc}, should be in the \verb$snacc/bin/$
directory, the C runtime libraries, {\em libasn1csbuf.a}, {\em
libasn1cebuf.a}, and {\em libasn1cmbuf.a}, should be in the
\verb$snacc/c_lib$ and the C++ runtime library, {\em libasn1c++.a}
should be in the \verb$snacc/c++_lib$.  The type table library, {\em
libasn1tbl.a} will be in \verb$snacc/tbl_lib$.  The type table tools,
{\em ptbl}, {\em pval} and {\em mkchdr} will be in \verb$snacc/bin$.
The \verb$.o$ and other junk files will have been removed.

After compiling the libraries, you can test the library routines with
\linebreak \verb$snacc/c_examples/test_lib/test_lib$ or \linebreak
\verb$snacc/c++_examples/test_lib/test_lib$.  These programs run
simple encoding and decoding tests on all of the library types.  You
can test the snacc compiler with the other examples.

A manual page that contains information on running snacc can be found in
\verb$snacc/doc/snacc.1$.  This should be installed in section~1 of
the manual.  You can use \verb$nroff -man snacc.1$ to view it if you
don't want to install it.


\section{\label{old-run-section}Running snacc}

Snacc is typically invoked from the shell command line and has the synopsis:
\begin{verbatim}
snacc [-h] [-P] [-t] [-e] [-d] [-p] [-f]
      [ -c | -C | -T <table output file>]
      [-u <useful types ASN.1 file>]
      [-mf <max file name len>]
      [-l <neg number>]
      <ASN.1 file list>
\end{verbatim}

Snacc generates C or C++ source code for BER encode and decode
routines as well as print and free routines for each type in the given
ASN.1 modules.  Alternatively, snacc can produce type tables that can
be used for table based/interpreted encoding and decoding.  The type
table based methods tend to be slower than their C or C++ counterparts
but they tend use less memory (table size vs. C/C++ object code).

Most of the 1990~ASN.1 features are parsed although some do not affect
the generated code.  Fairly rigourous error checking is performed on
the ASN.1 source; any errors detected will be reported (printed to
\verb$stderr$).

Each file in the ASN.1 file list should contain a complete ASN.1
module.  ASN.1 modules that use the IMPORTS feature must be compiled
together (specify all necessary modules in the ASN.1 file list).  The
generated source files will include each module's header file in the
command line order.  This makes it important to order the modules from
least dependent to most dependent on the command line to avoid type
ordering problems. Currently, snacc assumes that each ASN.1 file
given on the command line depends on all of the others on the command
line.  No attempt is made to only include the header files from
modules referenced in the import list for that module.

If the target language is C, snacc will generate a \verb$.h$ and
\verb$.c$ file for each specified ASN.1 module.  If the target
language is C++, snacc will generate a \verb$.h$ and \verb$.C$ file
for each module.  The generated file names will be derived from the
module names.

The command line options are:
% zap bullet for items
%\def\labelitemi{}

\begin{description}
\item[-h    ] {short for ``help'', prints a synopsis of snacc
and exits.}

\item[-c    ] {causes snacc to generate C source code.
This is the default behaviour of snacc if neither of the \verb$-c$ or
\verb$-C$ options are given.  Only one of the \verb$-c$, \verb$-C$ or
\verb$-T$ options should be specified.}

\item[-C    ] {causes snacc to generate C++ source code.}

\item[-T {\em file}] {causes snacc to generate type tables and
write them to the given file {\em file}.}

\item[-P    ] {causes snacc to print the parsed ASN.1
modules to \verb$stdout$ after the types have been linked, sorted, and
processed.  This option is useful for debugging snacc and observing
the modifications snacc performs on the types to make code generation
simpler.}
\end{description}

The options, \verb$-t, -v, -e, -d, -p,$ and \verb$-f$ affect
what types and routines go into the generated source code.
These options do not affect type table generation. If none of
them are given on the command line, snacc assumes that all of them are
in effect.  For example, if you do not need the Free or Print
routines, you should give the \verb$-t -v -e -d$ options to snacc.
This lets you trim the size of the generated code by removing
unnecessary routines; the code generated from large ASN.1
specifications can produce very large binaries.

\begin{description}
\item[-t    ] {causes snacc to generate type definitions in the
target language for each ASN.1 type.}

\item[-v    ] {causes snacc to generate value definitions in the
target language for each ASN.1 value.  Currently value definitions are
limited to INTEGERs, BOOLEANs and OBJECT IDENTIFIERs.}

\item[-e    ] {causes snacc to generate encode routines in the
target language for each ASN.1 type.}

\item[-d    ] {causes snacc to generate decode routines in the
target language for each ASN.1 type.}

\item[-p    ] {causes snacc to generate print routines in the
target language for each ASN.1 type.}

\item[-f    ] {causes snacc to generate free routines in the
target language for each ASN.1 type.  This option only works when the
target language is C\@.  The free routines hierarchically free C values.
A more efficient approach is to use the provided nibble-memory system.
The nibble memory permits freeing an entire decoded value without
traversing the decoded value.  This is the default memory allocator
used by snacc generated decoders.  See file
\verb$snacc/c_include/asn_config.h$ to change the default memory
system.  For more information on the memory management see Section~\ref{lib-mem-C-section}.}

\item[-u {\em file}] {causes snacc to read the useful types
definitions from the ASN.1 module in file {\em file}
for linking purposes.  For some ASN.1 specifications, such as SNMP,
the useful types are not needed.  The types in the given useful types
file are globally available to all modules; a useful type definition
is overridden by a local or explicitly imported type with the same
name. The useful type module can be found in
\verb$snacc/asn1specs/asn-useful.asn1$ and contains:


\begin{itemize}
\setlength{\itemsep}{0pt}
\setlength{\parsep}{0pt}
\nspace{0}
\item ObjectDesccriptor
\item NumericString
\item PrintableString
\item TeletexString
\item T61String
\item VideoTexString
\item IA5String
\item GraphicString
\item ISO646String
\item GeneralString
\item UTCTime
\item GeneralizedTime
\item EXTERNAL
\end{itemize}}


\item[-mf {\em number}] {causes the names of the generated source
files to have a maximum length of {\em number} characters, including
their suffix.  The {\em number} argument must be at least~3.  This option
is useful for supporting operating systems that only support short
file names.  A better solution is to shorten the module name of each
ASN.1 module.}

\item[-l {\em number}] {this is fairly obscure but may be useful.  Each
error that the decoders can report is given an id number.  The number
{\em number} is where the error ids start decreasing from as they are
assigned to errors.  The default is -100 if this option is not given.
Avoid using a number in the range -100 to 0 since they may conflict
with the library routines' error ids.  If you are re-compiling the
useful types for the library use -50.  Another use of this option is
to integrate newly generated code with older code; if done correctly,
the error ids will not conflict.}

\end{description}

Since ASN.1 has different scoping rules than C and C++, some name munging
is done for types, named-numbers etc. to eliminate conflicts.
Some capitalization schemes were chosen to fit common C programming
style.  For all names, dashes in the ASN.1 source are converted to
underscores.  See Sections \ref{naming-C-section} and \ref{naming-C++-section}
for more naming information.

The module name is used as a base name for the generated source file
names.  It will be put into lowercase and dashes will be replaced with
underscores.  Module names that result in file names longer than
specified with the \verb$-mf$ option will be truncated.  If the
\verb$-mf$ option was not given, file names will be truncated if they
are too long for the target file system. You may want to shorten long
module names to meaningful abbreviations. This will avoid file name
conflicts for module names that are truncated to the same substring.
Any module name and file name conflicts will be reported.

If your ASN.1 modules have syntactic or semantic errors, each error
will be printed to \verb$stderr$ along with the file name and line number of
where it occurred. These errors are usable by GNU emacs compiling
tools.  See the next chapter for more information on the types of
errors snacc can detect.

More errors can be detected and reported in a single compile if type
and value definitions are separated by semi-colons.  Separating type
and value definitions with semi-colons is not required, and if used,
need not be used to separate all type and value definitions.
Semi-colons are necessary after some macros that introduce ambiguity.
In general, if you get a parse error you can't figure out, try
separating the surrounding type/value definitions with semicolons.


\subsection{Known Bugs}

Snacc has problems with the following case:

\begin{verbatim}
Foo ::= SEQUENCE
{
    id IdType,
    val ANY DEFINED BY id
}

IdType ::= CHOICE
{
    a INTEGER,
    b OBJECT IDENTIFIER
}
\end{verbatim}

The error checking pass will print an error to the effect that the id
type must be INTEGER or OBJECT IDENTIFER\@.  To fix this you must modify
the error checking pass as well as the code generation pass.  To be
cheap about it, disable/fix the error checking and hand modify the
generated code.

The hashing code used for handling ANY DEFINED BY id to type mappings
will encounter problems if the hash table goes more than four levels
deep (I think this is unlikely).  To fix this just add linear chaining
at fourth level.

On the deficiency side of things, the C++ classes really need to have
free methods defined for them. (Unless you have replaced new with
something like nibble memory)

\section{\label{old-bug-section}Reporting Bugs and Your Own Improvements}

This (1.1) is the final release of snacc (I have finished my M.Sc).
Gerald Neufeld \verb$<neufeld@cs.ubc.ca>$ was my supervisor but he
does not have time to deal with support (it is all my code anyway).
Luckily, a colleague has kindly offered to receive the bug reports and
to coordinate work done by others (i.e.  you).  His name is Barry
Brachman \verb$<brachman@cs.ubc.ca>$.  He did not write the code
(35,000+ lines of C) but he has used snacc for X.500 work.  He may be
able to point you to someone who has fixed or encountered the same
bug.  Anyway, be nice to him, it's not his job.

Even though this is the second release of snacc, bugs are still
likely.  In fact, this release was quite rushed so there are probably
lots of stupid installation bugs etc.  If you find some bugs or have
other comments, please send email to \verb$snacc-bugs@cs.ubc.ca$
(these will get to Barry and Gerald).  Please include the offending
ASN.1 source, the command line options you were using and the hardware
and operating system configuration.

If you are really keen and hack in new goodies, please share.  Send
them to Barry or \verb$snacc-bugs@cs.ubc.ca$. Look in
\verb$snacc/README.future$ for things you could work on.

As I mentioned, I have entered the real world. I am now working with
Open Systems Solutions (based in New Jersey).  If your application
needs a commercially developed and supported ASN.1 compiler, try
calling 1-609-987-9073 (Yeah, I know this is a plug, but it's a good
company).


% file: .../doc/design.tex

% $Header: /usr/app/odstb/CVS/snacc/doc/design.tex,v 1.1 1997/01/01 22:47:31 rj Exp $
% $Log: design.tex,v $
% Revision 1.1  1997/01/01 22:47:31  rj
% first check-in
%

\chapter{\label{comp-des-chapter}Compiler Design}

\section{\label{comp-overview-section}Overview}
The Snacc compiler is implemented with {\ufn yacc}, {\ufn lex}
(actually GNU's equivalents, {\ufn bison} and {\ufn flex}) and
\verb$C$.  Despite the shortcomings of {\ufn lex} and {\ufn yacc},
they provide reasonable performance without too much programming
effort.  Since {\ufn yacc} parsers are extremely difficult to modify
during runtime, any macro that you want the compiler to handle must
be hand coded into the ASN.1 {\ufn yacc} grammar
({\ufn \dots/compiler/core/parse-asn1.y}) followed by recompilation of snacc.
Macro definitions do not need special consideration since they are
skipped by the compiler.  Macro definitions and complex value notation
are kept as text in the data structure resulting from a parse if you
want to try to parse and process them.

To handle the anti-compiler nature of ASN.1's syntax, snacc makes
several passes on parse tree data structure when compiling.  None of
these passes creates temporary files; this allows snacc to process
large ASN.1 specifications quite quickly.  Each compiler pass is
explained in the next sections.  The main passes of the compiler are
executed in the following order:

\begin{enumerate}
\item parse useful types ASN.1 module
\item parse all user specified ASN.1 modules
\item link local and imported type references in all modules
\item parse values in all modules
\item link local and imported value references in all modules
\item process any macro types
\item normalize types
\item mark recursive types and signal any recursion related errors
\item check for semantic errors in all modules
\item generate C/C++ type information for each ASN.1 type
\item Sort the types from least dependent to most dependent
\item generate the C, C++, IDL or type table
\end{enumerate}

The source code for the compiler resides in {\ufn \dots/compiler/} and the
back ends are in {\ufn \dots/compiler/back-ends/c-gen/}, {\ufn \dots/compiler/back-ends/c++-gen/} and {\ufn \dots/compiler/back-ends/idl-gen/}.

\section{\label{comp-pass1-section}Pass 1: Parsing the Useful Types Module}
The ASN.1 useful types are not hardwired into snacc.  Instead they
have been placed in a separate ASN.1 module.  This allows the user to
define his own useful types or re-define the existing ones without
modifying snacc.  This also has the benefit that names of useful types
are not keywords in the lexical analyzer.  This step is not really a
compiler pass on the module data, however it is described as one for
simplicity.

The useful types module should be passed to snacc with the {\ufn -u}
flag in front of it.  The {\ufn -u} flag tells snacc to treat the
module in a special way.  Instead of parsing the module and generating
code for it, snacc parses the module and makes the types in it
accessible to all of the other modules being parsed.  Note that the
other modules do not need to explicitly import from the useful types
module.  See Section~\ref{comp-pass3-section} for more information on how
useful types affect linking.

The encode, decode, and other routines for the useful types are in the
runtime library.  Currently, the useful types library routines are the
same as the ones the compiler would normally generate given the useful
types module.  However, since they are in the library, you can modify
them to check character sets (string types), or convert local time
formats into their BER equivalent (UTCTime, GeneralizedTime).

The following types are in the useful types module:
\begin{small}
\begin{verbatim}
ASN-USEFUL DEFINITIONS ::=
BEGIN
ObjectDescriptor ::= [UNIVERSAL 7]  IMPLICIT OCTET STRING
NumericString    ::= [UNIVERSAL 18] IMPLICIT OCTET STRING
PrintableString  ::= [UNIVERSAL 19] IMPLICIT OCTET STRING
TeletexString    ::= [UNIVERSAL 20] IMPLICIT OCTET STRING
T61String        ::= [UNIVERSAL 20] IMPLICIT OCTET STRING
VideotexString   ::= [UNIVERSAL 21] IMPLICIT OCTET STRING
IA5String        ::= [UNIVERSAL 22] IMPLICIT OCTET STRING
GraphicString    ::= [UNIVERSAL 25] IMPLICIT OCTET STRING
VisibleString    ::= [UNIVERSAL 26] IMPLICIT OCTET STRING
ISO646String     ::= [UNIVERSAL 26] IMPLICIT OCTET STRING
GeneralString    ::= [UNIVERSAL 27] IMPLICIT OCTET STRING
UTCTime          ::= [UNIVERSAL 23] IMPLICIT OCTET STRING
GeneralizedTime  ::= [UNIVERSAL 24] IMPLICIT OCTET STRING

EXTERNAL         ::= [UNIVERSAL 8] IMPLICIT SEQUENCE
{
  direct-reference      OBJECT IDENTIFIER OPTIONAL,
  indirect-reference    INTEGER OPTIONAL,
  data-value-descriptor ObjectDescriptor OPTIONAL,
  encoding CHOICE
  {
    single-ASN1-type [0] OCTET STRING,  -- should be ANY
    octet-aligned    [1] IMPLICIT OCTET STRING,
    arbitrary        [2] IMPLICIT BIT STRING
  }
}
END
\end{verbatim}
\end{small}

If you use the EXTERNAL type, you must provide the mechanism to encode
and decode the value in the embedded CHOICE, \verb$encoding$.  The
type and transfer syntax of the value in an EXTERNAL type is not known
when the ASN.1 code is compiled by snacc.  Snacc cannot generate
encoders and decoders without complete type information and only
supports a single set of encoding rules, BER\@.

\section{\label{comp-pass2-section}Pass 2: Parsing ASN.1 Modules}
During this pass, all of the specified modules are parsed into the {\em
Module} data structure.  The ASN.1 source files are not consulted
again, after they are parsed.  {\ufn Yacc} and {\ufn lex} are doing the work in
this step. (see files {\ufn snacc.c}, {\ufn lex-asn1.l}, {\ufn parse-asn1.y}
and {\ufn asn1module.h}).

A lexical tie-in is where the yacc parser puts the lexical analyzer
into a different mode (and is usually considered a hack).  The
different modes tokenize symbols differently, which is useful for
skipping well delimited sections that cannot be parsed easily by a
{\ufn yacc} parser on the first pass.  Lexical tie-ins are used in two
places to simplify the ASN.1 grammar sufficiently for {\ufn yacc} and
{\ufn lex}.  There are two special modes in the lexical analyzer, one
for ASN.1 macro definitions and the other for ASN.1 values enclosed in
\{\}'s.

The lexical tie-in for eating macro definition bodies works with macro
definitions of the following form:

\begin{verbatim}
<upper case identifier> MACRO ::= BEGIN ... END
\end{verbatim}

Everything between the {\ASN BEGIN} and {\ASN END} is stuffed into a
string by {\ufn lex} and passed back as single token to the
{\ufn yacc} parser.

Values within \{\}'s are grabbed in a similar way.  Value parsing
cannot really be done at this stage since complete type information is
needed and the types are not fully parsed or linked yet.

Most syntax errors are reported during this pass.  If syntax errors
are encountered, snacc will report as many as it can from the
offending module before the parser is hopelessly lost and then exit.
If the types and values are separated with semi-colons, the parser can
recover after a syntax error and attempt to find more errors in that
module before exiting.


\section{\label{comp-pass3-section}Pass 3: Linking Types}
The third pass links all type references.  Snacc attempts to resolve
any currently visible (i.\ e.\  not in macro definitions or constructed
values) type reference.  This includes type references in simple value
definitions and subtyping information.  The useful types module (if
given) is linked first.

Snacc will exit after this pass if any type references could not be
resolved.  Error messages with file and line number information will
be printed to {\C stderr}.

This pass also counts and stores the number of times a type definition is
referenced locally and from other modules.  This information is used
during the type sorting pass.

First, each module identifier is checked for conflicts with the
others.  If the module identifier includes an OBJECT IDENTIFIER, snacc
only checks for conflicts with the other module identifier OBJECT
IDENTIFIERs.  When only a module name is provided, snacc checks for
conflicts with the the other module names, even if the other module
identifiers include OBJECT IDENTIFIERs.  If the OBJECT IDENTIFIER of
a module identifier contains any value references, it will be ignored
for module look-up purposes.  Note that value references within the
module identifier OBJECT IDENTIFIERs are not allowed in the 1992
version of ASN.1 due to the difficulty in module name resolution they
present.

Two modules with the same name but different OBJECT IDENTIFIERs are
not considered an error within ASN.1.  However, because the generated
files use the module name as part of their name, the code generation
pass will gripe about and fail for modules with the same name.

Next, each module's import {\em lists} are resolved by finding the
named module and then verifying that the named module contains all of
the imported types.

Then for each module, each type reference (except those of the form
{\em modulename.typename}) is assumed to be a local type reference and
the linker attempts to find a local type definition of the same name
to resolve it with.  If a matching local definition is found, the type
reference is resolved and the linker continues with the next type
reference.

For each type reference of the form {\em modulename.typename}, the
linker looks in the module with name {\em modulename} for the type
{\em typename}.  If the type is found the reference is resolved,
otherwise a linking error is reported.  Note that this form of type
reference provides a special scope that does not conflict with other
local or imported types in that module.

For type references that failed to resolve locally and are not of the
form {\em modulename.typename}, the linker looks in the import lists
of the current type reference's module for a type to resolve with.  If
the type is found in the import lists, the reference is resolved.

For the remaining unresolved type references (failed local and legal
import resolution and are not of the form {\em modulename.typename}),
the linker looks in the useful types module, if one was specified with
the {\ufn -u} option.  If the type is found in the useful types module
then the reference is resolved, otherwise a linking error is reported.

Note that when a useful types module is specified, it is globally
available to all modules, but it has the lowest linking priority.
That is, if a type reference can be resolved legally without the
useful types module, it will be.

Some type checking must be done in this pass to link certain types
properly.  These include:
\begin{itemize}
\item {a SELECTION type must reference a field of a CHOICE type.}
\item {a COMPONENTS OF type in a SET must reference a SET.}
\item {a COMPONENTS OF type in a SEQUENCE must reference a SEQUENCE.}
\end{itemize}



\section{\label{comp-pass4-section}Pass 4: Parsing Values}
The fourth pass attempts to parse any value that is enclosed in \{\}'s in
the given modules.  INTEGERS, REALs and BOOLEANS that are not enclosed in
braces are parsed in the first pass.

The value parser is implemented without {\ufn yacc} and {\ufn lex} and
uses each value's type information to help parse the value.  Values
within \{\}'s hidden within types such as default values and parts of
subtypes are not parsed.  Since subtypes and default values do not
affect the generated code, upgrading the value parser in this respect
is not very useful.

The only type of value in \{\}'s that is parsed is the OBJECT
IDENTIFIER\@.  All of the OBJECT IDENTIFIER value forms are supported
but snacc loosens the restrictions on using arc names defined in the
OBJECT IDENTIFIER tree.

ASN.1 allows OBJECT IDENTIFIER values to reference special built-in
arc names from the OBJECT IDENTIFIER tree defined in Annexes B, C and
D of X.208.  For example the first arc in an OBJECT IDENTIFIER value
can be either {\ASN ccitt} {\ASN iso} or {\ASN joint-iso-ccitt}.  The
acceptable arc names are context dependent; for example the second arc
can be one of {\ASN standard}, {\ASN registration-authority},
{\ASN member-body} or {\ASN identified-organization} only if the first
arc was {\ASN iso} or 1.

Snacc uses a simplified algorithm to handle references to the arc
names defined in the OBJECT IDENTIFIER tree.  Any arc value that is
represented by a single identifier is checked to see if it is one of
the arc names defined in the OBJECT IDENTIFIER tree; context is
ignored.  If the identifier matches one of the arc names then its
value is set accordingly.  The lack of context sensitivity in snacc's
algorithm may cause the arc name to link with an arc name from the
OBJECT IDENTIFIER tree when a local or imported INTEGER was desired.
The following is the list special arc names that snacc understands and
their values (see {\ufn \dots/compiler/core/oid.c}):

\begin{itemize}
\setlength{\itemsep}{0pt}
\setlength{\parsep}{0pt}
\nspace{0}
\item {ccitt = 0}
\item {iso = 1}
\item {joint-iso-ccitt = 2}
\item {standard = 0}
\item {registration-authority = 1}
\item {member-body = 2}
\item {identified-organization = 3}
\item {recommendation = 0}
\item {question = 1}
\item {administration = 2}
\item {network-operator = 3}
\end{itemize}

\section{\label{comp-pass5-section}Pass 5: Linking Values}
The fifth pass links value references. The value linker looks for
value references to resolve in value definitions and type definitions,
including default values and subtyping information.  The value linking
algorithm is virtually identical to the type linking pass (see Section
\ref{comp-pass3-section}).

Currently the value parsing is limited to OBJECT IDENTIFIER values.
Simple values that are not between \{\}'s are parsed in the first
pass.  Here is an example that illustrates the OBJECT IDENTIFIER
parsing and linking.  The following values:

\begin{small}
\begin{verbatim}
foo OBJECT IDENTIFIER ::= { joint-iso-ccitt 2 88 28 }
bar OBJECT IDENTIFIER ::= { foo 1 }
bell INTEGER ::= 2
gumby OBJECT IDENTIFIER ::= { foo bell }
pokie OBJECT IDENTIFIER ::= { foo stimpy(3) }
\end{verbatim}
\end{small}

\noindent
are equivalent to this:

\begin{small}
\begin{verbatim}
foo OBJECT IDENTIFIER ::= { 2 2 88 28 }
bar OBJECT IDENTIFIER ::= { 2 2 88 28  1 }
bell INTEGER ::= 2
gumby OBJECT IDENTIFIER ::= { 2 2 88 28 2 }
pokie OBJECT IDENTIFIER ::= { 2 2 88 28 3 }
\end{verbatim}
\end{small}

Note that in version 1.0, named arcs (e.g. {\ASN stimpy(3)}) were
promoted to full integer values.  This was wrong---many standards
re-used them (e.g. X.500 and {\ASN ds(5)}) leading to multiply defined
integer values.  If you want to improve the value parsing, look in
{\ufn \dots/compiler/core/val-parser.c}

\section{\label{comp-pass6-section}Pass 6: Processing Macros}

The fifth pass processes macros.  For all macros currently handled,
snacc converts type definitions inside the macro to type references
and puts the type definition in the normal scope.  This way, the code
generator does not have to know about macros to generate code for the
types defined within them.

The only macro that receives any special processing is the SNMP
OBJECT-TYPE macro.  This macro's information defines an OBJECT
IDENTIFIER or INTEGER to type mapping for use with any ANY DEFINED BY
type.  Note that the OBJECT-TYPE macro has been extended beyond its
SNMP definition to allow integer values for INTEGER to type mappings.

ASN.1 allows you to define new macros within an ASN.1 module;  this
can change the grammar of the ASN.1 language.  Since snacc is
implemented with {\ufn yacc} and yacc grammars cannot be modified
easily during runtime, snacc cannot change its parser in response to
macro definitions it parses.

Any macro that snacc can parse has been explicitly added to the yacc
grammar before compiling snacc.  When a macro that snacc can parse is
parsed, a data structure that holds the relevant information from the
macro is added to the parse tree.  The type and value linking passes
as well as the macro processing and possibly the normalization pass
need to be modified to handle any new macros that you add.

The following macros are parsed:

\begin{itemize}
%\begin{linespacing}{0.5}
\setlength{\itemsep}{0pt}
\setlength{\parsep}{0pt}
\nspace{0}
\item{ OPERATION (ROS) }
\item{ ERROR (ROS) }
\item{ BIND (ROS) }
\item{ UNBIND (ROS) }
\item{ APPLICATION-SERVICE-ELEMENT (ROS) }
\item{ APPLICATION-CONTEXT }
\item{ EXTENSION (MTSAS)}
\item{ EXTENSIONS (MTSAS) }
\item{ EXTENSION-ATTRIBUTE (MTSAS) }
\item{ TOKEN (MTSAS) }
\item{ TOKEN-DATA (MTSAS)}
\item{ SECURITY-CATEGORY (MTSAS) }
\item{ OBJECT (X.407) }
\item{ PORT   (X.407) }
\item{ REFINE (X.407)}
\item{ ABSTRACT-BIND (X.407) }
\item{ ABSTRACT-UNBIND (X.407) }
\item{ ABSTRACT-OPERATION (X.407) }
\item{ ABSTRACT-ERROR (X.407) }
\item{ ALGORITHM (X.509)}
\item{ ENCRYPTED (X.509)}
\item{ PROTECTED (X.509)}
\item{ SIGNATURE (X.509)}
\item{ SIGNED    (X.509)}
\item{ OBJECT-TYPE (SNMP) }
%\end{linespacing}
\end{itemize}

However, no code is generated for these macros.  As stated above, only
the OBJECT-TYPE macro affects the encoders and decoders.

\section{\label{comp-pass7-section}Pass 7: Normalizing Types}
The sixth pass normalizes the types to make code generation simpler.
The following is done during normalization:
\begin{itemize}

\item[1.] { COMPONENTS OF types are replaced with the contents of the SET
or SEQUENCE components that they reference.}

\item[2.] { SELECTION types are replaced with the type they reference.}

\item[3.] { SEQUENCE, SET, CHOICE, SET OF and SEQUENCE OF {\em definitions}
embedded in other types are made into separate type definitions. }

\item[4.] { For modules in which ``IMPLICIT TAGS'' is specified, tagged
type references such as {\ASN [APPLICATION 2] Foo} are marked IMPLICIT
if the referenced type ({\ASN FOO} in this case) is not an untagged
CHOICE or untagged ANY type.}

\item[5.] { INTEGERs with named numbers, BIT STRINGs with named bits and
ENUMERATED types embedded in other types are made into separate type
definitions.}
\end{itemize}

The COMPONENTS OF and SELECTION type simplifications are obvious but
the motivation for the others may not be so obvious.  The third type of
simplification makes type definitions only one level deep.  This
simplifies the decoding routines since snacc uses local variables for
expected lengths, running length totals and tags instead of stacks.

The implicit references caused by ``IMPLICIT TAGS'' are marked
directly on type references that need it.  This saves the code
generators from worrying about whether implicit tagging is in effect
and which types can be referenced implicitly.

The types with named numbers or bits are made into a separate type to
allow the C++ back end to simply make a class that inherits from the
INTEGER or BIT STRING class and defines the named numbers or bits
inside an enum in the new class.  This is described further in the C++
code generation chapter.

\section{\label{comp-pass8-section}Pass 8: Marking Recursive Types}


This pass marks recursive types and checks for recursion related
errors.  To determine whether a type definition is recursive, each
type definition is traced to its leaves, checking for references to
itself.  Both local and imported type references within a type are
followed to reach the leaves of the type.  A leaf type is a simple
(non-aggregate) built-in type such as an INTEGER or BOOLEAN\@. At the
moment, recursion information is only used during the type dependency
sorting pass.

{\em Snacc} attempts to detect two types of recursion related errors.  The
first type of error results from a recursive type that is composed
solely of type references.  Types of this form contain no real type
information and would result in zero-sized values.  For example the
following recursive types will generate this type of warning:
\begin{small}
\begin{verbatim}
A ::= B
B ::= C
C ::= A
\end{verbatim}
\end{small}

The other recursion related error results from a type whose value will
always be infinite in size.  This is caused by recursion with no
optional component that can terminate the recursion.  If the recursion
includes an OPTIONAL member of a SET or SEQUENCE, a CHOICE member, or
a SET OF or SEQUENCE OF, the recursion can terminate.

Both of the recursion errors generate warnings from snacc but will
not stop code generation.


\section{\label{comp-pass9-section}Pass 9: Semantic Error Checking}
The ninth pass checks for semantic errors in the ASN.1 specification
that have not been checked already. Both the type linking pass and the
recursive type marking pass do some error checking as well. Snacc attempts
to detect the following errors in this pass:

\begin{itemize}
\item { elements of CHOICE and SET types must have distinct tags.}

\item { CHOICE, ANY, and ANY DEFINED BY types cannot be implicitly tagged. }

\item { type and value names within the same scope must be unique. }

\item { field names in a SET, SEQUENCE or CHOICE must be distinct.  If
a CHOICE is a member of a SET, SEQUENCE or CHOICE and has no field name,
then the embedded CHOICE's field names must be distinct from its
parents to avoid ambiguity in value notation.}

\item { an APPLICATION tag code can only be used once per module. }

\item { each value in a named bit list (BIT STRINGs) or named number
list (INTEGERs and ENUMERATED) must be unique within its list.}

\item { each identifier in a named bit list or named number list must
be unique within its list.}

\item { the tags on a series of one or more consecutive OPTIONAL or DEFAULT
SEQUENCE elements and the following element must be distinct. }

\item { gives a warning if an ANY DEFINED BY type appears in a
SEQUENCE before its identifier or in a SET\@.  These would allow encodings
where the ANY DEFINED BY value was prior to its identifier in the
encoded value; ANY DEFINED BY values are difficult to decode without
knowing their identifier.}

\end{itemize}

Snacc does not attempt to detect the following errors due the
limitations of the value parser.
\begin{itemize}
\item { SET and SEQUENCE values can be empty (\{\}) only if the SET or
SEQUENCE type was defined as empty or all of its elements are marked
as OPTIONAL or DEFAULT.}

\item { each identifier in a BIT STRING value must from that BIT
STRING's named bit list (this could be done in an improved value
linker instead of this pass).}
\end{itemize}


\section{\label{comp-pass10-section}Pass 10: Generating C/C++ Type Information}

This pass fills in the target language type information.  The process
is different for the C and C++ back ends since the C++ ASN.1 model is
different and it was developed later (more design flaws had been
corrected for the C++ backend).

For C and C++ there is an array that contains the type {\em definition}
information for each built-in type.  For each built-in ASN.1 type, the
C array holds:

\begin{description}
\item[typename] {the C {\C typedef} name for this type definition.}

\item[isPdu]    {TRUE if this type definition is a PDU\@. This is set
for types used in ANY and ANY DEFINED BY types and those indicated by
the user via compiler directives.  Additional interfaces to the encode
and decode routines are generated for PDU types.  The SNMP OBJECT-TYPE
macro is the current means of indicating whether a type is used within
an ANY or ANY DEFINED BY type.}

\item[isPtrForTypeDef] { TRUE if other types defined solely by this type
definition are defined as a pointer to this type.}

\item[isPtrForTypeRef] { TRUE if type references to this type
definition from a SET or SEQUENCE are by pointer.}

\item[isPtrForOpt] { TRUE if OPTIONAL type references to this type
definition from a SET or SEQUENCE are by pointer.}

\item[isPtrInChoice] { TRUE if type references to this type
definition from a CHOICE are by pointer.}

\item[optTestRoutineName] { name of the routine to test whether an
OPTIONAL element of this type in a SET or SEQUENCE is present.
Usually just the name of a C macro that tests for NULL.}

\item[printRoutineName] { name of this type definition's printing routine.}
\item[encodeRoutineName]{ name of this type definition's encoding routine.}
\item[decodeRoutineName]{ name of this type definition's decoding routine.}
\item[freeRoutineName]  { name of this type definition's freeing routine.}
\end{description}

The C++ type definition array is similar to C's. It contains:

\begin{description}
\item[classname] { holds the C++ {\C class} name for this type definition.}
\item[isPdu] { same as C isPdu except that is does not affect the code
generation since the C++ back end includes the extra PDU encode and
decode routines by default.}
\item[isPtrForTypeDef] { same as C isPtrForTypeDef. }
\item[isPtrForOpt] { same as C isPtrForOpt.}
\item[isPtrInChoice] { same as C isPtrInChoice}
\item[isPtrInSetAndSeq] { whether type references to this class
from a SET or SEQUENCE are by pointer.}
\item[isPtrInList] {whether type references to this class
from a SET OF or SEQUENCE OF are by pointer.}
\item[optTestRoutineName] { name of the routine to test whether an
OPTIONAL element of this type in a SET or SEQUENCE is present.
Usually is just name of a C macro that tests for NULL.}
\end{description}

The first step of this pass uses the type arrays to fill in the C or
C++ type {\em definition} information for each module's ASN.1 type
definitions.  This is done for the useful types module as well.

The next step goes through each constructed type and fills in the type
{\em reference} information for each reference to a built-in, user defined
or useful type.  Much of the type reference information is taken from
the referenced type's definition information.  The type reference
information contains the following (for both C and C++):

\begin{description}
\item[fieldName] { field name for this type if it is referenced from
a CHOICE, SET or SEQUENCE.}
\item[typeName] { type name of the referenced type.}
\item[isPtr] { whether this reference is by pointer.}
\item[namedElmts] { named elements for INTEGER, ENUMERATED or BIT
STRING types with their C names and values.}
\item[choiceIdValue] { if this type reference is in a CHOICE, this
holds the value of the CHOICE's choiceId that indicates the presence
of this field.}
\item[choiceIdSymbol] { if this type reference is in a CHOICE, this
holds the C enum value symbol that has the choiceIdValue value.}
\item[optTestRoutineName] { name of the routine or macro to test for
the presence of this element if it is an OPTIONAL element of a SET or SEQUENCE.}
\end{description}

\section{\label{comp-pass11-section}Pass 11: Sorting Types}

This pass sorts the type definitions within each module in order of
dependence.  ASN.1 does not require the types to be defined before
they are referenced but both C and C++ do.  Without this pass, the
generated types/classes would probably not compile due to type
dependency problems.  There is no attempt to order the modules;
command line order is used for the module dependence.  If you have
problems with mutually dependent modules, the simplest approach is to
combine the dependent modules into a single ASN.1 module.

Some compilers such as CASN1 \cite{CASN1} require the user to order
the types within the ASN.1 modules.  This can be tedious and since
snacc may generate new type definitions from nested aggregate type
definitions in the normalization pass, the user does not have complete
control over the order of every type definition.  (The user could use
the {\ufn -P} option to get the normalized ASN.1 and then order it but
that is painful as well.)

Snacc attempts to sort the types from least dependent to most
dependent using the following convoluted algorithm:

First, separate the type definitions within a module into the groups:
\begin{itemize}
\item[1.] { type definitions that are defined directly from simple built-in
types such as INTEGER.}

\item[2.] { types such as SET, SEQUENCE, SET OF, SEQUENCE OF and CHOICE
that contain no references to types defined in this module.  That, is
they are defined from only simple built-in types, imported types or
useful types.}

\item[3.] { type definitions that reference locally defined types.}

\item[4.] { type definitions that are not referenced by any local types.}
\end{itemize}

Only the 3rd group of type definitions needs more sorting.  After it
has been sorted, the groups are merged in the order 1, 2, 3, 4 to
yield a sorted type definition list.

Now we describe how the 3rd group of type definitions is sorted.
\begin{itemize}

\item[1.] {for each type definition in the third group, a list of its local type
references is built and attached to it.  This type reference list only
goes one level deep; it does not follow type references to find more
type references.}

\item[2.] { all of the linearly-dependent types are removed and sorted.
This is done by repeatedly removing type definitions that do not
directly depend on any other type definitions that remain in the 3rd
group.  The process of removing the type definitions sorts them.}

\item[3.] { the type definitions that were not removed in step 2 are
divided into two groups: recursive and non-recursive.  The
non-recursive types depend on the recursive ones since they are still
in the list after step 2.}

\item[4.] { the non-recursive types from step 3 are sorted as in step
2.  All of them should sort linearly since none are recursive. }

\item[5.] { if the target language is C, any SET OF or SEQUENCE OF
types are separated from the recursive type definitions built in step 3.
This is done because the C representation of a list type is generic
(uses a {\C void~*} to reference the list element) and therefore does
not really depend on the list's element type.}

\item[6] { the list of local type references for the recursive types
from step 3 is re-generated as in step 1 using a relaxation:  types
referenced as pointers are not added to a type's reference list.}

\item[7] { the recursive types from step two are re-sorted as in step
2 using their new local type reference lists. Two lists are formed,
those that sorted linearly and those that did not. Hopefully the
latter list will be empty.}
\end{itemize}

To form a sorted third group, the lists are merged in the following order:
\begin{itemize}
\item {linearly sorted types from step 2}
\item {separated list types (C only) from step 5}
\item {sorted recursive types from step 7}
\item {unsorted recursive types from step 7 (hopefully empty)}
\item {sorted non-recursive types from step 4}
\end{itemize}


In C, the code generator defines both {\C typedef} names and
{\C struct} tags (names).  For example,
\begin{verbatim}
Foo ::= SET { a INTEGER, b BOOLEAN }

Bar ::= SEQUENCE { a OBJECT IDENTIFIER, b Foo }
\end{verbatim}
translates to the following C data types:
\begin{verbatim}
typedef struct Foo /* SET */
{
    AsnInt a; /* INTEGER */
    AsnBool b; /* BOOLEAN */
} Foo;

typedef struct Bar /* SEQUENCE */
{
    AsnOid a; /* OBJECT IDENTIFIER */
    struct Foo *b; /* Foo */
} Bar;
\end{verbatim}

Note that both the {\C struct} and the {\C typedef} have the name
{\C Foo}.  Also note that the Bar type references the {\C Foo} via
{\C struct Foo~*}.

For types such as {\C Bar} that contain the {\C Foo} type,
{\C Foo} is referenced as {\C struct Foo~*} instead of just
{\C Foo~*} because C allows you to use the type {\C struct Foo~*}
(incomplete type) in defining types even prior to the actual
declaration of the the {\C struct Foo}. The {\C Foo~*} type can
{\em only} be used after the {\C Foo typedef} declaration.  The use
of incomplete types can often overcome recursion related type ordering
problems (not relevant in this example since they are not recursive).

\section{\label{comp-pass12-section}Pass 12: Generating Code}

This pass creates and fills the source files with C or C++ code or
produces a type table containing the type descriptions from all of the
parsed modules, including the useful types module (if given).  The
purpose of the normalization, sorting and error detection passes is to
simplify this pass.

The normalization pass simplified the ASN.1 types in various ways to
make C/C++ type and code generation simpler.

The type sorting pass hopefully eliminates type dependency problems in the
generated code.  The C/C++ type generator simply proceeds through the
ordered type list writing the C/C++ type definitions to a header file.

The error detection and linking passes will make snacc exit if errors
are found, so the code generation pass can assume the ASN.1 types are
virtually error free.  This usually allows snacc to exit gracefully
instead of crashing due to an undetected error.

The type table data structure is similar to snacc's parse tree for the
ASN.1 modules but it is much simpler.  This is because all of the type
linking and error checking has been done.  The type definitions in the
type tables are in defined by the type sorting pass (dependency).

The next chapters describe the code that is generated by snacc and the
libraries the generated code uses.


% file: .../doc/c-gen.tex

% $Header: /usr/app/odstb/CVS/snacc/doc/c-gen.tex,v 1.1 1997/01/01 22:47:33 rj Exp $
% $Log: c-gen.tex,v $
% Revision 1.1  1997/01/01 22:47:33  rj
% first check-in
%

\chapter{\label{c-code-gen-chapter}C Code Generation}
\section{\label{intro-C-section}Introduction}
Snacc was designed primarily to provide high-performance encoders and
decoders.  Key areas to optimize are buffer and memory management.
Buffers are used to hold encoded values and the memory management is
used when building the internal representation of a value when decoding.

C macros are used where possible to eliminate function call overhead
for small, commonly used routines.  Using macros with constant
expressions as parameters allows smarter C compilers to do some of the
calculations at compile time.  In general, short-cuts that can be taken
without sacrificing the robustness of code are used.

The generated code can be quite large; large reductions of the size of
the binaries can be achieved by using the optimizing options of your C
compiler.

We will use an example ASN.1 module, EX1, to help explain snacc's code
generation.  The EX1 module uses some of the common built-in types and
contains some simple values.  The field names have been left out
to show snacc naming conventions.  The C generation code is in
{\ufn \dots/compiler/back-ends/c-gen/} if you want to alter it.

\begin{small}
\begin{verbatim}
EX1 DEFINITIONS ::=
BEGIN

anOidVal      OBJECT IDENTIFIER ::= { joint-iso-ccitt 40 foobar(29) }
theSameOidVal OBJECT IDENTIFIER ::= { 2 40 29 }
anIntVal      INTEGER ::= 1
aBoolVal      BOOLEAN ::= TRUE

T1 ::= SEQUENCE
{
   INTEGER OPTIONAL,
   OCTET STRING OPTIONAL,
   ENUMERATED { a(0), b(1), c(2) },
   SEQUENCE OF INTEGER,
   SEQUENCE { id OBJECT IDENTIFIER, value OCTET STRING },
   CHOICE { INTEGER, OBJECT IDENTIFIER }
}

END
\end{verbatim}
\end{small}

\noindent
Use the following command to compile the EX1 ASN.1 module:

\noindent
{\ufn \%1 snacc -u \dots/asn1specs/asn-useful.asn1 \dots/asn1specs/ex1.asn1}
\noindent

This produces the files {\ufn ex1.h} and {\ufn ex1.c}.

For each ASN.1 type an equivalent C data type, a BER encoding routine,
a BER decoding routine, a printing routine and a freeing routine will
be generated.  C values will also be generated from simple ASN.1
values.  Each aspect of the C code generation will be discussed in the
next sections.

\section{\label{naming-C-section}ASN.1 to C Naming Conventions}

For any given module, snacc may produce C type definitions, functions
and {\C \#define}s.  We assume that all C {\C typedef},
{\C struct}, {\C enum} and {\C union} tag, {\C enum} value,
variable, {\C \#define} and function names share a single name space.

The C type name for a type is the same as its ASN.1 type name (with
any hyphens converted to underscores) unless there is a conflict.
Since, unlike ASN.1, the C types for each ASN.1 module share the same
name space, snacc makes sure the C typenames are unique amoung all the
modules and that they do not conflict with C keywords.  The conflicts
are resolved by appending digits to the conflicting name.  To avoid
confusing numbered type names etc., you should edit the ASN.1 source
and name them properly.

Named numbers, ENUMERATED values and named bits are put in entirely in
upper case to match the common C convention for {\C \#define} and
{\C enum} values.

Empty field names in SETs, SEQUENCEs, and CHOICEs will be filled.  The
field name is derived from the type name for that field. The library
types such as INTEGER have default field names defined by the compiler
(see {\ufn \dots/compiler/back-ends/c-gen/rules.c} and
{\ufn \dots/compiler/back-ends/c++-gen/rules.c}).  The first letter of the field
name is in lower case.  Again, empty field names should be fixed
properly by adding them to the ASN.1 source.

New type definitions will be generated for SETs, SEQUENCEs, CHOICEs,
ENUMERATED, INTEGERs with named numbers and BIT STRING with named bits
whose definitions are embedded in other SET, SEQUENCE, SET OF,
SEQUENCE OF, or CHOICE definitions.  The name of the new type is
derived from the name of the type in which it was embedded.  Perhaps a
better way would use the field name as well, if present.

\section{\label{type-gen-C-section}ASN.1 to C Data Structure Translation}

To handle the different scoping rules between ASN.1 and C, the names
of some ASN.1 data structure elements such as ENUMERATED type symbols
may be altered to avoid conflicts.  The T1 type in example ASN.1
module EX1 has no field names so snacc will generate them.  It is
recommended to provide field names in the ASN.1 source instead of
relying on compiler generated names.  The following is the generated C
data structure for the EX1 module from the {\ufn ex1.h} file (function
prototypes have been removed):

\begin{small}
\begin{verbatim}

typedef enum
{
    A = 0,
    B = 1,
    C = 2
} T1Enum; /* ENUMERATED { A(0), B(1), C(2) }  */

typedef struct T1Choice /* CHOICE */
{
    enum T1ChoiceChoiceId
    {
        T1CHOICE_INT1,
        T1CHOICE_OID
    } choiceId;
    union T1ChoiceChoiceUnion
    {
        AsnInt int1; /* INTEGER */
        AsnOid *oid; /* OBJECT IDENTIFIER */
    } a;
} T1Choice;

typedef struct T1Seq /* SEQUENCE */
{
    AsnOid id; /* OBJECT IDENTIFIER */
    AsnOcts value; /* OCTET STRING */
} T1Seq;

typedef AsnList T1SeqOf; /* SEQUENCE OF INTEGER */

typedef struct T1 /* SEQUENCE */
{
    AsnInt *int1; /* INTEGER OPTIONAL */
    AsnOcts octs; /* OCTET STRING OPTIONAL */
    T1Enum t1Enum; /* T1Enum */
    T1SeqOf *t1SeqOf; /* T1SeqOf */
    struct T1Seq *t1Seq; /* T1Seq */
    struct T1Choice *t1Choice; /* T1Choice */
} T1;
\end{verbatim}
\end{small}

Every ASN.1 type definition maps into a C {\C typedef}.  SETs and
SEQUENCEs map into C structures and other simple types map into their
obvious C counterpart.  SET OF and SEQUENCE OF types map into a
generic list type which is doubly linked and NULL terminated.  The
reverse link on the lists allows for simpler backwards encoding.  More
information on the library types can be found in Chapter~\ref{lib-C-chapter}.

Comments that contain a fragment of each type's ASN.1 definition are
inserted in the header file to clarify cases where elements have been
re-named.

Aggregate types that are defined in other type definitions are moved
to their own type definitions.  For example, notice how the SEQUENCE
and CHOICE that are in type {\C T1} have been moved to the types
{\C T1Seq} and {\C T1Choice} in the C code.  This simplifies code
generation at the cost of introducing new types.

Identifiers for named numbers from INTEGER and ENUMERATED types and
named bits from the BIT STRING type are capitalized in the C
representation.  The ENUMERATED type maps to a C {\C enum} and the INTEGER
and BIT STRING named numbers/bits are handled with {\C \#define}
statements.

Most OPTIONAL elements of SEQUENCEs and SETs are referenced by
pointer.  An element is considered present if its pointer is non-NULL\@.
OCTET STRINGs, BIT STRINGs and OBJECT IDENTIFIERs are the exceptions,
and are included by value even when they are OPTIONAL because they are
small and contain an internal pointer that can be used to determine
their presence.  For an example of this, look at the first two
elements of type {\C T1}.  The INTEGER type is referenced by pointer
because it is OPTIONAL, but the OCTET STRING type is included
(non-pointer) in the {\C T1} type even though it is OPTIONAL\@.


\section{\label{encode-gen-C-section}Encode Routines}

Snacc generates two kinds of encoding routines. One is PDU oriented
and encodes the type's tag, length and content and the other
only encodes the type's content.  The generated encoders only call the
content encoders, except in the case of ANY and ANY DEFINED BY types.
Typically, you will only call the PDU oriented routines from your
code.

The content and PDU encoding routine interfaces are similar for all
ASN.1 types.  They both take two parameters, one is a buffer pointer
and the other is a pointer to the value to be encoded.  For example
the {\C T1} type from module EX1 has the following prototypes for
its encoding routines.

\begin{verbatim}
AsnLen BEncT1Content (BUF_TYPE b, T1 *v);
AsnLen BEncT1 (BUF_TYPE b, T1 *v);
\end{verbatim}

{\C BEnc} is short for ``BER Encode''. The {\C BUF\_TYPE}
parameter is the buffer to encode the value into and the {\C T1~*}
parameter is a pointer to the instance of the {\C T1} type that is to be
encoded.

The {\C BEncT1Content} routine only encodes the content of a {\C T1}
type and returns its encoded length; it does not encode its tag
(UNIVERSAL (CONSTRUCTED) 16 for SEQUENCE) or length.  The job of
encoding the tag and length is up to any type that encapsulates {\C T1}.
This design allows decisions about implicit tagging to be made at code
generation time instead of runtime, improving performance.  Also,
different encoding rules may fit into this model more easily.

The {\C BEncT1} routine encodes the tag (UNIVERSAL (CONSTRUCTED) 16
for SEQUENCE), length and content of a {\C T1} type and returns its encoded
length.  This is the PDU oriented routine and will only be generated
if the user designates the type as a PDU type via a compiler directive
or the type is used as
the content of an ANY or ANY DEFINED BY type (as indicated by an
OBJECT-TYPE macro).  A PDU type is a type that defines an entire PDU;
the user will typically be calling the encode and decode routine for
PDU types directly.  See Section~\ref{compiler-dir-C-section} for how to
designate PDU types with compiler directives.

The snacc encoders are somewhat strange; they encode a value starting
from the end of its BER representation and work back to its beginning.
This ``backwards'' encoding technique simplifies the use of definite
lengths on constructed values.  Other encoders that encode forwards,
such as those of CASN1, use an intermediate buffer format so that a
buffer containing the encoded length of a constructed value can be
inserted before its encoded content, after the content has been
encoded.  Use of intermediate buffers hurts performance.  Other
compilers' approaches have been to only encode indefinite lengths for
constructed values, however, this will not support some encoding rules
such as DER\@.  The drawback of encoding backwards is that BER values
cannot be written to stream-oriented connections as they are encoded.

Both definite and indefinite length encodings for constructed values'
lengths are supported.  Currently the choice is made when compiling
the generated code, via the {\C USE\_INDEF\_LEN} flag.  If both length
forms, definite and indefinite, are required, it easy to modify the
length encoding macros to check a global variable for the length
form to use.  For most types, using definite lengths produces smaller
encodings with little performance difference.

After calling an encode routine you should always check the buffer
you encoded into for a write error.  This is the only error reporting
mechanism used for the encoders.  See the C buffer section (Section
\ref{lib-buf-section}) for how to check a buffer for a write error.


\section{\label{decode-gen-C-section}Decode Routines}

Decoding routines are like the encoding routines in that there are two
kinds, one that decodes the type's tag, length and content and one
that only decodes the type's content.  As mentioned in the encoder
section, the content style interface allows implicit tagging decisions
to be made at compile time.

Unlike the encoding routines, the PDU and content decoding routines
take different arguments.  For the {\C T1} type the following would be
produced:
\begin{verbatim}
void BDecT1Content (BUF_TYPE b, AsnTag tagId0, AsnLen elmtLen0, T1 *v, AsnLen *bytesDecoded, ENV_TYPE env);
void BDecT1 (BUF_TYPE b, T1 *v, AsnLen *bytesDecoded, ENV_TYPE env);
\end{verbatim}

Notice that the content decoder, {\C BDecT1Content}, has tag and
length parameters that the PDU decoder, {\C BDecT1}, does not have.
Since the content decoder does not decode the tag and length on the
value, it is necessary to pass them in as parameters. Only OCTET
STRING and BIT STRING decoders will actually use the information
in the tag parameter.

The {\C BUF\_TYPE} parameter is the buffer that holds the BER value
being decoded.

The {\C tagId0} parameter is the last tag that was decoded on the
content of the type that is about to be decoded.  In the case of type
{\C T1}, {\C BDecT1Content} gets a tagId0 of UNIVERSAL (CONSTRUCTED) 16,
unless it is implicitly tagged by another type.  Most content decoding
routines ignore the tag information. OCTET STRING and BIT STRING
decoders use the tag information to determine whether the contents are
constructed or primitive.  CHOICE decoders use the tag information to
determine which CHOICE element is present.  CHOICE values are treated
differently, as will be explained shortly.

The {\C elmtLen0} parameter is the length of the content of the type
being decoded.  This is simply the length decoded from the buffer by
the containing type's decoder just before calling this decode routine.

The {\C v} parameter is a pointer to space allocated for the type
being decoded.  This memory is not allocated by the decoding routine
itself; this supports the cases where the type is enclosed in the
struct of the parent (i.\,e.\  no extra allocation is necessary).  If
the type to be decoded is referenced by pointer from its parent type,
the parent type's decoding routine must allocate the type.

The {\C bytesDecoded} parameter maintains the running total of the
number of octets that have been decoded.  For example, if I call
{\C BDecT1Content} with a {\C bytesDecoded} parameter that points
to 20 and the encoded length of the {\C T1} value is 30 octets,
{\C bytesDecoded} will point to 50 when {\C BDecT1Content}
returns.  Maintaining the length is vital to determining the presence
or absence of OPTIONAL elements in a SET or at the end of SEQUENCE\@.
Local variables are used to hold the lengths; there is no global stack
of lengths as with CASN1.

The {\C env} parameter is used in conjunction with {\C longjmp}
calls.  When an decoder encounters a fatal error such as a missing
tag, it uses the {\C env} with a {\C longjmp} call to pop back to the
initial decode call. Section~\ref{lib-err-C-section} has more error
management details.

CHOICEs are decoded a little differently from other types.  For all
types except CHOICEs, all of the tag and length pairs on the content
are decoded by the parent type, and the last pair a passed into to
content decoding routine via the {\C tagId0} and {\C elmtLen0}
parameters.  For CHOICEs, all of the tag and length pairs on the
content are decoded and then the first tag and length pair in the
CHOICE content is decoded by the parent and passed into the CHOICE
content decoding routine.  The first tag in a CHOICE's content is the
important tag by which the CHOICE determines which element is present.
This technique simplifies the code for dealing with untagged CHOICEs
embedded in other CHOICEs.  CHOICEs nested in this way mean that a
single tag determines which element is present in more than one
CHOICE\@.

The decoding routines allocate memory to hold the decoded value.  By
default snacc decoders use nibble memory (see Section
\ref{lib-mem-C-section}) which is very efficient in allocation and
virtually cost free for freeing.

To save memory, decoders generated by some other tools build values
that reference the data in the encoded PDU for types like OCTET
STRING\@.  Snacc decoded values do not reference the BER data in any way
for several reasons.  One, the encoded value may be held in some
bizarre buffer making access to the value difficult.  Two, with more
encoding rules being formalized, this technique may not always work
since the encoded format may be different from the desired internal
format.  Three, snacc decoders concatenate any constructed BIT and
OCTET STRINGs values when decoding, to simplify processing in the
application.

Snacc decoders can detect a variety of errors which will be reported
by {\C longjmp}.  Any tagging errors are reported. SETs must contain
all non-OPTIONAL components and SEQUENCEs must be in order and contain
all non-OPTIONAL components. Extra components in SETs and SEQUENCEs
are considered an error.  Errors will also be reported if you attempt
to decode values that exceed the limitations of the internal
representation (e.\,g.\ an integer that is larger than a
{\C long int} allows).

\section{\label{print-gen-C-section}Print Routines}
All of the generated print routines take similar parameters. For
example the {\C T1} type's print routine prototype is:
\begin{verbatim}
void PrintT1 (FILE *f, T1 *v, unsigned short int indent);
\end{verbatim}

The print routine writes the given value, {\C v}, to the given {\C FILE~*},
{\C f}.  The printed value is indented by {\C indent} spaces.  The
values are printed in an ASN.1 value notation style.  {\C PrintT1}
prints in the following style:
\begin{small}
\begin{verbatim}
{ -- SEQUENCE --
    17,
    '436c696d6220617420537175616d697368'H  -- "Climb at Squamish" --,
    0,
    { -- SEQUENCE OF --
        18,
        19
    },
    { -- SEQUENCE --
        id {2 40 29},
        value '736f6d6520737472696e67'H  -- "some string" --
    },
    20
}
\end{verbatim}
\end{small}

OCTET STRINGs are printed in a hexadecimal notation, and any printable
characters are included after the string in an ASN.1 comment.  Note
that the enumerated type value, 0, did not print its symbol, ``A''
from the ENUMERATED type.  It would be fairly easy to modify the C and
C++ back ends to generate print routines that printed the ENUMERATED
types' symbols instead of their values.

\section{\label{free-gen-C-section}Free Routines}

Snacc generates free routines of the form:
\begin{verbatim}
void FreeT1 (T1 *v);
\end{verbatim}

These routines will free all the components named type.
For example the above {\C FreeT1} routine will free all the
components of the given {\C T1} value, but not the {\C T1} value itself.  The
passed in pointer is not freed because it may be embedded in another
type which will be freed by another call to {\C Asn1Free}.  All the pieces
of memory are freed using the {\C Asn1Free} macro defined in
{\ufn asn-config.h}.  Each library type has its own free routine that
may call {\C Asn1Free}. The values are typically allocated during
decoding, using the {\C Asn1Alloc} macro.

The memory management can be changed by editing the {\ufn asn-config.h}
file to use you own memory management routines.  By default the memory
manager uses the nibble memory system described in Section
\ref{lib-mem-C-section}.  The nibble memory system does not need explicit
frees of each component so the generated free routines are not needed.
However, if you change the memory management to use something like
{\C malloc} and {\C free}, you should use the generated free routines.


\section{\label{val-gen-C-section}ASN.1 to C Value Translation}

C values will be produced for INTEGER, BOOLEAN and OBJECT IDENTIFIER
values.  C {\C extern} declarations for the value are put at the end
of the header file (after all of the type definitions).  The value
definitions are put at the beginning of the source file. For example,
the following will be produced for the EX1 module (at the end of
file ex1.h):

\begin{small}
\begin{verbatim}
extern AsnOid anOidVal;
extern AsnOid theSameOidVal;
extern AsnInt anIntVal;
extern AsnBool aBoolVal;
extern AsnInt foobar;
\end{verbatim}
\end{small}

(at the beginning of file ex1.c):

\begin{small}
\begin{verbatim}
AsnOid anOidVal = { 2, "\170\35" };
AsnOid theSameOidVal = { 2, "\170\35" };
AsnInt anIntVal = 1;
AsnBool aBoolVal = TRUE;
AsnInt foobar = 29;
\end{verbatim}
\end{small}

\section{\label{compiler-dir-C-section}Compiler Directives}

Snacc allows the user to control some aspects of the generated code by
inserting special comments in the ASN.1 source.  Warning! only the
{\ASN isPdu} directive has been tested to any extent. Use the others
very carefully and only if you really need to.  The compiler
directives have the form:

\begin{verbatim}
--snacc <attribute>:"<value>"  <attribute>:"<value>" ...
\end{verbatim}

The {\ASN attribute} is the name of one of the accepted attributes and
the {\ASN value} is what the {\ASN attribute}'s new value will be.
The attribute value pairs can be listed in a single {\ASN --snacc}
comment or spread out in a list of consecutive comments.

Compiler directives are only accepted in certain places in the ASN.1
code.  Depending on their location in the ASN.1 source, the compiler
directives affect type definitions or type references.  The directives
for type definitions and references are different.  Module level
compiler directives to specify output file names and other information
would be useful, but are not implemented.

Here is an example to present some of the compiler directives and
their uses.  Let's say your data structure always deals with
{\C PrintableStrings} that are null terminated (internally, not in
the encoding).  The default snacc string type is a structure that
includes a length and {\C char~*} for the string octets.  To change
the default type to a simple {\C char~*} the best way would be define
your own string type, let's say {\ASN MyString} as follows:

\begin{small}
\begin{verbatim}
Foo ::= SET
{
    s1 [0] MyString OPTIONAL,
    s2 [1] MyString,
    i1 [2] INTEGER
}

Bar ::= CHOICE
{
    s1 MyString,
    i1 INTEGER
}

Bell ::= MyString

MyString ::= --snacc isPtrForTypeDef:"FALSE"
             --snacc isPtrForTypeRef:"FALSE"
             --snacc isPtrInChoice:"FALSE"
             --snacc isPtrForOpt:"FALSE"
             --snacc optTestRoutineName:"MYSTRING_NON_NULL"
             --snacc genPrintRoutine:"FALSE"
             --snacc genEncodeRoutine:"FALSE"
             --snacc genDecodeRoutine:"FALSE"
             --snacc genFreeRoutine:"FALSE"
             --snacc printRoutineName:"printMyString"
             --snacc encodeRoutineName:"EncMyString"
             --snacc decodeRoutineName:"DecMyString"
             --snacc freeRoutineName:"FreeMyString"
             PrintableString --snacc cTypeName:"char *"
\end{verbatim}
\end{small}

All but the last {\ASN --snacc} comment bind with the {\ASN MyString} type
definition. The last directive comment binds with the {\ASN PrintableString}
type.  The C data structure resulting from the above ASN.1 and compiler
directives is the following:
\begin{small}
\begin{verbatim}
typedef char *MyString; /* PrintableString */

typedef struct Foo /* SET */
{
    MyString s1; /* [0] MyString OPTIONAL */
    MyString s2; /* [1] MyString */
    AsnInt i1; /* [2] INTEGER */
} Foo;

typedef struct Bar /* CHOICE */
{
    enum BarChoiceId
    {
        BAR_S1,
        BAR_I1
    } choiceId;
    union BarChoiceUnion
    {
        MyString s1; /* MyString */
        AsnInt i1; /* INTEGER */
    } a;
} Bar;

typedef MyString Bell; /* MyString */
\end{verbatim}
\end{small}

The compiler directives used on the {\ASN MyString} type have some
interesting effects.  Notice that {\ASN MyString} is not referenced by
pointer in the CHOICE, SET, or type definition, {\ASN Bell}.

The generated code for encoding field {\C s1} of {\C Foo} type
will use the code\linebreak``{\C MYSTRING\_NON\_NULL (\&fooVal-->s1)}'' to check
for the presence of the OPTIONAL {\C s1} field.  The code associated
with MYSTRING\_NON\_NULL should return TRUE if the {\C s1} field
value is present and might look like:
\begin{verbatim}
#define MYSTRING_NON_NULL(s) (*s != NULL)
\end{verbatim}

The argument to {\C optTestRoutine} routine will be a pointer to the
field type's defining type.  Note that in the above example,
{\ASN MyString} is a {\C char~*}, therefore the {\C MYSTRING\_NON\_NULL}
macro's argument will be a {\C char~**}.

Setting the {\ASN genPrintRoutine} etc. attributes to false makes
snacc not define or generate any encode, decode, print, or free
routines for the {\ASN MyString} type.  You must provide these
yourself; the best approach is to take the normal {\ASN PrintableString}
routines and modify them to handle your special string type.

The names of the encode, decode, print and free routines used for the
{\ASN MyString} type will be based on the ones given with the
{\ASN printRoutineName} etc.\  attributes.  Snacc will prepend a
``B'' (for BER) and append a ``Content'' to the encode and decode
routines names, so you must provide the {\C BEncMyStringContent} and
{\C BDecMyStringContent} routines.  You may also need the
{\C BEncMyString} and {\C BDecMyString} routines if {\ASN MyString} is a
PDU type or used in an ANY or ANY DEFINED type.

The {\ASN PrintableString} type has its C type name changed to
{\C char~*} by the last compiler directive.  Thus {\ASN MyString} is defined
as a {\C char~*}.  This directive applies to the {\ASN PrintableString}
type reference.  Note that these directives do not affect the tags or
the encoded representation of the {\ASN MyString} type

The location of the {\ASN --snacc} comment(s) is important.
{\ASN --snacc} comment(s) between the {\ASN ::=} sign and the
following type are associated with the type being defined. Any
compiler directives after the type and before the next type or value
definition are associated with the type.  Fields in SETs, SEQUENCEs
and CHOICEs can be modified by putting the compiler directive after
the comma that follows the field type that you wish to modify.  In the
case of the last element of one of these types, where there is no
comma, just place it after the field and before the closing bracket of
the parent type.

Attributes shadow the type attributes filled in during the target
language type information generation pass of the compiler.  The type
definition attributes are:

\begin{description}
\item[cTypeName] { this is the type name that the generated type will
have. Its value can be any string that is valid as a C type name.}

\item[isPdu]     { whether this is a PDU type.  A PDU type will have
extra interfaces to the encode and decode routines generated. Its
value can be ``TRUE'' or ``FALSE''}

\item[isPtrForTypeDef] { TRUE if other types defined solely by this type
definition are defined as a pointer to this type. Its
value can be ``TRUE'' or ``FALSE''.}

\item[isPtrForTypeRef]{ TRUE if type references to this type
definition from a SET or SEQUENCE are by pointer. Its
value can be ``TRUE'' or ``FALSE''.}

\item[isPtrInChoice] {TRUE if type references to this type definition
from a CHOICE are by pointer. Its value can be ``TRUE'' or ``FALSE''.}


\item[isPtrForOpt]  { TRUE if OPTIONAL type references to this type
definition from a SET or SEQUENCE are by pointer. Its value can be
``TRUE'' or ``FALSE''.}

\item[optTestRoutineName] {name of the routine to test whether an
OPTIONAL element of this type in a SET or SEQUENCE is present. The
routine should return TRUE if the element is present.  The value of
this field is usually just the name of a C macro that tests for NON-NULL\@.
The argument to the routine will be a pointer to the type definition's
type.  The optTestRoutineName value can be any string value.}

\item[defaultFieldName] { if this type is used in a SET, SEQUENCE or
CHOICE without a field name then this value is used with a digit
appended to it.  Its value can be any string that is a valid C field
name in a struct or union.}
\item[printRoutineName] { name of this type definition's printing
routine.  Its value can be any string that is a C function or
macro name.}
\item[encodeRoutineName]{ name of this type definition's encoding
routine. Its value can be any string that is a C function or
macro name.}
\item[decodeRoutineName]{ name of this type definition's decoding
routine. Its value can be any string that is a C function or
macro name.}
\item[freeRoutineName]  { name of this type definition's freeing
routine. Its value can be any string that is a C function or
macro name.}

\item[isEncDec] {If this type is used in a SET or SEQUENCE then it is not
encoded or decoded.  Its value can be ``TRUE'' or ``FALSE''.  This is
handy for adding your own types to a standard that are only for local
use, and are not included in encoded values.}

\item[genTypeDef] { TRUE if you want a C type to be generated for this
type definition. Its values can be ``TRUE'' or ``FALSE''.}

\item[genPrintRoutine] { TRUE if you want a printing routine to be
generated for this type definition. Its values can be ``TRUE'' or
``FALSE''.}
\item[genEncodeRoutine] { TRUE if you want an encoding routine to be
generated for this type definition. Its values can be ``TRUE'' or
``FALSE''.}
\item[genDecodeRoutine] { TRUE if you want a decoding routine to be
generated for this type definition. Its values can be ``TRUE'' or
``FALSE''.}
\item[genFreeRoutine] { TRUE if you want a free routine to be
generated for this type definition. Its values can be ``TRUE'' or
``FALSE''.}
\end{description}


The type reference attributes are slightly different from the type
definition attributes due to the semantic differences between a type
definition and a type reference.  Type references will inherit some of
their attributes from the referenced type definition.  The following
are the valid type reference attributes:
\begin{description}
\item[cTypeName] { this is the type name that the generated type will
have. Its value can be any string that is valid as a C type name.}

\item[cFieldName] { if this is a field in a CHOICE, SET or SEQUENCE
then this holds the C field name for this reference. Its value can be
any string that is valid as a C field name.}

\item[isPtr] { TRUE if this is a pointer to the type named by
cTypeName.  This is usually determined from the referenced type
definitions attributes. Its value can be ``TRUE'' or ``FALSE''.}

\item[optTestRoutineName] {if this field is an OPTIONAL component then
this is the name of the routine to test whether it is present.  The
routine should return TRUE if the element is present.  The value of
this is usually just the name of a C macro that tests for NULL\@. The
argument to the routine will be a pointer to the type definition's
type.  The optTestRoutineName value can be any string value.}

\item[printRoutineName] { name of this type reference's printing
routine.  This and the other routine name attributes are useful for
special instances of the referenced type.  It is easier to modify the
referenced type definition if you want every instance of this type to
use a certain print etc.\  routine.  Its value can be any string that is
a value C function or macro name.}

\item[encodeRoutineName]{ name of this type reference's encoding
routine. Its value can be any string that is a function or
macro name.}

\item[decodeRoutineName]{ name of this type reference's decoding
routine. Its value can be any string that is a C function or
macro name.}

\item[freeRoutineName]  { name of this type reference's freeing
routine. Its value can be any string that is a C function or
macro name.}

\item[isEncDec] { If this type is used in a SET or SEQUENCE then the
field is not encoded or decoded.  Its value can be ``TRUE'' or
``FALSE''.  This is handy for adding your own types to a standard that
are only for local use, and are not included in encoded values.}

\item[choiceIdSymbol] {if this is a component of a CHOICE, this string
attribute will be the defined/enum symbol whose value in the choiceId
field indicates the presence of this field.}
\item[choiceIdValue]  {if this is a component of a CHOICE, this integer
attribute will be the value associated with the symbol in choiceIdSymbol.}

\end{description}




\section{\label{compiling-gen-C-section}Compiling the Generated C Code}

The generated C code (and libraries) can be compiled by both ANSI and K\&R C compilers.
C function prototypes use the {\C PROTO} macro and C function declarations use the {\C PARAMS} macro.
These macros are defined in {\ufn \dots/snacc.h} and their definitions depend on whether the {\C \_\_USE\_ANSI\_C\_\_} flag has been defined in {\ufn \dots/config.h}.

When compiling the generated C code you will need:
\begin{enumerate}
\item
  The include directory where the files from {\ufn \dots/c-lib/inc/} have been installed into in your include path so the C sources can include the library header files.
  The header files should be included with statements like {\C \#include <snacc/c/asn-incl.h>} and your C compiler should be supplied with {\ufn -I/usr/local/include} in case snacc got installed under {\ufn /usr/local/}.
\item
  to link with the correct C ASN.1 runtime library, depending on the buffer type you choose.
  In case snacc got installed under {\ufn /usr/local/}, your linker may need to be supplied with {\ufn -L/usr/local/lib} and one of {\ufn -lasn1cebuf}, {\ufn -lasn1cmbuf} or {\ufn -lasn1csbuf} as arguments.
\item
  to link with the math library ({\ufn -lm}), since the ASN.1 REAL type's encode and decode routine use some math routines.
\end{enumerate}

See the example in {\ufn \dots/c-examples/simple/} for a complete
example. The makefile and main routines are probably the most
important.  There are several other examples in the
{\ufn \dots/c-examples/} directory.

% file: .../doc/c-lib.tex

% $Header: /usr/app/odstb/CVS/snacc/doc/c-lib.tex,v 1.1 1997/01/01 22:47:34 rj Exp $
% $Log: c-lib.tex,v $
% Revision 1.1  1997/01/01 22:47:34  rj
% first check-in
%

\chapter{\label{lib-C-chapter}C ASN.1 Library}
\section{\label{libover-C-section}Overview}

Each library type has a file in the {\ufn \dots/c-lib/src/} and
{\ufn \dots/c-lib/inc/} directories.  Each source file contains the encode,
decode, free and print routines for the given type. This chapter
contains a description of each library type and its routines.
This library is also referred to as the runtime library.

After installing Snacc, you should test the library types to make sure
that they are encoding and decoding properly.  Use the
{\ufn \dots/c-examples/test-lib/} example to check them.

In addition to other errors, most decoding routines will report an
error if they attempt to read past the end of the data.  Be aware that
some buffer types do not support this type of checking.  This is
explained more in the buffer management section.

\section{\label{tag-C-section}Tags}

Snacc's tag representation was motivated by several things.
\begin{enumerate}
\item the tags must be easy to compare for equality in {\C if} and {\C switch} statements to make tag-based decisions cheap.
\item a tag must be cheap to decode.
\item a tag must be cheap to encode.
\end{enumerate}

The first requirement meant that tags had to be integer types (for the
{\C switch} statement).  The representation of the tag within the integer
was set by the second requirement.

The best way to decode cheaply is minimize the transformation between
the encoded and decoded (internal) format.  So the four (can be set-up
for two) bytes of the long integer are used to hold the encoded tag,
starting with the first octet of the tag in the most significant byte
of the integer and the rest (if any) following.  Any unused (always
trailing) bytes in the integer are zero.  This limits the
representable tag code to less than $2^{21}$ but for reasonable ASN.1
specifications this should not be a problem.

To meet the third requirement the decoded tag representation was
bypassed entirely by using macros ({\C BEncTag1()} etc.) that
write the encoded tag octet(s) to the buffer. The writing of an
encoded tag octet involves bit shifting, bitwise ands and bitwise ors
with constant values; most optimizing C compilers can compute these at
compile time.  This simplifies encoding a tag to writing some constant
byte value(s) to the buffer.

The following excerpt from {\ufn \dots/c-lib/inc/asn-tag.h} shows some
of the tag routines.
\begin{small}
\begin{verbatim}
typedef unsigned long int AsnTag;

#define MAKE_TAG_ID( class, form, code) ...
#define TAG_IS_CONS( tag) ...

#define BEncTag1( b, class, form, code) ...
#define BEncTag2( b, class, form, code) ...
#define BEncTag3( b, class, form, code) ...
#define BEncTag4( b, class, form, code) ...
#define BEncTag5( b, class, form, code) ...

AsnTag BDecTag (BUF_TYPE b, AsnLen *bytesDecoded, ENV_TYPE env);
\end{verbatim}
\end{small}

The generated decode routines use the {\C BDecTag} to decode a tag
from the buffer.  The returned tag value is either used in an
{\C if} expression or as the argument to {\C switch} statements.
The {\C MAKE\_TAG\_ID} macro is used to make a tag for comparison to
the one returned by {\C BDecTag}. The {\C MAKE\_TAG\_ID} is used is
{\C switch} statement case labels and in {\C if} statements.

Most of the time tags are only compared for equality, however, the
OCTET STRING and BIT STRING decoders check the constructed bit in the
tag using the {\C TAG\_IS\_CONS} macro.

The {\C BEncTag} macros are quite fragile because they return the
encoded length of the tag; they cannot be treated as a single
statement.  This requires careful use of braces when using them in
your own code in places such as the sole statement in an {\C if}
statement.  This ugliness is caused by the difficulty in returning
values from multi-line macros (macros are used for performance here
since encoding tags can be a significant part of BER encoding).

The {\C BDecTag} routine will report an error via {\C longjmp} if
the encoded tag is longer than can be held in the {\C AsnTag} type
or if it read past the end of the data when decoding the tag.

\section{\label{len-C-section}Lengths}

Decoded lengths are represented by unsigned long integers, with the
maximum value indicating indefinite length.

Snacc users can choose between using only indefinite or only definite
lengths when encoding constructed values' lengths when compiling the
generated code.  Of course, the generated decoders can handle both
forms.  Define the {\C USE\_INDEF\_LEN} symbol when compiling the
generated code if you want to use indefinite lengths when encoding
constructed values.  Primitive values are always encoded with definite
lengths as required by the standard; this is necessary to avoid
confusion between a value's content and the End-Of-Contents marker.

There is no loss of performance when using definite lengths with snacc
encoders.  This is due the ``backwards'' encoding as described in
Section~\ref{encode-gen-C-section}.  The schemes used by other compilers'
encoders to handle definite lengths may hurt performance.

Most of the routines in the following code are obvious except for
{\C BEncDefLenTo127()}.  This is used instead of {\C BEncDefLen}
in the generated code when the compiler knows the value being encoded
will not be over 127 octets long.  Values such as BOOLEANs,
INTEGERs, and REALs are assumed to be shorter than 127 octets
(constraints on the decoded representation of INTEGERs and REALs make
this valid).
\begin{small}
\begin{verbatim}
typedef unsigned long int AsnLen;

/* max unsigned value - used for internal rep of indef len */
#define INDEFINITE_LEN          ~0L

#ifdef USE_INDEF_LEN
#define BEncEocIfNec( b)        BEncEoc (b)
#define BEncConsLen(b, len)     2 + BEncIndefLen (b)
#else
#define BEncEocIfNec( b)
#define BEncConsLen( b, len)    BEncDefLen (b, len)
#endif

#define BEncIndefLen( b) ...
#define BEncDefLenTo127( b, len) ...
AsnLen  BEncDefLen (BUF_TYPE b, AsnLen len);
AsnLen  BDecLen (BUF_TYPE b, AsnLen *bytesDecoded, ENV_TYPE env);

#define BEncEoc( b) ...
#define BDEC_2ND_EOC_OCTET( b, bytesDecoded, env) ...
void    BDecEoc (BUF_TYPE b, AsnLen *bytesDecoded, ENV_TYPE env);
\end{verbatim}
\end{small}

The {\C BDecLen} routine will report an error via {\C longjmp} if
it attempts to read past the end of the data or the decoded length is
too large to be held in the {\C AsnLen} representation.
{\C BDecEoc} will report an error if it attempts to read past the
end of the data or one of the EOC (End-Of-Contents) octets is
non-zero.

\section{\label{bool-C-section}BOOLEAN}

The BOOLEAN type is represented by an {\C unsigned char}.  It has
the following routines for manipulating it.
\begin{small}
\begin{verbatim}
typedef unsigned char AsnBool;

AsnLen BEncAsnBool (BUF_TYPE b, AsnBool *data);
void   BDecAsnBool (BUF_TYPE b, AsnBool *result, AsnLen *bytesDecoded,
                   ENV_TYPE env);

AsnLen BEncAsnBoolContent (BUF_TYPE b, AsnBool *data);
void   BDecAsnBoolContent (BUF_TYPE b, AsnTag tag, AsnLen len,
                          AsnBool *result, AsnLen *bytesDecoded,
                          ENV_TYPE env);

#define FreeAsnBool( v)
void PrintAsnBool (FILE *f, AsnBool *b, unsigned short int indent);
\end{verbatim}
\end{small}

As discussed in Sections \ref{encode-gen-C-section} and \ref{decode-gen-C-section},
{\C BEncAsnBool} and {\C BDecAsnBool} encode/decode the UNIVERSAL
tag, length and content of the given BOOLEAN value.  The\linebreak {\C BEncAsnBoolContent} and {\C BDecAsnBoolContent} routine only
encode/decode the content of the given BOOLEAN value.

The {\C FreeAsnBool} routine does nothing since the BOOLEAN type
does not contain pointers to data; the free routine generator does not
have to check which types need freeing and simply calls the type's
free routine.  It also allows the user to modify the types and their
free routines without changing the free routine generator.  However,
the ANY and ANY DEFINED BY type hash table initialization routine
generator does need to know which types have empty free routines
because the hash entries contain pointers to the free functions (NULL
is used for the empty free functions like {\C FreeAsnBool}).  The
INTEGER, NULL, REAL and ENUMERATED types have empty free routines for
the same reason.

{\C BDecAsnBool} will report an error if the tag is not
UNIVERSAL-PRIM-1.  {\C BDecAsnBoolContent} will report an error if it
decodes past the end of the data or the length of the encoded value
(given by the {\C len} parameter) is not exactly one octet.

\section{\label{int-C-section}INTEGER}

The INTEGER type is represented by a 32 bit integer type, {\C AsnInt}.
The C integer type chosen depends on the machine and compiler and may be {\C int}, {\C long} or {\C short}, whatever is 32 bits in size.
If you are using INTEGER types that are only positive (via subtyping or
protocol definition) you may want to use the {\C UAsnInt} and
associated routines that use the unsigned int for a larger positive value range.
\begin{small}
\begin{verbatim}
typedef int AsnInt;
typedef unsigned int UAsnInt;

AsnLen BEncAsnInt (BUF_TYPE b, AsnInt *data);
void BDecAsnInt (BUF_TYPE b, AsnInt *result, AsnLen *bytesDecoded,
                ENV_TYPE env);

AsnLen BEncAsnIntContent (BUF_TYPE b, AsnInt *data);
void BDecAsnIntContent (BUF_TYPE b, AsnTag tag, AsnLen elmtLen,
                       AsnInt  *result, AsnLen *bytesDecoded,
                       ENV_TYPE env);

#define FreeAsnInt( v)
void PrintAsnInt (FILE *f, AsnInt *v, unsigned short int indent);

AsnLen BEncUAsnInt (BUF_TYPE b, UAsnInt *data);
void BDecUAsnInt (BUF_TYPE b, UAsnInt *result, AsnLen *bytesDecoded,
                 ENV_TYPE env);

AsnLen BEncUAsnIntContent (BUF_TYPE b, UAsnInt *data);
void BDecUAsnIntContent (BUF_TYPE b, AsnTag tagId, AsnLen len,
                        UAsnInt *result, AsnLen *bytesDecoded,
                        ENV_TYPE env);

#define FreeUAsnInt( v)
void PrintUAsnInt (FILE *f, UAsnInt *v, unsigned short int indent);
\end{verbatim}
\end{small}

{\C BDecAsnInt} will report an error if the tag is not
UNIVERSAL-PRIM-2.  {\C BDecAsnIntContent} will report an error if it
decodes past the end of the data or the integer value is too large for
an {\C AsnInt}.

\section{\label{null-C-section}NULL}

The NULL type is represented by the {\C AsnNull} type.  Its content
is always empty and hence its encoded length always is zero.
\begin{small}
\begin{verbatim}
typedef char AsnNull;

AsnLen BEncAsnNull (BUF_TYPE b, AsnNull *data);
void BDecAsnNull (BUF_TYPE b, AsnNull *result, AsnLen *bytesDecoded,
                 ENV_TYPE env);

/* 'return' length of encoded NULL value, 0 */
#define BEncAsnNullContent(b, data) 0
void BDecAsnNullContent (BUF_TYPE b, AsnTag tag, AsnLen len,
                        AsnNull *result, AsnLen *bytesDecoded,
                        ENV_TYPE env);

#define FreeAsnNull( v)
void PrintAsnNull (FILE *f, AsnNull * b, unsigned short int indent);
\end{verbatim}
\end{small}

\section{\label{real-C-section}REAL}

The REAL type is represented by {\C AsnReal}, a double.  This type's
representation can depend on the compiler or system you are using so
several different encoding routines are provided.
Even so, you may need to modify the code.

If you are using the REAL type in your ASN.1 modules, you should call the
{\C InitAsnInfinity()} routine to setup the {\C PLUS\_INFINITY}
and {\C MINUS\_INFINITY} values.

There are three encode routines included and they can be selected by
defining one of {\C IEEE\_REAL\_FMT}, {\C IEEE\_REAL\_LIB} or nothing.
Defining {\C IEEE\_REAL\_FMT} uses the encode routine that assumes the
double representation is the standard IEEE double \cite{68881}.
Defining {\C IEEE\_REAL\_LIB} uses the encode routine that assumes the
IEEE functions library (isinf, scalbn, signbit etc.\ ) is available.
If neither are defined, the default encode routine uses {\C frexp}.

There is only one content decoding routine and it builds the value
through multiplication and the {\C pow} routine (requires the math
library).  The content decoding routine only supports the binary
encoding of a REAL, not the decimal encoding.

\begin{small}
\begin{verbatim}
typedef double AsnReal;

extern AsnReal PLUS_INFINITY;
extern AsnReal MINUS_INFINITY;

void InitAsnInfinity();
AsnLen BEncAsnReal (BUF_TYPE b, AsnReal *data);
void BDecAsnReal (BUF_TYPE b, AsnReal *result, AsnLen *bytesDecoded,
                       ENV_TYPE env);

AsnLen BEncAsnRealContent (BUF_TYPE b, AsnReal *data);
void BDecAsnRealContent (BUF_TYPE b, AsnTag tag, AsnLen len,
                               AsnReal *result, AsnLen *bytesDecoded,
                               ENV_TYPE env);

/* do nothing */
#define FreeAsnReal( v)
void PrintAsnReal (FILE *f, AsnReal *b, unsigned short int indent);
\end{verbatim}
\end{small}

{\C BDecAsnReal} will report an error if the value's tag is not UNIVERSAL-PRIM-9.
{\C BDecAsnRealContent} will report an error if the base is not supported or the decimal type REAL encoding is received.


\section{\label{bits-C-section}BIT STRING}

The BIT STRING type is represented by the {\C AsnBits} structure.  It
contains a pointer to the bits and integer that holds the length
in bits of the BIT STRING\@.

In addition to the standard encode, decode, print and free routines,
there are some other utility routines.  {\C AsnBitsEquiv} returns
TRUE if the given BIT STRINGs are identical.  The {\C SetAsnBit},
{\C ClrAsnBit} and {\C GetAsnBit} are routines for writing and
reading a BIT STRING value.

You may notice that the AsnBits type does not have any means of
handling linked pieces of BIT STRINGs. Some ASN.1 tools use lists of
structures like {\C AsnBits} to represent BIT STRINGs.  This is done
because, as you should be aware, BIT STRINGs can be encoded in a
nested, constructed fashion.  The snacc BIT STRING decoder attempts to
save you the hassle of dealing with fragments of BIT STRINGs by
concatenating them in the decoding step.  Every BIT STRING value
returned by the decoder will have contiguous bits.

Some people contend that fragmented BIT STRINGs are necessary to
support systems that lack enough memory to hold the entire value.
Snacc encodes value ``backwards'' so the entire value must be encoded
before it can be sent, thus you must have enough memory to hold the
whole encoded value.  If the fragmented representation is useful to
your protocol implementation for other reasons, it should be fairly
simple to modify the BIT STRING routines.  Remember, no significance
should be placed on where constructed BIT STRING values are fragmented.

Snacc uses a table to hold pointers to the BIT STRING fragments in the
buffer while it is decoding them. Once the whole BIT STRING value has
been decoded, a block of memory that is large enough to hold the
entire BIT STRING is allocated and the fragments are copied into it.
The table initially can hold pointers to 128 fragments.  If more table
entries are needed the stack will grow via {\C realloc} (with
associated performance loss) and will not shrink after growing.  If
you wish to modify this behaviour, change the
{\ufn \dots/c-lib/inc/str-stk.h} file.

The {\C FreeAsnBits} routine will free memory referenced by the
{\C bits} pointer.

\begin{small}
\begin{verbatim}
typedef struct AsnBits
{
  int   bitLen;
  char  *bits;
} AsnBits;

extern char numToHexCharTblG[];
#define TO_HEX( fourBits)        (numToHexCharTblG[(fourBits) & 0x0f])
#define ASNBITS_PRESENT( abits)  ((abits)->bits != NULL)

AsnLen BEncAsnBits (BUF_TYPE b, AsnBits *data);
void BDecAsnBits (BUF_TYPE b, AsnBits *result, AsnLen *bytesDecoded,
                 ENV_TYPE env);

AsnLen BEncAsnBitsContent (BUF_TYPE b, AsnBits *bits);
void BDecAsnBitsContent (BUF_TYPE b, AsnLen len, AsnTag tagId,
                        AsnBits *result, AsnLen *bytesDecoded,
                        ENV_TYPE env);

void FreeAsnBits (AsnBits *v);
void PrintAsnBits (FILE *f, AsnBits *b, unsigned short int indent);

int AsnBitsEquiv (AsnBits *b1, AsnBits *b2);
void SetAsnBit (AsnBits *b1, unsigned long int bit);
void ClrAsnBit (AsnBits *b1, unsigned long int bit);
int GetAsnBit (AsnBits *b1, unsigned long int bit);
\end{verbatim}
\end{small}

{\C BDecAsnBits} will report an error if the tag is not UNIVERSAL-CONS-3 or UNIVERSAL-PRIM-3.
When decoding constructed BIT STRING BER values, an error will be reported if a component other than the last one has non-zero unused bits in its last octet or an internal component does not have the UNIVERSAL-3 tag.
If the decoder attempts to read past the end of the data an error will be reported.


\section{\label{octets-C-section}OCTET STRING}
The OCTET STRING type is represented by the {\C AsnOcts} structure.
It contains a pointer to the octets and an integer that holds the length in octets of the OCTET STRING\@.

As with BIT STRINGs, OCTET STRINGs can have constructed values.  These
are handled in the same way as the constructed BIT STRING values. The
decoded representation of an OCTET STRING is always contiguous.

The {\C FreeAsnOcts} routine will free the memory referenced by the
{\C octs} pointer.  The {\C AsnOctsEquiv} routine will return TRUE
if the given OCTET STRINGs are identical.

\begin{small}
\begin{verbatim}
typedef struct AsnOcts
{
  unsigned long int  octetLen;
  char               *octs;
} AsnOcts;

#define ASNOCTS_PRESENT( aocts)  ((aocts)->octs != NULL)

AsnLen BEncAsnOcts (BUF_TYPE b, AsnOcts *data);

void BDecAsnOcts (BUF_TYPE b, AsnOcts *result, AsnLen *bytesDecoded,
                 ENV_TYPE env);

AsnLen BEncAsnOctsContent (BUF_TYPE b, AsnOcts *octs);
void BDecAsnOctsContent (BUF_TYPE b, AsnLen len, AsnTag tagId,
                         AsnOcts *result, AsnLen *bytesDecoded,
                         ENV_TYPE env);

void FreeAsnOcts (AsnOcts *o);
void PrintAsnOcts (FILE *f, AsnOcts *o, unsigned short int indent);

int AsnOctsEquiv (AsnOcts *o1, AsnOcts *o2);
\end{verbatim}
\end{small}

{\C BDecAsnOcts} will report an error if the tag is not
UNIVERSAL-CONS-4 or UNIVERSAL-PRIM-4.  When decoding constructed OCTET
STRING BER values, an error will be reported if an internal component
does not have the UNIVERSAL-4 tag. If the decoder attempts to read
past the end of the data an error will be reported.


\section{\label{oid-C-section}OBJECT IDENTIFIER}

In snacc, OBJECT IDENTIFIERs are kept in their encoded form to improve
performance.  The {\C AsnOid} type is defined as {\C AsnOcts}, as
it holds the octets of the encoded OBJECT IDENTIFIER\@.  It seems that
the most common operation with OBJECT IDENTIFIERs is to compare for
equality, for which the encoded representation (which is canonical)
works well.

There is a linked OBJECT IDENTIFIER representation called {\C OID}
and routines to convert it to and from the {\C AsnOid} format, but it
should not be used if performance is an issue.

Since the OBJECT IDENTIFIERs are represented {\C AsnOcts}, the
{\C AsnOcts} content encoding routine can be used for the
{\C AsnOid} content encoding routine.  The other {\C AsnOcts}
encoding and decoding routines cannot be used because the OBJECT
IDENTIFIER has a different tag and cannot be encoded in a constructed
fashion.

An OBJECT IDENTIFIER must have a minimum of two arc numbers but the
decoding routines do not check this.

\begin{small}
\begin{verbatim}
typedef AsnOcts AsnOid;

#define ASNOID_PRESENT( aoid)  ASNOCTS_PRESENT (aoid)

AsnLen BEncAsnOid (BUF_TYPE b, AsnOid *data);
void BDecAsnOid (BUF_TYPE b, AsnOid *result, AsnLen *bytesDecoded,
                ENV_TYPE env);

#define BEncAsnOidContent(b, oid)  BEncAsnOctsContent(b, oid)
void BDecAsnOidContent (BUF_TYPE b, AsnTag tag, AsnLen len,
                       AsnOid  *result, AsnLen *bytesDecoded,
                       ENV_TYPE env);

#define FreeAsnOid  FreeAsnOcts
void PrintAsnOid (FILE *f, AsnOid *b, unsigned short int indent);

#define AsnOidsEquiv( o1, o2)   AsnOctsEquiv (o1, o2)
\end{verbatim}
\end{small}


\section{\label{list-C-section}SET OF and SEQUENCE OF}

The SET OF and SEQUENCE OF type are represented by the {\C AsnList}
structure.  An {\C AsnList} consists of a head object that has
pointers to the first, current and last nodes and the current number
of nodes in the list.  Each list node has a pointer to its next and
previous list member and the node's data.  The first list node's
previous pointer is always NULL and the last list node's next pointer
is always NULL\@.

Each SET OF or SEQUENCE OF type is defined as an {\C AsnList}, so the
element type information (kept via a {\C void~*}) is not kept,
therefore, the {\C AsnList} type is not type safe.

The {\C AsnList} is a doubly linked list to simplify ``backwards''
encoding.  The reverse link allows the list to be traversed in reverse
so the components can be encoded from last to first.

Initially, the lists were designed to allow the list element itself to
be contained in the list node (hence the {\C elmtSize} parameter to
the AsnListNew() routine).  The design eventually changed such that
every list element was reference by pointer from the list node.

A small problem with the {\C AsnListNew} routine is the memory
allocation.  Since it is used by the decoding routines to allocate new
lists, it uses whatever memory management you have setup with the
{\C Asn1Alloc} macro (see Section~\ref{lib-mem-C-section}).  This may not be
desirable when building values to be transmitted.  You may need to
provide another AsnListNew routine that uses a different allocation
scheme to solve this.

\begin{small}
\begin{verbatim}
typedef struct AsnListNode
{
    struct AsnListNode *prev;
    struct AsnListNode *next;
    void            *data; /* this must be the last field of this structure  */
} AsnListNode;

typedef struct AsnList
{
    AsnListNode *first;
    AsnListNode *last;
    AsnListNode *curr;
    int        count;    /* number of elements in list               */
    int        dataSize; /* space required in each node for the data */
} AsnList;

#define FOR_EACH_LIST_ELMT( elmt, list) ...
#define FOR_EACH_LIST_ELMT_RVS( elmt, list) ...
#define FOR_REST_LIST_ELMT( elmt, al) ...

#define CURR_LIST_ELMT( al)    (al)->curr->data
#define NEXT_LIST_ELMT( al)    (al)->curr->next->data
#define PREV_LIST_ELMT( al)    (al)->curr->prev->data
#define LAST_LIST_ELMT( al)    (al)->last->data
#define FIRST_LIST_ELMT( al)   (al)->first->data
#define LIST_EMPTY(al) (( al)->count == 0)

#define CURR_LIST_NODE( al) ((al)->curr)
#define FIRST_LIST_NODE( al) ((al)->first)
#define LAST_LIST_NODE( al) ((al)->last)
#define PREV_LIST_NODE( al) ((al)->curr->prev)
#define NEXT_LIST_NODE( al) ((al)->curr->next)
#define SET_CURR_LIST_NODE( al, listNode)  ((al)->curr = (listNode))

void  AsnListRemove (AsnList *l);
void *AsnListAdd (AsnList *l);
void *AsnListInsert (AsnList *list);
void  AsnListInit (AsnList *list, int dataSize);
AsnList *AsnListNew (int elmtSize);
void *AsnListPrev (AsnList *);
void *AsnListNext (AsnList *);
void *AsnListLast (AsnList *);
void *AsnListFirst (AsnList *);
void *AsnListPrepend (AsnList *);
void *AsnListAppend (AsnList *);
void *AsnListCurr (AsnList *);
int   AsnListCount (AsnList *);
AsnList *AsnListConcat (AsnList *, AsnList *);
\end{verbatim}
\end{small}

There are a number of macros for dealing with the list type, the
most important being the list traversal macros.  The
{\C FOR\_EACH\_LIST\_ELMT} macro acts like a ``for'' statment that
traverses forward through the list.  The first parameter should be a
pointer to the list element type that will be used to hold the current list
element for each iteration of the ``for'' loop.  The second parameter is
the list of elements that you wish to traverse.

The {\C FOR\_EACH\_LIST\_ELMT\_RVS} macro is identical to the
{\C FOR\_EACH\_LIST\_ELMT} macro except that is moves from the back of
the list to the front.   The {\C FOR\_REST\_LIST\_ELMT} macro is
similar to the other two but it does not reset the {\C curr} pointer
in the {\C AsnList} type.  This has the effect of iterating from the
current element to the end of the list.   Look in the generated code
for a better indication of how to use these macros.  The other macros
are straight forward.


\section{\label{any-C-section}ANY and ANY DEFINED BY}


The ANY and ANY DEFINED BY type are classically the most irritating
ASN.1 types for compiler writers.  They rely on mechanisms outside of
ASN.1 to specify what types they contain.  The 1992 ASN.1 standard has
rectified this by adding much stronger typing semantics and eliminating
macros.

The ANY DEFINED BY type can be handled automatically by {\em snacc} if
the SNMP OBJECT-TYPE \cite{snmp} macro is used to specify the
identifier value to type mappings.  The identifier can be an INTEGER
or OBJECT IDENTIFIER\@.  Handling ANY types properly will require
modifications to the generated code since there is no identifier
associated with the type.

The general approach used by {\em snacc} to handle ANY DEFINED BY
types is to lookup the identifier value in a hash table for the
identified type.  The hash table entry contains information about the
type such as the routines to use for encoding and decoding.

Two hash tables are used, one for INTEGER to type mappings and the
other for OBJECT IDENTIFIER to type mappings.  {\em Snacc} generates
an {\tt InitAny} routine for each module that uses the OBJECT-TYPE
macro.  This routine adds entries to the hash table(s).  The {\tt
InitAny} routine(s) is called once before any encoding or decoding is
done.


The hash tables are constructed such that an INTEGER or OBJECT
IDENTIFIER value will hash to an entry that contains:
\begin{itemize}
\item {the {\tt anyId}}
\item {the INTEGER or OBJECT IDENTIFIER that maps to it}
\item {the size in bytes of the identified data type}
\item {a pointer to the type's PDU encode routine}
\item {a pointer to the type's PDU decode routine}
\item {a pointer to the type's print routine}
\item {a pointer to the type's free routine}
\end{itemize}
The referenced encode and decode routines are PDU oriented in that
they encode the type's tag(s) and length(s) as well as the type's
content.

{\em Snacc} builds an {\tt enum} called {\tt AnyId} that enumerates
each mapping defined by the OBJECT-TYPE macros.  The name of the value
associated with each macro is used as part of the enumerated
identifier.  The {\tt anyId} in the hash table holds the identified
type's {\tt AnyId enum} value.  The {\tt anyId} is handy for making
decisions based on the received identifier, without comparing OBJECT
IDENTIFIERs.  If the identifiers are INTEGERs then the {\tt anyId} is
less useful.

With ANY DEFINED BY types, it is important to have the identifier
decoded before the ANY DEFINED BY type is decoded.  Hence, an ANY
DEFINED BY type should not be declared before its identifier in a SET
since SETs are un-ordered. An ANY DEFINED BY type should not be
declared after its identifier in a SEQUENCE\@. {\em Snacc} will print a
warning if either of these situations occur.

The hash tables may be useful to plain ANY types which do not have an
identifier field like the ANY DEFINED BY types; the OBJECT-TYPE macro
can be used to define the mappings and the {\tt SetAnyTypeByInt} or
{\tt SetAnyTypeByOid} routine can be called with the appropriate
identifier value before encoding or decoding an ANY value.  The
compiler will insert calls to these routines where necessary with some
of the arguments left as ``???''.  There will usually be a ``{\tt /*
ANY -- Fix me! */}'' comment before code that needs to be modified to
correctly handle the ANY type.  The code generated from an ASN.1
module that uses the ANY type will not compile without modifications.

OPTIONAL ANYs and ANY DEFINED BY types that have not been tagged are a
special problem for {\em snacc}.  Unless they are the last element of a SET
or SEQUENCE, the generated code will need to be modified.  {\em Snacc} will
print a warning message when it encounters one of these cases.

To illustrate how ANY DEFINED BY values are handled, we present
typical encoding and decoding scenarios. Each ANY or ANY DEFINED BY
type is represented in C by the {\tt AsnAny} type which contains only
a {\tt void *} named {\tt value} to hold a pointer to the value and a
{\tt AnyInfo *} named {\tt ai} which points to a hash table entry.

When encoding, before the ANY DEFINED BY value is encoded, {\tt
SetAnyTypeByOid} or {\tt SetAnyTypeByInt} (depending on the type of
the identifier) is called with the current identifier value to set the
{\tt AsnAny} value's {\tt ai} pointer to the proper hash table entry.
Then to encode the ANY DEFINED BY value, the encode routine pointed to
from the hash table entry is called with the {\tt value} {\tt void *}
from the {\tt AsnAny} value.  The {\tt value} {\tt void *} in the {\tt
AsnAny} should point to a value of the correct type for the given
identifier, if the user set it up correctly.  Note that setting the
{\tt void *} value is not type safe; one must make sure that the
value's type is the same as indicated by the identifier.

For decoding, the identifier must be decoded prior to the ANY DEFINED
BY value otherwise the identifier will contain an uninitialized value.
Before the ANY or ANY DEFINED BY value is decoded, {\tt
SetAnyTypeByOid} or {\tt SetAnyTypeByInt} (depending on the type of
the identifier) is called to set the {\tt AsnAny} value's {\tt ai}
pointer to the proper hash table entry.  Then a block of memory of the
size indicated in the hash table entry is allocated, and its pointer
stored in the {\tt AsnAny} value's {\tt void *} entry.  Then the decode
routine pointed to from the hash table entry is called with the newly
allocated block as its value pointer parameter.  The decode routine
fills in the value assuming it is of the correct type. Simple!

There is a problem with {\em snacc}'s method for handling ANY DEFINED
BY types for specifications that have two or more ANY DEFINED BY types
that share some identifier values.  Since only two hash tables are
used and they are referenced using the identifier value as a key,
duplicate identifiers will cause unresolvable hash collisions.

Here is some of the {\C AsnAny} related code from the header file.  It
should help you understand the way things are done a bit better.  Look
in the {\ufn hash.c} and {\ufn hash.h} files as well.
\begin{small}
\begin{verbatim}
/*
 * 1 hash table for integer keys
 * 1 hash table for oid keys
 */
extern Table *anyOidHashTblG;
extern Table *anyIntHashTblG;

typedef (*EncodeFcn) (BUF_TYPE b, void *value);
typedef void (*DecodeFcn) (BUF_TYPE b, void *value,
                           AsnLen *bytesDecoded, ENV_TYPE env);
typedef void (*FreeFcn) (void *v);
typedef void (*PrintFcn) (FILE *f, void *v);

/*
 * this is put into the hash table with the
 * int or oid as the key
 */
typedef struct AnyInfo
{
  int          anyId;  /* will be a value from the AnyId enum */
  AsnOid       oid;    /* will be zero len/null if intId is valid */
  AsnInt       intId;
  unsigned int size;  /* size of the C data type (ie as ret'd by sizeof) */
  EncodeFcn    Encode;
  DecodeFcn    Decode;
  FreeFcn      Free;
  PrintFcn     Print;
} AnyInfo;

typedef struct AsnAny
{
    AnyInfo   *ai; /* point to entry in hash tbl that has routine ptrs */
    void      *value; /* points to the value */
} AsnAny;

/*
 * Returns anyId value for the given ANY type.
 * Use this to determine to the type of an ANY after decoding
 * it. Returns -1 if the ANY info is not available
 */
#define GetAsnAnyId( a)  (((a)->ai)? (a)->ai->anyId: -1)

/*
 * used before encoding or decoding a type so the proper
 * encode or decode routine is used.
 */
void SetAnyTypeByInt (AsnAny *v, AsnInt id);
void SetAnyTypeByOid (AsnAny *v, AsnOid *id);


/*
 * used to initialize the hash table(s)
 */
void InstallAnyByInt (int anyId,  AsnInt intId,
                     unsigned int size, EncodeFcn encode,
                     DecodeFcn decode, FreeFcn free, PrintFcn print);

void InstallAnyByOid (int anyId, AsnOid *oid, unsigned int size,
                     EncodeFcn encode, DecodeFcn decode, FreeFcn free,
                     PrintFcn print);

/*
 * Standard enc, dec, free, & print routines.
 * for the AsnAny type.
 * These call the routines referenced from the
 * given value's hash table entry.
 */
void FreeAsnAny (AsnAny *v);
AsnLen BEncAsnAny (BUF_TYPE b, AsnAny *v);
void BerDecAsnAny (BUF_TYPE b, AsnAny  *result, AsnLen *bytesDecoded,
                  ENV_TYPE env);
void PrintAsnAny (FILE *f, AsnAny *v, unsigned short indent);


/* AnyDefinedBy is the same as AsnAny */
typedef AsnAny                 AsnAnyDefinedBy;
#define FreeAsnAnyDefinedBy    FreeAsnAny
#define BEncAsnAnyDefinedBy    BEncAsnAny
#define BDecAsnAnyDefinedBy    BDecAsnAny
#define PrintAsnAnyDefinedBy   PrintAsnAny
\end{verbatim}
\end{small}


\section{\label{lib-buf-section}Buffer Management}

Encoding and decoding performance is heavily affected by the cost of
writing to and reading from buffers, thus, efficient buffer management
is necessary.  Flexibility is also important to allow integration of
the generated encoders and decoders into existing environments.  To
provide both of these features, the calls to the buffer routines are
actually macros that can be configured as you want (see
{\ufn \dots/c-lib/inc/asn-config.h}). Virtually all buffer calls will
be made from the encode/decode library routines.  So macros used in
the generated code will make buffer calls.

If your environment uses a single, simple buffer type, the buffer
routine macros can be defined as the macros for your simple buffer type.
This results in the buffer type being bound at compile time, with no
function call overhead from the encode or decode routines.  This also
means that the runtime library only works for that buffer type.

If multiple buffer formats must be supported at runtime, the buffer
macros can be defined like the ISODE buffer calls, where a buffer type
contains pointers to the buffer routines and data of the current
buffer type.  This approach will hurt performance since each buffer
operation will be an indirect function call.  I have implemented
buffers like this for the table tools (performace is already hosed so
slower buffer routines are a drop in the bucket).  See the type tables
section for their description.

The backwards encoding technique requires special buffer primitives
that write from the end of the buffer towards the front.  This
requirement will make it impossible to define buffer primitives that
write directly to stream oriented objects such as TCP connections.  In
cases such as this, you must encode the entire PDU before sending it.
(Or else extend the back-end of the compiler to produce ``forwards''
encoders as well).

Nine buffer primitives are required by the runtime library's encode
and decode routines:
\begin{itemize}
\item {\C unsigned char BufGetByte (BUF\_TYPE b);}
\item {\C unsigned char BufPeekByte (BUF\_TYPE b);}
\item {\C char *BufGetSeg (BUF\_TYPE b, unsigned long int *lenPtr);}
\item {\C void BufCopy (char *dst, BUF\_TYPE b, unsigned long int *lenPtr);}
\item {\C void BufSkip (BUF\_TYPE b, unsigned long int len);}
\item {\C void BufPutByteRv (BUF\_TYPE b,  unsigned char byte);}
\item {\C void BufPutSegRv (BUF\_TYPE b, char *data, unsigned long int len);}
\item {\C int BufReadError (BUF\_TYPE b);}
\item {\C int BufWriteError (BUF\_TYPE b);}
\end{itemize}

These buffer operations are described in the next subsections.  The
{\C ExpBuf}, {\C SBuf} and {\C MinBuf} buffer formats that come
with the Snacc distribution and how to configure the buffer operations
are discussed following that.

\subsection{\label{buf-read-c-section}Buffer Reading Routine Semantics}

The buffer reading routines are called by the decoder routines.  The
following is the list of necessary buffer reading routines and their
semantics.   Be sure to setup the buffer in reading mode before
calling any of these routines.  The means of putting a buffer in
reading mode depends on the buffer type.

\begin{verbatim}
unsigned char BufGetByte (BUF_TYPE b);
\end{verbatim}
Returns the next byte from the buffer and advances the current pointer
such that a subsequent buffer read returns the following byte(s).
This will set the read error flag if an attempt to read past the end
of the data is made.

\begin{verbatim}
unsigned char BufPeekByte (BUF_TYPE b);
\end{verbatim}
Returns the next byte from the buffer without advancing the current
pointer.

\begin{verbatim}
char *BufGetSeg (BUF_TYPE b, unsigned long int *lenPtr);
\end{verbatim}
Returns a pointer to the next bytes from the buffer and advances the
current pointer. {\C *lenPtr} should contain the number of bytes to
read.  If the buffer has a least {\C *lenPtr} contiguous bytes
remaining to be read before calling {\C BufGetSeg}, a pointer to
them will be returned and {\C *lenPtr} will be unchanged.  If there
are less than {\C *lenPtr} contiguous bytes remaining in the buffer
before the call to {\C BufGetSeg}, a pointer to them is returned and
{\C *lenPtr} is set to the actual number of bytes that are
referenced by the returned pointer.  The current pointer will be
advanced by the value returned in {\C *lenPtr} (this may advance to the
next buffer segment if any).  Note that the read error flag is not set
if  {\C *lenPtr} is greater than the remaining number of unread
bytes.

\begin{verbatim}
unsigned long int BufCopy (char *dst, BUF_TYPE b, unsigned long int len)
\end{verbatim}
Copies the next {\C len} bytes from the buffer into the {\C dst char~*}
and advances the current pointer appropriately.  Returns the
number of bytes actually copied.  The number of bytes copied will be
less than requested only if the end of data is reached, in which case
the read error flag is set.


\begin{verbatim}
void BufSkip (BUF_TYPE b, unsigned long int len);
\end{verbatim}
Advances the buffer's current pointer by {\C len} bytes.  This will set the
read error flag if less than {\C len} unread bytes remain in the
buffer before the call to {\C BufSkip}.

\begin{verbatim}
int BufReadError (BUF_TYPE b);
\end{verbatim}
Returns non-zero if a read error occurred for the given buffer.
Read errors occur if one of the buffer reading routines attempted to
read past the end of the buffer's data.

\subsection{\label{buf-write-c-section}Buffer Writing Routine Semantics}

Encoding routines call the buffer writing routines.  Here is a list of
the buffer writing routine and their semantics.  Before calling the
writing routines, you should make sure the buffer is setup for
writing in reverse mode.  The means of doing this depends on the
buffer type.

\begin{verbatim}
void BufPutByteRvs (BUF_TYPE b, unsigned char byte);
\end{verbatim}
Writes the given byte to the beginning of the data in the given
buffer.  The newly written byte becomes part of the buffer's data such
that subsequent writes place bytes before the newly written byte.  If
a buffer write error occurs, subsequent writes do nothing.

\begin{verbatim}
void BufPutSegRvs (BUF_TYPE b, char *data, unsigned long int len);
\end{verbatim}
Prepends the given bytes, {\C data}, of length {\C len} to the
beginning of the data in the given buffer {\C b}.  The {\C data}
bytes are written such that the first byte in {\C data} becomes the
first byte of the buffer's data, followed by the rest. (This means the
bytes in {\C data} are not reversed, they are simply prepended as a
unit to the buffer's original data). If a buffer write error occurs,
subsequent writes do nothing.

\begin{verbatim}
int BufWriteError (BUF_TYPE b);
\end{verbatim}
Returns non-zero if a write error occurred for the given buffer.
Write errors occur if the buffer runs out of space for data or cannot
allocate another data block (depends on the buffer type).

\subsection{Buffer Configuration}

The runtime library's encode and decode routines as well as the
generated code access the buffers via the nine buffer macros
described in the last two sections.  These macros can be defined to
call simple macros for speed or to call functions.  Note that the
buffer configuration is bound at the time the library and generated
code are compiled.

The following is from {\ufn \dots/include/asn-config.h} and shows how to
configure the buffer routines.  This setup will make all calls to
{\C BufGetByte} in the library and  generated code call your
{\C ExpBufGetByte} routine; the other buffer routines are mapped to
their {\C ExpBuf} equivalents in a similar way.

\begin{verbatim}
#include "exp-buf.h"
#define BUF_TYPE			ExpBuf **
#define BufGetByte( b)			ExpBufGetByte (b)
#define BufGetSeg( b, lenPtr)		ExpBufGetSeg (b, lenPtr)
#define BufCopy( dst, b, lenPtr)	ExpBufCopy (dst, b, lenPtr)
#define BufSkip( b, len)		ExpBufSkip (b, len)
#define BufPeekByte( b)			ExpBufPeekByte (b)
#define BufPutByteRv( b,  byte)		ExpBufPutByteRv (b, byte)
#define BufPutSegRv( b, data, len)	ExpBufPutSegRv (b, data, len)
#define BufReadError( b)		ExpBufReadError (b)
#define BufWriteError( b)		ExpBufWriteError (b)
\end{verbatim}

If you want to use your own buffer type, simply edit the
{\ufn asn-config.h} file such that it includes your buffer's header
file, sets the {\C BUF\_TYPE} type, and defines the nine buffer
routines ({\C BufGetByte} etc.) to call your buffer routines.  Your
buffer routines should have the semantics and prototypes described in
the last two sections (Sections \ref{buf-read-c-section} and~\ref{buf-write-c-section}).

\subsection{ExpBuf Buffers}

The {\C ExpBuf} buffers are a doubly linked series of buffers that
can be expanded when encoding by adding new buffers as necessary.
Each {\C ExpBuf} consists of two blocks of memory, one for the
control and linking information and the other for the data; when
refering to an {\C ExpBuf} both parts are included. {\C ExpBuf} is
short for ``Expanding Buffer''.  Look in {\ufn \dots/c-lib/exp-buf.c}
for an ASCII drawing of the {\C ExpBuf} buffers. Take a look a the
{\ufn \dots/c-examples/simple/expbuf-ex.c} file for a quick
introduction on how to use {\C ExpBufs}.

{\C ExpBufs} are fairly general and useful when a reasonable upper
bound can not be put on the size of the encoded values that will be
encountered by the protocol.  The flexibility of these buffer routines
will hurt the performance as many of the {\C ExpBuf} calls are not
macros and new buffers may need to be allocated during encoding.

For encoding you need to write into the {\C ExpBufs}. Start with a
single ExpBuf (or the last one in a list of ExpBufs from a previous
encoding).  Make sure this ExpBuf has been reset is ``Write Reverse''
mode (use {\C ExpBufResetInWriteRvsMode}).  This clears the write
error flag (and sets the read error flag in case you try a read) and
resets the data start and data end pointers such that the buffer is
empty and ready for writing from the end towards the front.

During encoding, if an {\C ExpBuf}'s data part fills up, a new
{\C ExpBuf} before (since writing is reversed) the current buffer is
needed.  If the {\C prev} pointer in the current buffer is non-NULL,
the previous buffer is reset for writing and becomes the current
buffer.  If the {\C prev} pointer in the current buffer si NULL, a new
buffer is allocated, its pointer is placed in {\C prev} and it
becomes the current buffer.  The notion of current buffer is handled
by the parameter to the encoding and decoding routines.  The buffer
parameter is an {\C ExpBuf~**} and it always holds the current
{\C ExpBuf~*} (current buffer).

When encoding is finished and the encoded value has been transmitted,
you have two options.   You can free the entire buffer list or you can
keep them around and re-use them for the next encoding.  Freeing the
buffers after each encoding may be quite slow.  If you re-use the
buffers, the buffer list will grow to the size of the largest encoding
and stay there.  You can easily implement other management schemes.
By default the {\C ExpBuf}s (both parts) are allocated and freed with
{\C malloc} and {\C free}; you may want to change this to fit your
environment better. If buffer allocation fails during a write, the
writeError flag will be set and subsequent writes will do nothing.

For decoding you will want to put the encoded data into the
{\C ExpBuf} format.   For example, if your encoded value is
contiguous in a single block of memory, you could use
{\C ExpBufInstallDataInBuf} to attach your data to a single ExpBuf.
Once your data is in the ExpBuf format, you should call
{\C ExpBufResetInReadMode} on the first buffer in the list (if more
than one).  Then you can pass it to the desired decode routine.

If a decode routine attempts to read past the end of a buffer (usually
due to an erroneous encoding), the readError flag will be set for the
current {\C ExpBuf} in the list.  This error will typically cause
the decoding routine that called the buffer read routine to call
{\C longjmp}.

The {\C BUF\_TYPE} is defined as {\C ExpBuf~**} so that the buffer
parameter {\C b} can be set to the next active {\C ExpBuf} by the
buffer routines.  This saves having a head of the list type structure
that keeps track of the first, last and current buffers (the
indirectness of this approach would hurt performance).

There are many routines for administrating the {\C ExpBufs} if you
want to treat them like an abstract data type.  Sometimes it may be
easier to skip the utility routines and modify the fields directly.

The following routines are the required nine buffer routines.  Compile
the library and the generated code with the {\C USE\_EXP\_BUF} symbol
defined to map buffer routines that the generated and library code
calls to the {\C ExpBuf} routines (see
{\ufn \dots/c-lib/inc/asn-config.h}). These {\C ExpBuf} routines
adhere to the buffer routine prototypes and semantics defined in
Sections \ref{buf-read-c-section} and~\ref{buf-write-c-section}.

\begin{verbatim}
void          ExpBufSkip (ExpBuf **, unsigned long len);
int           ExpBufCopy (char *dst, ExpBuf **b, unsigned long len);
unsigned char ExpBufPeekByte (ExpBuf **b);
char         *ExpBufGetSeg (ExpBuf **b, unsigned long *len);
void          ExpBufPutSegRvs (ExpBuf **b, char *data, unsigned long len);
unsigned char ExpBufGetByte (ExpBuf **b);
void          ExpBufPutByteRvs (ExpBuf **b, unsigned char byte);

#define ExpBufReadError( b)   ((*b)->readError)
#define ExpBufWriteError( b)  ((*b)->writeError)
\end{verbatim}


The following {\C ExpBuf} routines are also provided.  Their
descriptions can be found in the code.
\begin{verbatim}
void ExpBufInit (unsigned long dataBlkSize);
void ExpBufInstallDataInBuf (ExpBuf *b, char *data, unsigned long int len);

void ExpBufResetInReadMode (ExpBuf *b);
void ExpBufResetInWriteRvsMode (ExpBuf *b);

ExpBuf *ExpBufAllocBufAndData();
void ExpBufFreeBufAndData (ExpBuf *b);
void ExpBufFreeBufAndDataList (ExpBuf *b);

ExpBuf *ExpBufNext (ExpBuf *b);
ExpBuf *ExpBufPrev (ExpBuf *b);
ExpBuf *ExpBufListLastBuf (ExpBuf *b);
ExpBuf *ExpBufListFirstBuf (ExpBuf *b);

int ExpBufAtEod (ExpBuf *b);
int ExpBufFull (ExpBuf *b);
int ExpBufHasNoData (ExpBuf *b);

char *ExpBufDataPtr (ExpBuf *b);
unsigned long ExpBufDataSize (ExpBuf *b);
unsigned long ExpBufDataBlkSize (ExpBuf *b);
\end{verbatim}

\subsection{SBuf Buffers}

The {\C SBuf}s are simple buffers of a fixed size, much like an
{\C ExpBuf} that cannot expand.  If you attempt to write
past the end of the buffer, the writeError flag will be set and the
encoding will fail.  If you attempt to read past the end of a buffer
the readError flag will be set and the decoding will fail.

The {\C SBuf}s are useful if you can put a reasonable upper bound on
the size of the encodings you will be dealing with.  The buffer
operations are much simpler because the data is contiguous.  In fact,
all of the {\C SBuf} buffer operations are implemented by macros.

Look in {\ufn \dots/c-examples/simple/sbuf-ex.c} for a quick
introduction to using {\C SBuf}s in your code. The following
operations are defined for the {\C SBuf} buffers.
\begin{verbatim}
/* The nine required buffer operations */
#define SBufSkip(b, skipLen) ...
#define SBufCopy(dst, b, copyLen) ...
#define SBufPeekByte(b) ...
#define SBufGetSeg( b, lenPtr) ...
#define SBufPutSegRvs(b, seg, segLen) ...
#define SBufGetByte(b) ...
#define SBufPutByteRvs(b, byte) ...
#define SBufReadError(b) ...
#define SBufWriteError(b) ...

/* other useful buffer operations */
#define SBufInit(b, data, dataLen) ...
#define SBufResetInReadMode(b) ...
#define SBufResetInWriteRvsMode(b) ...
#define SBufInstallData(b, data, dataLen) ...
#define SBufDataLen(b) ...
#define SBufDataPtr(b) ...
#define SBufBlkLen(b) ...
#define SBufBlkPtr(b) ...
#define SBufEod(b) ...
\end{verbatim}

Snacc is configured to use {\C SBuf}s by default.  The symbols that
will affect the buffer configuration during compilation of the
libraries and generated code are {\C USE\_EXP\_BUF} and
{\C USE\_MIN\_BUF}.

\subsection{MinBuf Buffers}

The {\C MinBuf}s provide maximum performance but should only be used under
restricted conditions (to avoid segmentation faults etc.).  No checks are
made to determine whether a decoder is reading past the end of the
buffer or if an encoder is writing ``past'' the beginning of the data
block (remember, snacc encoders write backwards).

A {\C MinBuf} is just a {\C char~**}; the referenced {\C char~*} points
to the next byte to be read or the last byte that was written. The
read routine advances the {\C char~*} and the write reverse routines
move the {\C char~*} backwards.

When you start encoding, the {\C MinBuf} {\C char~**} should be a
pointer to a pointer to the byte AFTER the last valid byte in your
buffer. For example the following C fragment would work:
\begin{verbatim}
PersonnelRecord pr;
char blk[128];
char *minBuf;

minBuf = blk + 128; /* start writing a end of block */
BEncPersonnelRecord (&minBuf, pr);
\end{verbatim}

The {\C MinBuf}s should only be used during encoding if the size of
the {\C MinBuf}'s buffer is guaranteed to be large enough to hold
the encoded value. Otherwise, the encoder will blindly continue
writing into whatever lies after the {\C MinBuf}'s buffer.

When you start decoding, the {\C MinBuf} value should be a pointer
to a pointer to the first byte of the BER value to be decoded.  Look
in {\ufn \dots/c-examples/simple/minbuf-ex.c} for a real example.

The {\C MinBuf}s should only be used for decoding when the value
being decoded is certain to contain no encoding errors. Otherwise, for
encodings that are incomplete or contain length errors, the decoder may
attempt to read the memory that follows the {\C MinBuf}s. If you are
lucky, the decoder will return an error with the {\C longjmp}
mechanism. If your system has memory protection and you are unlucky
this may abort your program.  If you are really unlucky, the data
following the {\C MinBuf} may fool the decoder into thinking that it
is valid and you receive a wrong PDU with no error indication.  This
risky technique has been used successfully in some systems where the
encodings are not guaranteed to be correct.

To configure the generated code to use the {\C MinBuf}s, compile it
with the {\C USE\_MIN\_BUF} symbol defined.

\subsection{Hybrid Buffer Solutions}

The decoding routines only call the buffer reading routines and the
encoding routines only call the buffer writing routines.  You may wish
to choose a different buffer format for the encoding and decoding to
gain performance.  For instance, if you can be sure that the size of
outgoing encodings is less than a certain upper bound, but don't want
to risk segmentation faults when decoding incoming values, you could
use {\C MinBuf}s for the the buffer writing (encoding) operations
and {\C SBuf}s or {\C ExpBuf}s for the buffer reading (decoding)
operations.

In this case you will need to massage the generated code to achieve
the desired results.

\section{\label{lib-mem-C-section}Dynamic Memory Management}

Like buffer management, efficient memory management is very important
for efficient decoders.  As a decoder decodes a value, it allocates
memory to hold the internal representation of the value.

The runtime librarys and the generated decode routines allocate memory
using the\linebreak {\C Asn1Alloc} routine. The runtime librarys
and the generated free routines free memory using the {\C Asn1Free}
routine.  The decoding routines also use {\C CheckAsn1Alloc} to make
sure that each allocation succeeded.  These memory routines are defined
in the
{\ufn asn-config.h} and have the prototypes:
\begin{verbatim}
void *Asn1Alloc (unsigned long int size);
void  Asn1Free (void *ptr);
int   CheckAsn1Alloc (void *ptr, ENV_TYPE env);
\end{verbatim}

The decoders assume that {\C Asn1Alloc} returns a \emph{zeroed} block
of memory.  This saves explicit initialization of OPTIONAL elements with
NULL in the generated decoders.  It wouldn't be too hard to modify the
compiler to produce decoders that initialized OPTIONAL elements
explicitly.

The generated free routines hierarchically free all a value's
memory using a depth first algorithm.  If you use the Nibble Memory
scheme, you will not need the generated free routines.

By default, snacc uses a ``Nibble Memory'' scheme to provide efficient
memory management.  Nibble Memory works by allocating a large block of
memory for allocating from.  When the decoded value has been
processed, you can free the entire value by calling a routine that
simply resets a few pointers.  There is no need to traverse the entire
value freeing a piece at a time.  The following is from
{\ufn nibble-alloc.h}.
\begin{verbatim}
void InitNibbleMem (unsigned long int initialSize,
                   unsigned long int incrementSize);
void *NibbleAlloc (unsigned long int size);
void ResetNibbleMem();
void ShutdownNibbleMem();
\end{verbatim}

You must explicitly initialize the Nibble Memory with the
{\C InitNibbleMem} routine before using a decoder.  You must specify
the initial size of the nibble block and the size that it should grow
by.  If you attempt to allocate a block that is larger that the
initial nibble block or its grow size, a new block of the correct size
will be allocated.  Note that the ``growth'' occurs by linking
separate blocks, not by the potentially slow alternative,
{\C realloc}.

When you have processed the decoded value you can free it by calling
{\C ResetNibbleMem}.  This resets a couple pointers and frees any
extra blocks that were allocated to handle values larger than the
initial block size.  The original memory block is zeroed
using {\C memset} so that all allocations will return zeroed values.
This is necessary to support the implicit initialization of OPTIONAL
elements to NULL\@.  The zeroing is done in this routine instead of
{\C NibbleAlloc} under the assumption that zeroing one large block
is more efficient than zeroing pieces of it as they are allocated.

When you no longer need the Nibble Memory, you can release it by
using\linebreak {\C ShutDownNibbleMem}.  This frees all of the
memory associated with Nibble Memory, both the control data and the
block(s) used for allocation.

There are some problems with this memory management scheme.  Currently
the Nibble Memory control information is kept track of via a global
variable that holds a pointer to the control information.  This can
present a problem if separate Nibble Memory contexts are needed, for
example, one to hold one value that will be kept after decoding and
another to hold a decoded value that will soon be discarded.

The problem of separate contexts could be solved by adding another
layer that would use identifiers for different memory contexts.  This
would require you to set the context using its identifier before
calling a decoding routine and to pass the context identifier to the
{\C ResetNibbleMem} routine.

Another problem has to do with building the values to be encoded.
There is no restriction on what allocator you use to build internal
values.  However, it is convenient to use the {\C AsnListNew}
routine to allocate and initialize a list type.  Unfortunately,
{\C AsnListNew} is used by the decoding routines so it uses the
{\C Asn1Alloc} routine to allocate the new list.  You should be
aware of this if {\C Asn1Alloc} is not what you are using to
allocate the rest of the value. This could be fixed with a different
interface to the {\C AsnListNew} routine.

It is possible to change the memory management system without too much
difficulty.  For example if you are not too worried about performance
and want to use {\C malloc} and {\C free}, you could change the
{\ufn asn-config.h} file as follows:
\begin{verbatim}
#include "malloc.h"
#define Asn1Alloc( size)  calloc (1, size)
#define Asn1Free( ptr)    free (ptr)
#define CheckAsn1Alloc( ptr, env)\
   if ((ptr) == NULL)\
      longjmp (env, -27);
\end{verbatim}
If you use {\C malloc} based allocators such as {\C calloc}, you
must use the generated free routines to free your values.  Note that
this example used {\C calloc} instead of {\C malloc} because
{\C calloc} {\em zeroes} each allocated block of memory, as required
by the decoders.


\section{\label{lib-err-C-section}Error Management}

The decoding routines use {\C longjmp} to handle any errors they
encounter in the value being decoded.  {\C longjmp} works by rolling
back the stack to where the {\C setjmp} call was made.  Every decode
routine takes a {\C jmp\_buf env} parameter (initialized by the
{\C setjmp} call) that tells the {\C longjmp} routine how to
restore the processor to the correct state.  {\C longjmp} makes the
error management much simpler since the decoding routines do not have
to pass back error codes or check ones from other decoding routines.

Before a PDU can be decoded, the {\C jmp\_buf env} parameter to the
decoding routine must be initialized using the {\C setjmp} routine.
This should be done immediately and only once before calling the
decoding routine.  This parameter will be passed down to any other
decoding routines called within a decoding routine. The following code
fragment from {\ufn \dots/c-examples/simple/exbuf-ex.c} shows how to
use {\C setjmp} before decoding.

\begin{small}
\begin{verbatim}
if ((val = setjmp (env)) == 0)
   BDecPersonnelRecord (&buf, &pr, &decodedLen, env);
else
{
    decodeErr = TRUE;
    fprintf (stderr, "ERROR - Decode routines returned %d\n", val);
}
\end{verbatim}
\end{small}

The code that will signal an error typically looks like:
\begin{small}
\begin{verbatim}
if (mandatoryElmtCount1 != 2)
{
    Asn1Error ("BDecChildInformationContent: ERROR - non-optional elmt missing from SET.\n");
    longjmp (env, -108);
}
\end{verbatim}
\end{small}


Most {\C longjmp} calls are preceded by a call to {\C Asn1Error}
which takes a single {\C char~*} string as a parameter.  The library
routines and the generated code try to use meaningful messages as the
parameter. {\C Asn1Error} is defined in {\ufn \dots/c-lib/inc/asn-config.h} and
currently just prints the given string to {\C stderr}.  You may wish
to make it do nothing, which may shrink the size of your binary
because all of the error strings will be gone.  {\C Asn1Warning} is
similar but is not used by the library or generated code anymore.

The encoding routines do no error checking except for buffer
overflows.  Hence, they do not use the {\C longjmp} mechanism and
instead require you to check the status of the buffer after encoding
(use {\C BufWriteError()}).  If you are not building your values
properly, for example having random pointers for uninitialized
OPTIONAL elements, the encode routines will fail, possibly
catastrophically.

% file: .../doc/c++-gen.tex

% $Header: /usr/app/odstb/CVS/snacc/doc/c++-gen.tex,v 1.1 1997/01/01 22:47:35 rj Exp $
% $Log: c++-gen.tex,v $
% Revision 1.1  1997/01/01 22:47:35  rj
% first check-in
%

\chapter{\label{c++-code-gen-chapter}C++ Code Generation}

\section{\label{intro-C++-section}Introduction}

The C++ backend of snacc was designed after the C backend had been
written.  The basic model that the generated C++ uses is similar to
that of the generated C, but benefits from the object oriented features of
C++.  This was my first real foray into C++ which may be evident from
some of the design.

As with C, two files are generated for each ASN.1 module, a {\ufn .C}
and a {\ufn .h} file.

Some cleaner designs were rejected either due to their poor
performance or the inability of the available C++ compiler to handle
those features.

Tags and lengths would fit nicely into their own classes but
performance was considerably worse than the technique used in the C
environment.  The C design was retained in the C++ model for its
superior performance.

For error management C++'s {\C try} and {\C throw} are obvious
replacements for the {\C setjmp} and {\C longjmp} used by the C
decoders. Unfortunately this is a newer C++ feature and is not yet
supported by gcc.

C++ templates are very attractive for type safe lists (for SET OF and
SEQUENCE OF) without duplicating code.  Template support was shaky in
gcc at the time the generated code was being tested so they were
rejected. Instead, each list generates its own new class with all of
the standard list routines.

As with the C code generation chapter, we will use the EX1 module to
help illustrate some of the code generation.  The following is the
same EX1 module used in the C section.

\begin{ASNcode}
EX1 DEFINITIONS ::=\\
BEGIN\\
\\
anOidVal      OBJECT IDENTIFIER ::= \{ joint-iso-ccitt 40 foobar(29) \}\\
theSameOidVal OBJECT IDENTIFIER ::= \{ 2 40 29 \}\\
anIntVal      INTEGER ::= 1\\
aBoolVal      BOOLEAN ::= TRUE\\
\\
T1 ::= SEQUENCE\\
\{\+\\
  INTEGER OPTIONAL,\\
  OCTET STRING OPTIONAL,\\
  ENUMERATED \{ a(0), b(1), c(2) \},\\
  SEQUENCE OF INTEGER,\\
  SEQUENCE \{ id OBJECT IDENTIFIER, value OCTET STRING \},\\
  CHOICE \{ INTEGER, OBJECT IDENTIFIER \}\-\\
\}\\
\\
END
\end{ASNcode}

The C++ backend to snacc is in the {\ufn \dots/compiler/back-ends/c++-gen/} directory if you want to alter it.

\section{\label{naming-C++-section}ASN.1 to C++ Naming Conventions}

The C++ name for a type or value is the same as its ASN.1 name with
any hyphens converted to underscores.

When an ASN.1 type or value name (after converting any hyphens to
underscores) conflicts with a C++ keyword or the name of a type in
another ASN.1 module (name clashes within the same ASN.1 scope are
considered errors and are detected earlier), the resulting C++ class
name will be the conflicting name with digits appended to it.

Empty field names in SETs, SEQUENCEs, and CHOICEs will be filled.  The
field name is derived from the type name for that field. The library
types such as INTEGER etc. have default field names defined by the
compiler (see {\ufn \dots/compiler/back-ends/c-gen/rules.c} and
{\ufn \dots/compiler/back-ends/c++-gen/rules.c}).
The first letter of the field name is in lower case.
Empty field names should be fixed properly by adding them to the ASN.1 source.

New type definitions will be generated for SETs, SEQUENCEs, CHOICEs,
ENUMERATED, INTEGERs with named numbers and BIT STRINGs with named bits
whose definitions are embedded in other SET, SEQUENCE, SET OF,
SEQUENCE OF, or CHOICE definitions.  The name of the new type
is derived from the name of the type in which it was embedded and will
be made unique by appending digits if necessary.

\section{\label{type-gen-C++-section}ASN.1 to C++ Class Translation}

This section describes how C++ classes are used to represent each
ASN.1 type.  First, the general characteristics of each ASN.1 type's
C++ class will be discussed followed by how the aggregate types (SETs,
SEQUENCEs, CHOICEs, SET OFs, and SEQUENCE OFs) are represented.  The
representations of non-aggregate types (INTEGER, BOOLEAN, OCTET
STRING, BIT STRING, OBJECT IDENTIFIER) and ANY and ANY DEFINED BY
types are presented in the next chapter since they form part of the
C++ ASN.1 runtime library.

Every ASN.1 type is represented by a C++ class with the following
characteristics:
\newcounter{saveenumi}
\begin{enumerate}
  \item it inherits from the {\C AsnType} base class
  \item it has a parameterless constructor
  \item it has a copy constructor
  \item it has a destructor
  \item it has a clone method, {\C Clone}
  \item it has an assignment operator
  \item it has a content encode and decode method, {\C BEncContent} and {\C BDecContent}
  \item it has a PDU encode and decode method, {\C BEnc} and {\C BDec}
  \item it has a top level interfaces to the PDU encode and decode methods (handles the {\C setjmp} etc.) for the user, {\C BEncPdu} and {\C BDecPdu}
  \item it has a print method, {\C Print}, a virtual function that gets called from a global <\/<-operator
  \setcounter{saveenumi}{\value{enumi}}
\end{enumerate}
If the metacode has been enabled:
\begin{enumerate}
  \setcounter{enumi}{\value{saveenumi}}
  \item it has a virtual function {\C \_getdesc} that returns the classes meta description% (only if metacode is enabled)
  \item if it is a structured type, it has a virtual function {\C \_getref} that returns a pointer to one of its components/members, specified through its name% (only if metacode is enabled)
  \setcounter{saveenumi}{\value{enumi}}
\end{enumerate}
If the Tcl code has been enabled:
\begin{enumerate}
  \setcounter{enumi}{\value{saveenumi}}
  \item it has a virtual function {\C TclGetDesc} to access the metacode's {\C \_getdesc} routine from Tcl% (only if Tcl code is enabled)
  \item it has a virtual function {\C TclGetVal} to retrieve an instance's value% (only if Tcl code is enabled)
  \item it has a virtual function {\C TclSetVal} to change an instance's value% (only if Tcl code is enabled)
  \item for SET, SEQUENCE, SET OF and SEQUENCE of: it has a virtual function {\C TclUnsetVal} to clear OPTIONAL members or to delete list elements, respectively% (only if Tcl code is enabled)
\end{enumerate}

The following C++ fragment shows the class features listed above in greater
detail.
\begin{Ccode}
class Foo: public AsnType\\
\{\+\\
  \dots // data members\\
\\
\<public:\\
          \>\>Foo();\\
          \>\>Foo (const Foo \&);\\
          \>\>\~{}Foo();\\
  AsnType \>\>*Clone() const;\\
  Foo     \>\>\&operator = (const Foo \&);\\
\\
  // content encode and decode routines\\
  AsnLen  \>\>BEncContent (BUF\_TYPE b);\\
  void    \>\>BDecContent (BUF\_TYPE b, AsnTag tag, AsnLen elmtLen,\\
                     \`AsnLen \&bytesDecoded, ENV\_TYPE env);\\
\\
  // PDU (tags/lengths/content) encode and decode routines\\
  AsnLen  \>\>BEnc (BUF\_TYPE b);\\
  void    \>\>BDec (BUF\_TYPE b, AsnLen \&bytesDecoded, ENV\_TYPE env);\\
\\
  // methods most likely to be used by your code.\\
  // Returns non-zero for success\\
  int     \>\>BEncPdu (BUF\_TYPE b, AsnLen \&bytesEncoded);\\
  int     \>\>BDecPdu (BUF\_TYPE b, AsnLen \&bytesDecoded);\\
\\
  void    \>\>Print (ostream \&os) const;\\
\\
\<\#if META\\
  const AsnTypeDesc  \>\>*\_getdesc() const;\\
  AsnType            \>\>*\_getref (const char *membername, bool create = false);\\
\<\#if TCL\\
  int                \>\>TclGetDesc (Tcl\_DString *) const;\\
  int                \>\>TclGetVal (Tcl\_Interp *) const;\\
  int                \>\>TclSetVal (Tcl\_Interp *, const char *valstr);\\
  int                \>\>TclUnsetVal (Tcl\_Interp *, const char *membername);\\
\<\#endif\\
\<\#endif\\
\<\};
\end{Ccode}

{\C BEnc} and {\C BDec} are PDU encode and decode methods.
{\C BEnc} encodes the tag and length pairs for the object's type as
well as the content (the object's value) to the given buffer,
{\C b}, and returns the number of bytes written to the buffer for
the encoding.

{\C BDec} decodes the expected tag and length pairs as well as the
content of the object it is invoked upon from the given buffer,
{\C b}, and increments {\C bytesDecoded}  by the byte length of
the tag(s), length(s) and value decoded.  The {\C env} parameter
will be used with {\C longjmp} if any decoding error occurs.
Decoding errors can be reported via {\C longjmp} from any of the
routines that {\C BDec} calls, such as {\C BDecContent};
{\C BDec} will call {\C longjmp} directly if the value does not
have the correct tag(s).

{\C BEncContent} and {\C BDecContent} only deal with the content
of the type their object represents.  {\C BEncContent} encodes the
object's value to the given buffer, {\C b}.

{\C BDecContent} decodes the object's value from the given buffer,
{\C b}.  The last tag and length pair on the content must be passed in
via the {\C tag} and {\C elmtLen} parameters.  The {\C tag},
although always present, will only be used when decoding OCTET STRING and
BIT STRING related types, to determine whether the encoding is
constructed.  The {\C elmtLen} is the length of the content and may
be the indefinite length form.  {\C bytesDecoded} is incremented by
the actual number of bytes in the content; this is normally the same
as {\C elmtLen} unless the indefinite length form was decoded.  The
{\C env} parameter will be used with {\C longjmp} if any decoding
error occurs.  The possible decoding errors depend on the type that is
being decoded.

{\C BEncPdu} and {\C BDecPdu} are top-level interfaces to the PDU
encode and decode routines.  They present the simplest interface;
they return TRUE if the operation succeeded and FALSE if an error
occurred.   Note that the {\C BDecPdu} routine sets up the
{\C env} parameter using {\C setjmp} for any {\C longjmp} calls
that may occur.  If you call {\C BDec} or {\C BDecContent}
directly from your code, you must use {\C setjmp} to setup the
{\C env} parameter.  {\C BEncPdu} checks for any buffer writing
errors and {\C BDecPdu} checks for any buffer reading errors.

The {\C Print} method prints the object's value in ASN.1 value
notation.  When printing SETs and SEQUENCEs, a global variable
is used for the current indent.

The {\C AsnType} base class, parameterless constructor and
{\C Clone} method are required by the ANY and ANY DEFINED BY type
handling mechanism explained in Sections \ref{asntype-C++-section} and
\ref{any-C++-section}.  In brief, the {\C AsnType} provides a base type
that has virtual {\C BEnc}, {\C BDec} and {\C Clone} routines.
The {\C Clone} routine is used to generate a new instance (not a
copy) of the object that it is invoked on.  This allows the ANY
DEFINED BY type decoder to create a new object of the correct type
from one stored in a hash table, when decoding (the {\C Clone}
routine calls the parameterless constructor).  The virtual {\C BEnc}
and {\C BDec} are called from {\C AsnAny} {\C BEnc} and
{\C BDec} methods.

The meta routines and the Tcl interface will be described in chapters \ref{meta-chapter} and \ref{tcl-if-chapter}, respectively.

\subsection{\label{C++-set-seq-section}SET and SEQUENCE}

SET and SEQUENCE types generate classes that have their components as
public data members.  This makes accessing the components similar to
referencing the fields of a C struct.  For example the {\C T1} type in
module EX1 will produce the following C++ class:

\begin{Ccode}
class T1: public AsnType\\
\{\\
public:\+\\
  AsnInt    \>\>*integer;\\
  AsnOcts   \>\>*octs;\\
  T1Enum    \>\>t1Enum;\\
  T1SeqOf   \>\>t1SeqOf;\\
  T1Seq     \>\>*t1Seq;\\
  T1Choice  \>\>*t1Choice;\\
\\
            \>\>T1();\\
%  {\\
%    /* init optional/default elements to NULL */\\
%    integer = NULL;\\
%    octs = NULL;\\
%  }\\
	    \>\>T1 (const T1 \&);\\
	    \>\>\~{}T1();\\
%  AsnType *Clone() { return new T1; }\\
  AsnType   \>\>*Clone() const;\\
\\
  T1        \>\>\&operator = (const T1 \&);\\
\\
  AsnLen    \>\>BEnc (BUF\_TYPE b);\\
  void      \>\>BDec (BUF\_TYPE b, AsnLen \&bytesDecoded, ENV\_TYPE env);\\
\\
  AsnLen    \>\>BEncContent (BUF\_TYPE b);\\
  void      \>\>BDecContent (BUF\_TYPE b, AsnTag tag, AsnLen elmtLen,\\
                         \`AsnLen \&bytesDecoded, ENV\_TYPE env);\\
\\
  int       \>\>BEncPdu (BUF\_TYPE b, AsnLen \&bytesEncoded);\\
  int       \>\>BDecPdu (BUF\_TYPE b, AsnLen \&bytesDecoded);\\
\\
  void      \>\>Print (ostream \&os) const;\\
\\
\<\#if META\\
  static const AsnSequenceTypeDesc     \\
				\>\>\_desc;\\
  static const AsnSequenceMemberDesc   \\
				\>\>mdescs[];\\
  const AsnTypeDesc             \>\>*\_getdesc() const;\\
  AsnType                       \>\>*\_getref (const char *membername, bool create = false);\\
\\
\<\#if TCL\\
  int                   \>\>TclGetDesc (Tcl\_DString *) const;\\
  int                   \>\>TclGetVal (Tcl\_Interp *) const;\\
  int                   \>\>TclSetVal (Tcl\_Interp *, const char *valstr);\\
  int                   \>\>TclUnsetVal (Tcl\_Interp *, const char *membname);\\
\<\#endif // TCL\\
\<\#endif // META\\
\<\};
\end{Ccode}

All OPTIONAL components in a SET or SEQUENCE are referenced by pointer.
The constructor will automatically set OPTIONAL fields to {\C NULL}\@.  The
other methods are as described at the beginning of this section.

SETs and SEQUENCEs must contain all non-OPTIONAL components and
SEQUENCEs must be ordered, otherwise an error is reported.  Tagging
errors are also reported.  All detected errors abort the decoding
process via {\C longjmp}.

\subsection{\label{C++-choice-section}CHOICE}

Each CHOICE type generates a class that has an anonymous union to hold the
components of the CHOICE and a {\C choiceId} field to indicate which
component is present.

Anonymous (un-named) unions allow you to reference the choice components
with just the field name of the component; this makes referencing the
contents of a CHOICE the same a referencing the contents of a SET or
SEQUENCE\@.

The {\C choiceId} field contains a value in the {\C ChoiceIdEnum}
that indicates the CHOICE field that is present.  The names in the
enumeration are derived from the field names of the CHOICE components.

When building a local value to be encoded, you must be sure to set the
{\C choiceId} such that it corresponds to the value in the union.  The
decoder will set the {\C choiceId} when decoding incoming values.

Tagging errors are reported and abort the decoding process via
{\C longjmp}.

The following C++ class is produced for the CHOICE in the EX1 module.

\begin{Ccode}
class T1Choice: public AsnType\\
\{\\
public:\+\\
  enum ChoiceIdEnum\\
  \{\+\\
    integerCid = 0,\\
    oidCid = 1\-\\
  \};\\
\\
  enum ChoiceIdEnum \>\>choiceId;\\
  union\\
  \{\+\\
    AsnInt   \>*integer;\\
    AsnOid   \>*oid;\-\\
  \};\\
\\
             \>\>T1Choice();\\
	     \>\>T1Choice (const T1Choice \&);\\
	     \>\>\~{}T1Choice();\\
  AsnType    \>\>*Clone() const;\\
\\
  T1Choice   \>\>\&operator = (const T1Choice \&);\\
\\
  AsnLen     \>\>BEncContent (BUF\_TYPE b);\\
  void       \>\>BDecContent (BUF\_TYPE b, AsnTag tag, AsnLen elmtLen,\\
                          \`AsnLen \&bytesDecoded, ENV\_TYPE env);\\
\\
  AsnLen     \>\>BEnc (BUF\_TYPE b);\\
  void       \>\>BDec (BUF\_TYPE b, AsnLen \&bytesDecoded, ENV\_TYPE env);\\
\\
  int        \>\>BEncPdu (BUF\_TYPE b, AsnLen \&bytesEncoded);\\
  int        \>\>BDecPdu (BUF\_TYPE b, AsnLen \&bytesDecoded);\\
\\
  void       \>\>Print (ostream \&os) const;\\
\\
\<\#if META\\
  static const AsnChoiceTypeDesc        \>\>\_desc;\\
  static const AsnChoiceMemberDesc      \\
					\>\>\_mdescs[];\\
\\
  const AsnTypeDesc             \>\>*\_getdesc() const;\\
  AsnType                       \>\>*\_getref (const char *membername, bool create = false);\\
\\
\<\#if TCL\\
  int                   \>\>TclGetDesc (Tcl\_DString *) const;\\
  int                   \>\>TclGetVal (Tcl\_Interp *) const;\\
  int                   \>\>TclSetVal (Tcl\_Interp *, const char *valstr);\\
\<\#endif // TCL\\
\<\#endif // META\\
\<\};
\end{Ccode}



\subsection{\label{C++-set-of-section}SET OF and SEQUENCE OF}

Each SET OF and SEQUENCE OF type produces its own list class, unlike
the C backend which uses a single generic list type for all lists.
This makes the C++ list routines type safe which allows the C++
compiler to detect more programmer errors.

C++ templates should be used to reduce the code duplication when they
become widespread and reliably implemented.  The duplicated list
handling methods may bloat the size of the generated code.

Any tagging errors are reported and abort the decoding process via
{\C longjmp}.

From the EX1 ASN.1 module the following list is produced:
\begin{Ccode}
class T1SeqOf: public AsnType\\
\{\\
protected:\+\\
    unsigned long int\>\>count;\\
    struct AsnListElmt\\
    \{\+\\
        struct AsnListElmt \>*next;\\
        struct AsnListElmt \>*prev;\\
        AsnInt             \>*elmt;\-\\
    \}                   \>\>*first, *curr, *last;\\
\\
\<public:\\
                         \>\>T1SeqOf() \{ count = 0; first = curr = last = NULL; \}\\
                         \>\>\~{}T1SeqOf();\\
    AsnType              \>\>*Clone() const;\\
\\
    void                 \>\>SetCurrElmt (unsigned long int index);\\
    unsigned long int    \>\>GetCurrElmtIndex();\\
    void                 \>\>SetCurrToFirst();\\
    void                 \>\>SetCurrToLast();\\
\\
    // reading member fcns\\
    int                  \>\>Count() const;\\
    AsnInt               \>\>*First() const;\\
    AsnInt               \>\>*Last() const;\\
    AsnInt               \>\>*Curr() const;\\
    AsnInt               \>\>*Next() const;\\
    AsnInt               \>\>*Prev() const;\\
\\
    // routines that move the curr elmt\\
    AsnInt               \>\>*GoNext();\\
    AsnInt               \>\>*GoPrev();\\
\\
    // write \& alloc fcns--returns new elmt\\
    AsnInt               \>\>*Append();  // add elmt to end of list\\
    AsnInt               \>\>*Prepend(); // add elmt to beginning of list\\
    AsnInt               \>\>*InsertBefore(); // insert elmt before current elmt\\
    AsnInt               \>\>*InsertAfter();  // insert elmt after current elmt\\
\\
    // write \& alloc \& copy--returns list after copying elmt\\
    T1SeqOf              \>\>\&AppendCopy (AsnInt \&elmt);  // add elmt to end of list\\
    T1SeqOf              \>\>\&PrependCopy (AsnInt \&elmt); // add elmt to beginning of list\\
    T1SeqOf              \>\>\&InsertBeforeAndCopy (AsnInt \&elmt); // insert elmt before current elmt\\
    T1SeqOf              \>\>\&InsertAfterAndCopy (AsnInt \&elmt); // insert elmt after current elmt\\
\\
    // removing the current elmt from the list\\
    void                 \>\>RemoveCurrFromList();\\
\\
    // encode and decode routines\\
    AsnLen               \>\>BEncContent (BUF\_TYPE b);\\
    void                 \>\>BDecContent (BUF\_TYPE b, AsnTag tag, AsnLen elmtLen,\\
                         \` AsnLen \&bytesDecoded, ENV\_TYPE env);\\
\\
    AsnLen               \>\>BEnc (BUF\_TYPE b);\\
    void                 \>\>BDec (BUF\_TYPE b, AsnLen \&bytesDecoded, ENV\_TYPE env);\\
\\
    int                  \>\>BEncPdu (BUF\_TYPE b, AsnLen \&bytesEncoded);\\
    int                  \>\>BDecPdu (BUF\_TYPE b, AsnLen \&bytesDecoded);\\
\\
    void                 \>\>Print (ostream \&os);\\
\\
\<\#if META\\
  static const AsnListTypeDesc  \>\>\_desc;\\
  const AsnTypeDesc             \>\>*\_getdesc() const;\\
  AsnType                       \>\>*\_getref (const char *index, bool create = false);\\
\\
\<\#if TCL\\
  int                   \>\>TclGetDesc (Tcl\_DString *) const;\\
  int                   \>\>TclGetVal (Tcl\_Interp *) const;\\
  int                   \>\>TclSetVal (Tcl\_Interp *, const char *valstr);\\
  int                   \>\>TclUnsetVal (Tcl\_Interp *, const char *valstr);\\
\<\#endif // TCL\\
\<\#endif // META\\
\<\};
\end{Ccode}

Each list is doubly linked to allow simple reverse traversal for
backwards encoding.  The {\C prev} pointer will be {\C NULL} for the
first element of the list and the {\C next} pointer will be {\C NULL} for
the last element of the list.

Each list maintains a pointer to the current element of the list.
Several routines are provided to manipulate the current item.  Since
there is only one current pointer, you may have to save and restore
the current pointer with the {\C GetCurrElmtIndex} and
{\C SetCurrElmt} methods if you call routines that deal with the
list while iterating through it.

In addition to the standard encode, decode and print methods, some
list utility routines are included in each list class.  They are
fairly simple and are described briefly here.

\begin{Ccode}
void \>\>\>SetCurrElmt (unsigned long int index);
\end{Ccode}
This sets the current pointer to the element with the given index.
Indexes start at zero, that is, the first element in the list has an
index of zero.  If the given index is greater than or equal to the
number of elements in the list, the current pointer is set to the last
element of the list.

\begin{Ccode}
unsigned long int \>\>\>GetCurrElmtIndex();
\end{Ccode}
This returns the index of the current element.  If the current pointer
is {\C NULL} (or does not reference an element of the list, which is an
error condition), the index returned will be greater than or equal to
the number of elements in the list (indexes start at zero so this is
an invalid index).


\begin{Ccode}
void \>\>\>SetCurrToFirst();
\end{Ccode}
This sets the current pointer to the first element of the list. If the
list is empty, it is set to {\C NULL}\@.
\begin{Ccode}
void \>\>\>SetCurrToLast();
\end{Ccode}
This sets the current pointer to the last element of the list. If the
list is empty, it is set to {\C NULL}\@.


\begin{Ccode}
int \>\>\>Count() const;
\end{Ccode}
This returns the number of elements in the list.

\begin{Ccode}
AsnInt \>\>\>*First() const;\\
AsnInt \>\>\>*Last() const;\\
AsnInt \>\>\>*Curr() const;\\
AsnInt \>\>\>*Next() const;\\
AsnInt \>\>\>*Prev() const;
\end{Ccode}
The above routines return a pointer to the list element that the
routine name indicates.  They return {\C NULL} if the requested element is
not present.  For example {\C First} will return a pointer to the
first element in the list or {\C NULL} if the list is empty.  These
routines do not affect the state of the list; the current pointer and
the count remain the same.

\begin{Ccode}
AsnInt \>\>\>*GoNext();\\
AsnInt \>\>\>*GoPrev();
\end{Ccode}
These routines change the current pointer to the next/previous element
and return a pointer to that element.  If the current element is {\C NULL} or
points to the last element, {\C GetNext} returns {\C NULL}\@. Similarly, if
the current element is {\C NULL} or points to the first element, {\C GetPrev}
returns {\C NULL}\@.


\begin{Ccode}
AsnInt \>\>\>*Append();
\end{Ccode}
This allocates a new list element, appends it to the end of the list
and returns a pointer to the new list element.  Notice that you must
set the value of the returned list element.

\begin{Ccode}
AsnInt \>\>\>*Prepend();
\end{Ccode}
This allocates a new list element, prepends it to the beginning of the
list and returns a pointer to the new list element. You must set the
value of the returned list element.

\begin{Ccode}
AsnInt \>\>\>*InsertBefore();
\end{Ccode}
This allocates a new list element, inserts it before the current list
element and returns a pointer to the new list element.  You must set
the value of the returned list element.  If the current pointer is
{\C NULL}, the new element is placed at the beginning of the list.

\begin{Ccode}
AsnInt \>\>\>*InsertAfter();
\end{Ccode}
This allocates a new list element, inserts it after the current list
element and returns a pointer to the new list element.  You must set
the value of the returned list element.  If the current pointer is
{\C NULL}, the new element is placed at the end of the list.

\begin{Ccode}
T1SeqOf \>\>\>\&AppendCopy (AsnInt \&elmt);\\
T1SeqOf \>\>\>\&PrependCopy (AsnInt \&elmt);\\
T1SeqOf \>\>\>\&InsertBeforeAndCopy (AsnInt \&elmt);\\
T1SeqOf \>\>\>\&InsertAfterAndCopy (AsnInt \&elmt);
\end{Ccode}
These are similar to the {\C Append}, {\C Prepend},
{\C InsertBefore} and {\C InsertAfter} routines except that a
copy of the given element's value is placed in the list and the list
itself is returned.

\subsection{\label{C++-enumerated-section}ENUMERATED, Named Numbers and Named Bits}

The C++ type generator encapsulates each ENUMERATED type, INTEGER
with named numbers and BIT STRING with named bits in a new class that
inherits from the proper base class and defines the named elements.
This provides a separate scope for these identifiers so their symbol
will be exactly the same as their ASN.1 counterpart.  Currently these
identifiers are not checked for conflicts with C++ keywords, so you
may have to modify some of them in the ASN.1 modules.

Inheritance is used for attaching ENUMERATED, named number and named
bit information.  ENUMERATED types inherit from the {\C AsnEnum} class,
INTEGERs with named number types inherit from the {\C AsnInt} class and BIT
STRINGs with named bits inherit from the {\C AsnBits} class.

If the tagging on the type is different from the type it inherits
from, the PDU encode and decode methods are re-defined with the
correct tags to override the PDU encode and decode methods of the base
class.

As with the other types, any tagging errors are reported and abort the
decoding process via {\C longjmp}.  No range checking is done on the
decoded values although it would be easy to provide a new
{\C BDecContent} method in the new class that calls the base class's
and then checks the range of the result.

\begin{Ccode}
/* ENUMERATED { a(0), b(1), c(2) } */\\
class T1Enum: public AsnEnum\\
\{\\
public:\\
\#if TCL\\
         \>               \>\>T1Enum(): AsnEnum (\_nmdescs[0].value) \{\}\\
\#else\\
         \>               \>\>T1Enum(): AsnEnum () \{\}\\
\#endif\+\\
                          \>\>T1Enum (int i): AsnEnum (i) \{\}\\
  enum\\
  \{\+\\
    a = 0,\\
    b = 1,\\
    c = 2\-\\
  \};\\
\\
\<\#if META\\
  static const AsnNameDesc      \>\>\_nmdescs[];\\
  static const AsnEnumTypeDesc  \>\>\_desc;\\
  const AsnTypeDesc             \>\>*\_getdesc() const;\\
\<\#endif // META\\
\<\};
\end{Ccode}

\section{\label{val-gen-C++-section}ASN.1 to C++ Value Translation}

C++ {\C const} values are used to hold ASN.1 defined values.  C++
values will be produced for INTEGER, BOOLEAN and OBJECT IDENTIFIER
ASN.1 values.  An {\C extern} declaration for each {\C const} value is
written at the end of the header file of the value's module.  The
{\C const} values are defined at the beginning of the {\ufn .C} file
of the value's module.  The {\C extern} declarations are at the end
of the header file so that any required class definitions are
available.

The following is from the end of the header file generated for the EX1
module:
\begin{Ccode}
extern const AsnOid  \>\>\>anOidVal;\\
extern const AsnOid  \>\>\>theSameOidVal;\\
extern const AsnInt  \>\>\>anIntVal;\\
extern const AsnBool \>\>\>aBoolVal;
% \\ extern const AsnInt  \>\>\>foobar;
\end{Ccode}

The following is from the beginning of the {\ufn .C} file generated
for the EX1 module:
\begin{Ccode}
const AsnOid  \>\>\>anOidVal (2, 40, 29);\\
const AsnOid  \>\>\>theSameOidVal (2, 40, 29);\\
const AsnInt  \>\>\>anIntVal (1);\\
const AsnBool \>\>\>aBoolVal (true);
% \\ const AsnInt  \>\>\>foobar (29);
\end{Ccode}

The C++ constructor mechanism is used to generate these values.  This
mechanism is superior to C static initialization because it allows C++
code to be run to initialize the values.

\section{\label{compiler-dir-C++-section}Compiler Directives}
Compiler directives are ignored by the C++ backend of snacc.  If you want
to implement them, look at the {\C FillCxxTypeDefInfo} routine in
file {\ufn \dots/compiler/back-ends/c++-gen/types.c}.  Then look at the
way it is done for the C backend (file
{\ufn \dots/compiler/back-ends/c-gen/type-info.c})

\section{\label{compiling-gen-C++-section}Compiling the Generated C++ Code}

When compiling the generated C++ code you will need:
\begin{enumerate}
\item
  The include directory where the files from {\ufn \dots/c++-lib/inc/} have been installed in your include path so that the C++ sources can include these library header files.
  The header files should be included with statements like {\C \#include <snacc/c++/asn-incl.h>} and your C++ compiler should be supplied with {\ufn -I/usr/local/include} in case snacc got installed under {\ufn /usr/local/}.
\item
  to link with the C++ ASN.1 runtime library, {\ufn \dots/c++-lib/libasn1c++.a}.
  In case snacc got installed under {\ufn /usr/local/}, your linker may need to be supplied with {\ufn -L/usr/local/lib} and {\ufn -lasn1c++} as arguments.
\item
  to link with the math library ({\ufn -lm}), since the ASN.1 REAL type's encode and decode routine use some math routines.
\end{enumerate}

See the example in {\ufn \dots/c++-examples/simple/} for a complete
example. The makefile and main routines are probably the most
important.  There are several other examples in the
{\ufn \dots/c++-examples/} directory.

% file: .../doc/c++-lib.tex

% $Header: /usr/app/odstb/CVS/snacc/doc/c++-lib.tex,v 1.1 1997/01/01 22:47:37 rj Exp $
% $Log: c++-lib.tex,v $
% Revision 1.1  1997/01/01 22:47:37  rj
% first check-in
%

\chapter{\label{lib-C++-chapter}C++ ASN.1 Library}
\section{\label{libover-C++-section}Overview}

The following sections describe the C++ representation of the
non-aggregate ASN.1 types, ANY and ANY DEFINED BY types and
the buffer and memory management.  These classes and routines make
up the C++ ASN.1 runtime library. Every aggregate ASN.1 type will be
composed of these library types. The source files for this library
are in {\ufn \dots/c++-lib/inc/} and {\ufn \dots/c++-lib/src/}.

As mentioned in the last chapter, each ASN.1 type is represented by a
C++ class which inherits from the {\C AsnType} base class.  In
addition to the standard encode, decode, print and clone methods
described in the last chapter, each ASN.1 type class in the library
may also have special constructors and other routines that simplify
their use.

Unlike the classes generated for some of the aggregate types such as
SETs and SEQUENCEs, the library types' data members are typically
protected and accessed via methods.

All of the library classes' {\C BDec} routines will report tagging errors
via {\C longjmp()} as described in section~\ref{type-gen-C++-section}.

The top level PDU encode and decode methods are the same for all
library types so they are defined as macros in
{\ufn \dots/c++-lib/inc/asn-config.h}. For clarity's sake, the macro
that is used to define these methods in the library type class
definitions will be replaced with the actual prototypes.

Run the {\ufn test-lib} program in {\ufn \dots/c++-examples/test-lib/}
to make sure the library routines are working properly for your
architecture.  The testing is not exhaustive but should point out
obvious problems.


\section{\label{tag-C++-section}Tags}

The C++ tags are identical to those used in snacc's C ASN.1
environment.  While it would have been nice to define a tag class, the
performance cost would likely have been noticeable.  Also, the snacc
users usually do not have to muck around with tags directly, so the
lack of a class interface will probably not be missed.  The C ASN.1
tags are described in Section~\ref{tag-C-section}.

Initially I defined a C++ class for tags, but close examination of the
produced assembly code led me to reject it.  The C++ class for tags
used the C tag representation internally and had constructor, encode
and decode methods.  The constructor could not be used as
{\C switch} statement case labels like {\C MAKE\_TAG\_ID} because it
did not reduce to an integer constant; this caused problems in the
generated decoders.

As with the C representation of tags, 4 byte long integers limit the
maximum representable tag code to $2^{21}$.  Again, this should not be
a problem.


\section{\label{len-C++-section}Lengths}

The C++ representation of lengths is the same at the C representation
described in Section~\ref{len-C-section}.  The length type was not given
its own C++ class for reasons similar to those of tags.


\section{\label{asntype-C++-section}The AsnType Base Class}

Every ASN.1 type's C++ class uses the {\C AsnType} as its base class.
The {\C AsnType} base class provides the following virtual functions:
\begin{itemize}
  \item the destructor
  \item {\C Clone()}
  \item {\C BDec()}
  \item {\C BEnc()}
  \item {\C Print()}
  \item {\C \_getdesc()} (metacode)
  \item {\C \_getref()} (metacode)
  \item {\C TclGetDesc()} (Tcl interface)
  \item {\C TclGetVal()} (Tcl interface)
  \item {\C TclSetVal()} (Tcl interface)
  \item {\C TclUnsetVal()} (Tcl interface)
\end{itemize}

The {\C AsnType} class is defined as follows:
\begin{Ccode}
class AsnType\\
\{\\
public:\+\\
  virtual			\>\>\~{}AsnType();\\
\\
\<\#ifdef SUPPORT\_ANY\_TYPE\\
  virtual AsnType		\>\>*Clone() const;\\
  virtual void			\>\>BDec (BUF\_TYPE b, AsnLen \&bytesDecoded, ENV\_TYPE env);\\
  virtual AsnLen		\>\>BEnc (BUF\_TYPE b);\\
\<\#else\\
  void				\>\>BDec (BUF\_TYPE b, AsnLen \&bytesDecoded, ENV\_TYPE env) \{\}\\
  AsnLen			\>\>BEnc (BUF\_TYPE b) \{ return 0; \}\\
\<\#endif\\
  virtual void			\>\>Print (ostream \&os) const;\\
\\
\<\#if META\\
  static const AsnTypeDesc	\>\>\_desc;\\
\\
  virtual const AsnTypeDesc	\>\>*\_getdesc() const;\\
  virtual AsnType		\>\>*\_getref (const char *membername, bool create=false);\\
\\
\<private:\\
  const char			\>\>*\_typename() const;\\
\-\\
\#if TCL\\
public:\+\\
  virtual int			\>\>TclGetDesc (Tcl\_DString *) const;\\
  virtual int			\>\>TclGetVal (Tcl\_Interp *) const;\\
  virtual int			\>\>TclSetVal (Tcl\_Interp *, const char *val);\\
  virtual int			\>\>TclUnsetVal (Tcl\_Interp *, const char *membernames);\-\\
\#endif // TCL\\
\#endif // META\\
\};
\end{Ccode}

The {\C AsnType} class and its virtual functions were added to
support the ANY DEFINED BY type handling mechanism.  This mechanism is
described in Section~\ref{any-C++-section}.

Even if you do not use the ANY or ANY DEFINED BY types, the
{\C AsnType} base class may be useful for adding features that are
common to all of the types, such as changing the {\C new} and
{\C delete} functions to improve performance.

Virtual functions provide the simplest method of handling ANY DEFINED
BY and ANY types. Unfortunately, calls to virtual functions are slower
than calls to normal functions due to their indirect nature.  If you
do not need support for the ANY DEFINED BY or ANY types you can remove
most of the virtual functions to improve performance by undefining the
{\C SUPPORT\_ANY\_TYPE} symbol (see the
{\ufn asn-type.h} file).

Note that a virtual destructor is included in the {\C AsnType} base
class as well.  This is done to make sure the {\C delete} routine
always gets the correct size.  See pages 215--217 of Stroustrup
\cite{stroustrup} for a discussion of this.

\section{\label{bool-C++-section}BOOLEAN}

The BOOLEAN type is represented by the {\C AsnBool} class. The
following is the class definition of {\C AsnBool} from the
{\ufn \dots/c++-lib/inc/asn-bool.h} file.

\begin{Ccode}
class AsnBool: public AsnType\\
\{\\
protected:\+\\
  bool				\>\>value;\\
\\
\<public:\\
				\>\>AsnBool (const bool val): value (val) \{\};\\
				\>\>AsnBool() \{\};\\
  AsnType			\>\>*Clone() const;\\
				\>\>operator bool() const \{ return value; \}\\
  AsnBool			\>\>\&operator = (bool newvalue) \{ value = newvalue; return *this; \}\\
\\
  AsnLen			\>\>BEnc (BUF\_TYPE b);\\
  void				\>\>BDec (BUF\_TYPE b, AsnLen \&bytesDecoded, ENV\_TYPE env);\\
\\
  AsnLen			\>\>BEncContent (BUF\_TYPE b);\\
  void				\>\>BDecContent (BUF\_TYPE b, AsnTag tagId, AsnLen elmtLen,\\
					\`AsnLen \&bytesDecoded, ENV\_TYPE env);\\
\\
  int				\>\>BEncPdu (BUF\_TYPE b, AsnLen \&bytesEncoded);\\
  int				\>\>BDecPdu (BUF\_TYPE b, AsnLen \&bytesDecoded);\\
\\
  void				\>\>Print (ostream \&os) const;\\
\\
\<\#if META\\
  static const AsnBoolTypeDesc	\>\>\_desc;\\
\\
  const AsnTypeDesc		\>\>*\_getdesc() const;\\
\\
\<\#if TCL\\
  int				\>\>TclGetVal (Tcl\_Interp *) const;\\
  int				\>\>TclSetVal (Tcl\_Interp *, const char *val);\-\\
\#endif // TCL\\
\#endif // META\\
\};
\end{Ccode}

The upcoming C++ standard \cite{c++-draft} defines a type {\C bool} accompanied by {\C false} and {\C true} to denote the boolean values.
The Snacc configuration script checks whether the C++ compiler already supplies this new type and defines a look-alike if it does not.

For backwards compatibility, {\C AsnBool::true} and {\C AsnBool::false} are still valid.

The {\C operator bool()} is defined such that when an {\C AsnBool}
value is cast to a boolean, it returns the C++ style boolean
value of the {\C AsnBool}s value.  There is also a constructor for {\C AsnBool}
that builds an {\C AsnBool} value from the given C++ style boolean
value. These two methods allow you to manipulate and access
{\C AsnBool} values in a straight forward way as the following code
illustrates.
\begin{Ccode}
Message::Send()\\
\{\+\\
  AsnBool	\>\>okToSend;\\
  bool		\>\>connectionOpen;\\
  bool		\>\>pduOk;\\
  \dots\\
  okToSend = connectionOpen \&\& pduOk; // assign AsnBool from bool\\
  if (okToSend) // cast AsnBool to bool\\
    \>\dots\\
\<\}
\end{Ccode}

The {\C AsnBool} class contains the standard encode and decode
methods that were described in Chapter~\ref{c++-code-gen-chapter}.

{\C BDecContent} will report an error via {\C longjmp} if the
length of an encoded BOOLEAN value's content is not exactly 1 octet.

Note that the {\C Clone} method returns an {\C AsnType~*} value
instead of an {\C AsnBool~*}.  It might be more obvious to return an
{\C AsnBool~*} since due to single inheritance an {\C AsnBool} is also
an {\C AsnType}.  However, it must return an {\C AsnType~*} for it
to override the virtual function {\C Clone} defined in the
{\C AsnType}.

The {\C Print} method will print either ``TRUE'' or ``FALSE''
depending on the {\C AsnBool} value.  No newline or other formatting
characters are printed.  The global indent information does not affect
the output from this method.


\section{\label{int-C++-section}INTEGER}

The INTEGER type is represented by the {\C AsnInt} class. The
following is the class definition of {\C AsnInt} from the
{\ufn \dots/c++-lib/inc/asn-int.h} file.
\begin{Ccode}
class AsnInt: public AsnType\\
\{\\
protected:\+\\
  AsnIntType			\>\>value;\\
\\
\<public:\\
				\>\>AsnInt() \{\}\\
				\>\>AsnInt (AsnIntType val): value (val) \{\}\\
\\
  AsnType			\>\>*Clone() const;\\
\\
				\>\>operator AsnIntType() \{ return value; \}\\
  AsnInt			\>\>\&operator = (AsnIntType newvalue) \{ value = newvalue; return *this; \}
\\
  void				\>\>Set (AsnIntType i) \{ value = i; \}\\
  void				\>\>ReSet (AsnIntType i) \{ value = i; \}\\
\\
  AsnLen			\>\>BEnc (BUF\_TYPE b);\\
  void				\>\>BDec (BUF\_TYPE b, AsnLen \&bytesDecoded, ENV\_TYPE env);\\
\\
  AsnLen			\>\>BEncContent (BUF\_TYPE b);\\
  void				\>\>BDecContent (BUF\_TYPE b, AsnTag tagId, AsnLen elmtLen,\\
					\`AsnLen \&bytesDecoded, ENV\_TYPE env);\\
\\
  int				\>\>BEncPdu (BUF\_TYPE b, AsnLen \&bytesEncoded);\\
  int				\>\>BDecPdu (BUF\_TYPE b, AsnLen \&bytesDecoded);\\
\\
  void				\>\>Print (ostream \&os) const \{ os <\/< value; \}\\
\\
\<\#if META\\
  static const AsnIntTypeDesc	\>\>\_desc;\\
\\
  const AsnTypeDesc		\>\>*\_getdesc() const;\\
\\
\<\#if TCL\\
  int				\>\>TclGetVal (Tcl\_Interp *) const;\\
  int				\>\>TclSetVal (Tcl\_Interp *, const char *val);\-\\
\#endif /* TCL */\\
\#endif /* META */\\
\};
\end{Ccode}

The internal representation of an ASN.1 INTEGER value is a {\C AsnIntType}.
This is a {\C typedef}, the real type may be {\C int}, {\C long} or {\C short}, whatever is found to be 32 bits in size.
The types' sizes depend on the machine and compiler and are determined at configuration time.
The {\C BDecContent} routine will signal an error if the integer
value being decoded will not fit into the {\C AsnIntType}
representation.

Unlike the C ASN.1 library, the non-negative
%{\C unsigned long int}
version of
INTEGER is not provided.  If you need it, it should be relatively
trivial to combine the C unsigned version with the existing C++ class.
The unsigned version of an integer is useful if your ASN.1 source uses
subtyping similar to:
\begin{ASNcode}
Counter ::= [APPLICATION 1] IMPLICIT INTEGER (0..4294967295)
\end{ASNcode}

\section{\label{enum-C++-section}ENUMERATED}

The ENUMERATED type is represented by the {\C AsnEnum} class. The
following is the class definition of {\C AsnEnum} from the
{\ufn \dots/c++-lib/inc/asn-enum.h} file.
\begin{Ccode}
class AsnEnum: public AsnInt\\
\{\\
public:\+\\
\<\#if !TCL\\
				\>\>AsnEnum(): AsnInt() \{\}\\
\<\#endif\\
				\>\>AsnEnum (int i): AsnInt (i) \{\}\\
\\
  AsnType			\>\>*Clone() const;\\
\\
  AsnLen			\>\>BEnc (BUF\_TYPE b);\\
  void				\>\>BDec (BUF\_TYPE b, AsnLen \&bytesDecoded, ENV\_TYPE env);\\
\\
  AsnLen			\>\>BEncContent (BUF\_TYPE b);\\
  void				\>\>BDecContent (BUF\_TYPE b, AsnTag tagId, AsnLen elmtLen,\\
           		                \`AsnLen \&bytesDecoded, ENV\_TYPE env);\\
\\
  int				\>\>BEncPdu (BUF\_TYPE b, AsnLen \&bytesEncoded);\\
  int				\>\>BDecPdu (BUF\_TYPE b, AsnLen \&bytesDecoded);\\
\\
\<\#if META\\
  static const AsnEnumTypeDesc	\>\>\_desc;\\
\\
  const AsnTypeDesc		\>\>*\_getdesc() const;\-\\
\#endif /* META */\\
\};
\end{Ccode}

Note that it is not derived from {\C class AsnType} directly but from {\C class AsnInt} instead.

\section{\label{null-C++-section}NULL}

The NULL type is provided by the {\C AsnNull} class.  This class has
no data members and includes only the standard methods.
\begin{Ccode}
class AsnNull: public AsnType\\
\{\\
public:\+\\
				\>\>AsnNull() \{\}\\
  AsnType			\>\>*Clone() const;\\
\\
  AsnLen			\>\>BEnc (BUF\_TYPE b);\\
  void				\>\>BDec (BUF\_TYPE b, AsnLen \&bytesDecoded, ENV\_TYPE env);\\
\\
  AsnLen			\>\>BEncContent (BUF\_TYPE b);\\
  void				\>\>BDecContent (BUF\_TYPE b, AsnTag tagId, AsnLen elmtLen,\\
					\`AsnLen \&bytesDecoded, ENV\_TYPE env);\\
\\
  int				\>\>BEncPdu (BUF\_TYPE b, AsnLen \&bytesEncoded);\\
  int				\>\>BDecPdu (BUF\_TYPE b, AsnLen \&bytesDecoded);\\
\\
  void				\>\>Print (ostream \&os) const \{ os <\/< "NULL"; \}\\
\\
\<\#if META\\
  static const AsnNullTypeDesc	\>\>\_desc;\\
\\
  const AsnTypeDesc		\>\>*\_getdesc() const;\\
\\
\<\#if TCL\\
  int				\>\>TclGetVal (Tcl\_Interp *) const;\\
  int				\>\>TclSetVal (Tcl\_Interp *, const char *val);\-\\
\#endif /* TCL */\\
\#endif /* META */\\
\};
\end{Ccode}

\section{\label{real-C++-section}REAL}

REAL types are represented by the {\C AsnReal} class.  Internally, a
{\C double} is used to hold the real value.  The following is from
{\ufn \dots/c++-lib/inc/asn-real.h}:
\begin{Ccode}
class AsnReal: public AsnType\\
\{\\
protected:\\
  \>double			\>\>value;\\
\\
public:\+\\
				\>\>AsnReal(): value (0.0) \{\}\\
				\>\>AsnReal (double val): value (val) \{\}\\
  AsnType			\>\>*Clone() const;\\
				\>\>operator double() const  \{ return value; \}\\
  AsnReal			\>\>\&operator = (double newvalue) \{ value = newvalue; return *this; \}\\
\\
  AsnLen			\>\>BEnc (BUF\_TYPE b);\\
  void				\>\>BDec (BUF\_TYPE b, AsnLen \&bytesDecoded, ENV\_TYPE env);\\
\\
  AsnLen			\>\>BEncContent (BUF\_TYPE b);\\
  void				\>\>BDecContent (BUF\_TYPE b, AsnTag tagId, AsnLen elmtLen,\\
					\`AsnLen \&bytesDecoded, ENV\_TYPE env);\\
\\
  int				\>\>BEncPdu (BUF\_TYPE b, AsnLen \&bytesEncoded);\\
  int				\>\>BDecPdu (BUF\_TYPE b, AsnLen \&bytesDecoded);\\
\\
  void				\>\>Print (ostream \&os) const \{ os <\/< value; \}\\
\\
\<\#if META\\
  static const AsnRealTypeDesc	\>\>\_desc;\\
\\
  const AsnTypeDesc		\>\>*\_getdesc() const;\\
\\
\<\#if TCL\\
  int				\>\>TclGetVal (Tcl\_Interp *) const;\\
  int				\>\>TclSetVal (Tcl\_Interp *, const char *val);\-\\
\#endif /* TCL */\\
\#endif /* META */\\
\};\\
\\
extern const AsnReal		\>\>\>PLUS\_INFINITY;\\
extern const AsnReal		\>\>\>MINUS\_INFINITY;
\end{Ccode}

The {\C double} representation and support routines can depend on
the compiler or system you are using so several different encoding
routines are provided.  Even so, you may need to modify the code.

There are three content encoding routines included and they can be selected by defining one of {\C IEEE\_REAL\_FMT}, {\C IEEE\_REAL\_LIB} or nothing.
Defining {\C IEEE\_REAL\_FMT} uses the encode routine that assumes the double representation is the standard IEEE double \cite{68881}.
Defining {\C IEEE\_REAL\_LIB} uses the encode routine that assumes the IEEE functions library (isinf, scalbn, signbit etc.) is available.
If neither are defined, the default encode routine uses {\C frexp}.
%The {\ufn \dots/configure} script tries to find the {\C isinf} and {\C finite} functions 
%Look at the related comments in {\ufn \dots/c++-lib/inc/asn-config.h}.
Currently, the {\ufn \dots/configure} script has not got any checks for the IEEE format or library and therefore does not define any of the symbols. (This should be fixed.)

{\C AsnReal} constants are used to hold {\C PLUS\_INFINITY} and
{\C MINUS\_INFINITY} values.  These values are initialized using the
{\C AsnReal} constructor mechanism with the {\C AsnPlusInfinity}
and {\C AsnMinusInfinity} routines.  If you do not define
{\C IEEE\_REAL\_FMT} or {\C IEEE\_REAL\_LIB}, you should rewrite the
{\C AsnPlusInfinity} routine such that it is correct for your system.

There is only one content decoding routine and it builds the value
through multiplication and the {\C pow} routine (requires the math
library). The content decoding routine only supports the binary
encoding of a REAL, not the decimal encoding.

\section{\label{bits-C++-section}BIT STRING}

The BIT STRING type is represented by the {\C AsnBits} class.  From\linebreak
{\ufn \dots/c++-lib/inc/asn-bits.h}:
\begin{Ccode}
class AsnBits: public AsnType\\
\{\\
private:\+\\
  int				\>\>BitsEquiv (AsnBits \&ab);\\
  void				\>\>BDecConsBits (BUF\_TYPE b, AsnLen elmtLen,\\
					\`AsnLen \&bytesDecoded, ENV\_TYPE env);\\
  void				\>\>FillBitStringStk (BUF\_TYPE b, AsnLen elmtLen0,\\
					\`AsnLen \&bytesDecoded, ENV\_TYPE env);\\
\<protected:\\
  size\_t			\>\>bitLen;\\
  char				\>\>*bits;\\
\\
\<public:\\
				\>\>AsnBits() \{ bits = NULL; bitLen = 0; \}\\
				\>\>AsnBits (const size\_t numBits) \{ Set (numBits); \}\\
				\>\>AsnBits (const char *bitOcts, const size\_t numBits) \\
					\`\{ Set (bitOcts, numBits); \}\\
				\>\>AsnBits (const AsnBits \&b) \{ Set (b); \}\\
				\>\>\~{}AsnBits() \{ delete bits; \}\\
\\
  AsnType			\>\>*Clone() const;\\
\\
  AsnBits			\>\>\&operator = (const AsnBits \&b) \{ ReSet (b); return *this; \}\\
\\
  size\_t			\>\>BitLen() \{ return bitLen; \}\\
\\
  bool				\>\>operator == (AsnBits \&ab) const \{ return BitsEquiv (ab); \}\\
  bool				\>\>operator != (AsnBits \&ab) const \{ return !BitsEquiv (ab); \}\\
\\
  // overwrite existing bits and bitLen values\\
  void				\>\>Set (size\_t numBits);\\
  void				\>\>Set (const char *bitOcts, size\_t numBits);\\
  void				\>\>Set (const AsnBits \&b);\\
\\
  // free old bits value, the reset bits and bitLen values\\
  void				\>\>ReSet (const size\_t numBits);\\
  void				\>\>ReSet (const char *bitOcts, size\_t numBits);\\
  void				\>\>ReSet (const AsnBits \&b);\\
\\
  void				\>\>SetBit (size\_t);\\
  void				\>\>ClrBit (size\_t);\\
  int				\>\>GetBit (size\_t) const;\\
\\
  AsnLen			\>\>BEnc (BUF\_TYPE b);\\
  void				\>\>BDec (BUF\_TYPE b, AsnLen \&bytesDecoded, ENV\_TYPE env);\\
\\
  AsnLen			\>\>BEncContent (BUF\_TYPE b);\\
  void				\>\>BDecContent (BUF\_TYPE b, AsnTag tagId, AsnLen elmtLen,\\
					\`AsnLen \&bytesDecoded, ENV\_TYPE env);\\
\\
  int				\>\>BEncPdu (BUF\_TYPE b, AsnLen \&bytesEncoded);\\
  int				\>\>BDecPdu (BUF\_TYPE b, AsnLen \&bytesDecoded);\\
\\
  void				\>\>Print (ostream \&os) const;\\
\\
\<\#if META\\
  static const AsnBitsTypeDesc	\>\>\_desc;\\
\\
  const AsnTypeDesc		\>\>*\_getdesc() const;\\
\\
\<\#if TCL\\
  int				\>\>TclGetVal (Tcl\_Interp *) const;\\
  int				\>\>TclSetVal (Tcl\_Interp *, const char *val);\-\\
\#endif /* TCL */\\
\#endif /* META */\\
\};
\end{Ccode}

The {\C AsnBits} class contains a pointer to the bits and an
integer that holds the length in bits of the BIT STRING\@.

In addition to the standard methods, the {\C AsnBits} class has
methods for initializing and comparing bit string values and methods
for setting and getting individual bits in a value.

An {\C AsnBits} value can be created three ways: from the number of
bits, from a {\C char~*} and its bit length or from another
{\C AsnBits} value.  Look at the constructors and the {\C Set} and
{\C ReSet} methods.

{\C SetBit} and {\C ClrBit} can be used for setting the values of
individual bits in the BIT STRING value.  Given the bit's index,
{\C SetBits} sets that bit to one.  {\C ClrBit} sets the bit
of the given index to zero.  The bit indexes start at zero, with zero
being the first (most signficant) bit in the BIT STRING\@. {\C GetBit}
will return {\C true} if the specified bit is one and {\C false} if the bit is
zero.  If the given bit index is too large, {\C SetBit} and
{\C ClrBit} do nothing and {\C GetBit} returns {\C false}\@.

The {\C ==} and {\C !=} operators have been overloaded such that
given two {\C AsnBits} values, they will behave as expected.

Each {\C AsnBits} value stores its bit string in a single contiguous
block of memory.  Received BIT STRING values that were encoded in the
constructed form are converted to the simple, flat form (see Section
\ref{bits-C-section}).  Snacc provides no facility for encoding or
internally representing constructed BIT STRING values.

\section{\label{octets-C++-section}OCTET STRING}
OCTET STRING values are represented with the {\C AsnOcts} class.
From\linebreak {\ufn \dots/c++-lib/inc/asn-octs.h}:
\begin{Ccode}
class AsnOcts: public AsnType\\
\{\\
private:\+\\
  int				\>\>OctsEquiv (const AsnOcts \&o);\\
  void				\>\>FillBitStringStk (BUF\_TYPE b, AsnLen elmtLen0,\\
					\`AsnLen \&bytesDecoded, ENV\_TYPE env);\\
  void				\>\>BDecConsOcts (BUF\_TYPE b, AsnLen elmtLen,\\
					\`AsnLen \&bytesDecoded, ENV\_TYPE env);\\
\\
\<protected:\\
  size\_t			\>\>octetLen;\\
  char				\>\>*octs;\\
\\
\<public:\\
  // constructor and Set alway copy strings so destructor can always delete\\
				\>\>AsnOcts(): octs (NULL), octetLen (0) \{\}\\
				\>\>AsnOcts (const char *str)  \{ Set (str); \}\\
				\>\>AsnOcts (const char *str, const size\_t len) \{ Set (str, len); \}\\
				\>\>AsnOcts (const AsnOcts \&o) \{ Set (o); \}\\
				\>\>\~{}AsnOcts() \{ delete octs; \}\\
  AsnType			\>\>*Clone() const;\\
\\
  AsnOcts			\>\>\&operator = (const AsnOcts \&o) \{ ReSet (o); return *this; \}\\
  AsnOcts			\>\>\&operator = (const char *str) \{ ReSet (str); return *this; \}\\
\\
  size\_t			\>\>Len() const \{ return octetLen; \}\\
				\>\>operator const char *() const \{ return octs; \}\\
				\>\>operator char *() \{ return octs; \}\\
\\
  bool				\>\>operator == (const AsnOcts \&o) const \{ return OctsEquiv (o); \}\\
  bool				\>\>operator != (const AsnOcts \&o) const \{ return !OctsEquiv (o); \}\\
\\
  // these set the octs and octetLen values\\
  void				\>\>Set (const char *str, size\_t len);\\
  void				\>\>Set (const AsnOcts \&o);\\
  void				\>\>Set (const char *str);\\
\\
  // these free the old octs value and then reset the octs and octetLen values\\
  void				\>\>ReSet (const char *str, size\_t len);\\
  void				\>\>ReSet (const AsnOcts \&o);\\
  void				\>\>ReSet (const char *str);\\
\\
  AsnLen			\>\>BEnc (BUF\_TYPE b);\\
  void				\>\>BDec (BUF\_TYPE b, AsnLen \&bytesDecoded, ENV\_TYPE env);\\
\\
  AsnLen			\>\>BEncContent (BUF\_TYPE b);\\
  void				\>\>BDecContent (BUF\_TYPE b, AsnTag tagId, AsnLen elmtLen,\\
					\`AsnLen \&bytesDecoded, ENV\_TYPE env);\\
\\
  int				\>\>BEncPdu (BUF\_TYPE b, AsnLen \&bytesEncoded);\\
  int				\>\>BDecPdu (BUF\_TYPE b, AsnLen \&bytesDecoded);\\
\\
  void				\>\>Print (ostream \&os) const;\\
\\
\<\#if META\\
  static const AsnOctsTypeDesc	\>\>\_desc;\\
\\
  const AsnTypeDesc		\>\>*\_getdesc() const;\\
\\
\<\#if TCL\\
  int				\>\>TclGetVal (Tcl\_Interp *) const;\\
  int				\>\>TclSetVal (Tcl\_Interp *, const char *val);\-\\
\#endif /* TCL */\\
\#endif /* META */\\
\};
\end{Ccode}

The {\C AsnOcts} class contains a pointer to the octets and an
integer that holds the length in octets of the OCTET STRING\@.

There are four constructors for {\C AsnOcts}.  The parameterless
constructor will initialize the octet string to zero length with a
{\C NULL} octets pointer.  The constructor that takes a single {\C char~*}
assumes that the given string is NUL terminated and initializes the octet
pointer with a pointer to a copy of the given string and sets the
{\C octetLen} to the {\C strlen} of the string (this does not usually
include the NUL terminator).  The constructor that takes
{\C char~*} and a length, {\C len}, initializes the octets pointer
to point to a copy of {\C len} characters from the given string and
sets the {\C octetLen} to {\C len}.  The last constructor will initialize
an {\C AsnOcts} value by copying the given {\C AsnOcts} value.

As with the BIT STRING content decoder, OCTET STRING content decoder
can handle constructed values.  These are handled in the same way as
the constructed BIT STRING values; they are converted to the simple
contiguous representation.  Every OCTET STRING value will
automatically have a NUL terminator appended to it; this extra
character will not be included in the string's length and will make
some strings easier to deal with for printing etc.

The {\C operator char~*()} is defined for the {\C AsnOcts} class to
return a pointer to the octets.  The {\C Len}
method returns the length in bytes of the string value.
These may be useful for passing the
octets to other functions such as {\C memcpy} etc.  

The {\C ==} and {\C !=} operators have been overloaded such that
given two {\C AsnOcts} values, they will behave as expected.



\section{\label{oid-C++-section}OBJECT IDENTIFIER}
OBJECT IDENTIFIER values are represented with the {\C AsnOid} class.
From\linebreak {\ufn \dots/c++-lib/inc/asn-oid.h}:
\begin{Ccode}
class AsnOid: public AsnType\\
\{\\
private:\\
  \>int				\>\>OidEquiv (AsnOid o);\\
\\
protected:\+\\
  size\_t			\>\>octetLen;\\
  char				\>\>*oid;\\
\\
\<public:\\
				\>\>AsnOid(): oid (NULL), octetLen (0) \{\}\\
				\>\>AsnOid (const char *encOid, size\_t len) \{ Set (encOid, len); \}\\
				\>\>AsnOid (const AsnOid \&o) \{ Set (o); \}\\
				\>\>AsnOid (unsigned long int a1, unsigned long int a2, long int a3 = -1,\\
					\`long int a4 = -1, long int a5 = -1, long int a6 = -1, long int a7 = -1,\\
					\`long int a8 = -1, long int a9 = -1, long int a10 = -1, long int a11 = -1);\\
				\>\>\~{}AsnOid() \{ delete oid; \}\\
  AsnType			\>\>*Clone() const;\\
\\
  AsnOid			\>\>\&operator = (const AsnOid \&o) \{ ReSet (o); return *this; \}\\
\\
  size\_t			\>\>Len() \{ return octetLen; \}\\
  const char			\>\>*Str() const \{ return oid; \}\\
				\>\>operator const char * () const \{ return oid; \}\\
				\>\>operator char * () \{ return oid; \}\\
  unsigned long int		\>\>NumArcs() const;\\
\\
  bool				\>\>operator == (AsnOid \&o) const \{ return OidEquiv (o); \}\\
  bool				\>\>operator != (AsnOid \&o) const \{ return !OidEquiv (o); \}\\
\\
  // Set methods overwrite oid and octetLen values\\
  void				\>\>Set (const char *encOid, const size\_t len);\\
  void				\>\>Set (const AsnOid \&o);\\
\\
  // first two arc numbers are mandatory.  rest are optional since negative arc nums are not allowed in the\\
  // encodings, use them to indicate the 'end of arc numbers' in the optional parameters\\
  void				\>\>Set (unsigned long int a1, unsigned long int a2, long int a3 = -1,\\
					\`long int a4 = -1, long int a5 = -1, long int a6 = -1, long int a7 = -1,\\
					\`long int a8 = -1, long int a9 = -1, long int a10 = -1, long int a11 = -1);\\
\\
  // ReSet routines are like Set except the old oid value is freed\\
  void				\>\>ReSet (const char *encOid, const size\_t len);\\
  void				\>\>ReSet (const AsnOid \&o);\\
  void				\>\>ReSet (unsigned long int a1, unsigned long int a2, long int a3 = -1,\\
					\`long int a4 = -1, long int a5 = -1, long int a6 = -1, long int a7 = -1,\\
					\`long int a8 = -1, long int a9 = -1, long int a10 = -1, long int a11 = -1);\\
\\
  AsnLen			\>\>BEnc (BUF\_TYPE b);\\
  void				\>\>BDec (BUF\_TYPE b, AsnLen \&bytesDecoded, ENV\_TYPE env);\\
\\
  AsnLen			\>\>BEncContent (BUF\_TYPE b);\\
  void				\>\>BDecContent (BUF\_TYPE b, AsnTag tagId, AsnLen elmtLen,\\
					\`AsnLen \&bytesDecoded, ENV\_TYPE env);\\
\\
  int				\>\>BEncPdu (BUF\_TYPE b, AsnLen \&bytesEncoded);\\
  int				\>\>BDecPdu (BUF\_TYPE b, AsnLen \&bytesDecoded);\\
\\
  void				\>\>Print (ostream \&os) const;\\
\\
\<\#if META\\
  static const AsnOidTypeDesc	\>\>\_desc;\\
\\
  const AsnTypeDesc		\>\>*\_getdesc() const;\\
\\
\<\#if TCL\\
  int				\>\>TclGetVal (Tcl\_Interp *) const;\\
  int				\>\>TclSetVal (Tcl\_Interp *, const char *val);\-\\
\#endif /* TCL */\\
\#endif /* META */\\
\};
\end{Ccode}

The {\C AsnOid} stores OBJECT IDENTIFIER values in their encoded form
to improve performance.  It seems that the most common operation with
OBJECT IDENTIFIERs is to compare for equality, for which the encoded
representation (which is canonical) works well.

The {\C AsnOid} is very similar to the {\C AsnOcts} class in all
respects, except that its content decoding routine does not need to
handle constructed encodings.

The {\C AsnOid} class has four constructors which are similiar to
those of the {\C AsnOcts} class.  A special constructor that takes
arc numbers as parameters and uses default parameters is provided.  An
OBJECT IDENTIFIER value must have at least two arc numbers so the
first two parameters do not have default values.  All of the other
parameters are optional; since their default value of {\C --1} is an
invalid arc number (they must be positive) they will not be used in the
value. For example to build the value {\C \{1 2 3\}} you simply use
{\C AsnOid (1, 2, 3)}.  This constructor is convenient but is more
expensive in terms of CPU time than the others.

The {\C operator char~*()} is defined for the {\C AsnOid} class to
return a pointer to the encoded OBJECT IDENTIFIER value.  The {\C Len}
method returns the length in bytes of the encode OBJECT IDENTIFIER
value (NOT the number arcs in the value).  These may be useful for
passing the octets to other functions such as {\C memcpy} etc.
{\C NumArcs} returns the number of arcs that the value is comprised of.

The {\C ==} and {\C !=} operators have been overloaded such that
given two {\C AsnOcts} values, they will behave as expected.



\section{\label{list-C++-section}SET OF and SEQUENCE OF}

In the C ASN.1 library, the list type was in the library because it
was generic and every SET OF and SEQUENCE OF was defined as an
{\C AsnList}.  In C++, a new class is defined every list, providing
a type safe list mechanism.  This was described in the previous chapter.


\section{\label{any-C++-section}ANY and ANY DEFINED BY}

The ANY DEFINED BY type can be handled automatically by snacc
provided you use the SNMP OBJECT-TYPE macro to specify the identifier
to type mappings.  The identifier can be an INTEGER or OBJECT
IDENTIFIER\@.  Handling ANY types properly will require modifications to
the generated code since there is no identifier associated with the
type.

Look at the C and C++ ANY examples and the {\ufn any.asn1} file
included with this release for information on using the OBJECT-TYPE
macro.  Note that the OBJECT-TYPE macro has been modified slightly to
allow INTEGER values (identifiers).

An ANY DEFINED BY type is represented by the {\C AsnAny} class.
The following is from {\ufn \dots/c++-lib/inc/asn-any.h}.
\begin{Ccode}
/* AnyInfo is a hash table entry */\\
class AnyInfo\\
\{\\
public:\+\\
  int				\>\>anyId;  // will be a value from the AnyId enum\\
  AsnOid			\>\>oid;    // will be zero len/null if intId is valid\\
  AsnInt			\>\>intId;\\
  AsnType			\>\>*typeToClone;\-\\
\};\\
\\
class AsnAny: public AsnType\\
\{\\
public:\+\\
  static Table			\>\>*oidHashTbl;  // all AsnAny class instances\\
  static Table			\>\>*intHashTbl;  // share these tables\\
  AnyInfo			\>\>*ai; // points to entry in hash tbl for this type\\
  AsnType			\>\>*value;\\
\\
				\>\>AsnAny() \{ ai = NULL; value = NULL; \}\\
\\
  // class level methods\\
  static void			\>\>InstallAnyByInt (AsnInt intId, int anyId, AsnType *type);\\
  static void			\>\>InstallAnyByOid (AsnOid \&oid, int anyId, AsnType *type);\\
\\
  int				\>\>GetId() \{ return ai ? ai-->anyId : -1; \}\\
  void				\>\>SetTypeByInt (AsnInt id);\\
  void				\>\>SetTypeByOid (AsnOid \&id);\\
\\
  AsnLen			\>\>BEnc (BUF\_TYPE b);\\
  void				\>\>BDec (BUF\_TYPE b, AsnLen \&bytesDecoded, ENV\_TYPE env);\\
\\
  int				\>\>BEncPdu (BUF\_TYPE b, AsnLen \&bytesEncoded);\\
  int				\>\>BDecPdu (BUF\_TYPE b, AsnLen \&bytesDecoded);\\
\\
  void				\>\>Print (ostream \&os) const \{ value-->Print (os); \}\\
\<\};
\end{Ccode}

The C++ mechanism is similar to the C mechanism which uses hash tables
to hold the identifier to type mappings.  In this section we will
discuss the main differences of the C++ ANY DEFINED BY handling
mechanism.  You should read Section~\ref{any-C-section} for caveats and
other important information.

In C, the hash table entry held the size of the type and pointers to
its encode, decode, free etc. routines to describe the type.  In C++
these have been replaced with a pointer to an instance of the type. A
hash table entry contains:
\begin{itemize}
\item {the {\C anyId}}
\item {the INTEGER or OBJECT IDENTIFIER that maps to it}
\item {a pointer to an instance of the identified type}
\end{itemize}

All C++ ASN.1 types use the {\C AsnType} base class which designates the following functions as virtual:
\begin{itemize}
  \item the destructor
  \item {\C Clone()}
  \item {\C BDec()}
  \item {\C BEnc()}
  \item {\C Print()}
  \item {\C \_getdesc()} (metacode)
  \item {\C \_getref()} (metacode)
  \item {\C TclGetDesc()} (Tcl interface)
  \item {\C TclGetVal()} (Tcl interface)
  \item {\C TclSetVal()} (Tcl interface)
  \item {\C TclUnsetVal()} (Tcl interface)
\end{itemize}

This allows the ANY DEFINED BY handling routines to treat a value of
any ASN.1 type as an {\C AsnType}.  So, for each type the ANY
DEFINED BY handling code has access to the virtual methods.  Note
that the {\C value} field in the {\C AsnAny} class and the
{\C typeToClone} field in the {\C AnyInfo} class are both
{\C AsnType~*}.

To build an ANY DEFINED BY value, simply set the value field in the
{\C AsnAny} object to point to the object you want to encode.  Then
set the identifier field for that ANY DEFINED BY value to the correct
identifier (as generated for its OBJECT-TYPE macro value).  It is very
important to do this correctly because the encoder will simply call
the virtual {\C BEnc} routine for the object pointed to by the
{\C AsnAny}'s value field.  There is no attempt to make sure that
the identifier field's value matches the object that was encoded.

A potential solution to the last identifier problem is to add a type
id field to the {\C AsnType} base class.  Snacc could generate a
unique identifier (that would be stored in the base class) for each
type.  The encoder could then check identifiers between the value
being encoded and the value stored in the hash table.  The identifier
in the base class could easily be automatically set (correctly) from
the constructors for each type (constructors are snacc generated).  It
would be difficult to ensure unique identifiers for each type between
modules if the ASN.1 modules were compiled separately.

Before an ANY DEFINED BY value can be decoded, the field that contains
its identifier must have been decoded and used with the {\C AsnAny}
value's {\C SetTypeByInt} or {\C SetTypeByOid} methods.  Then the
ANY DEFINED BY value can be decoded by calling its ({\C AsnAny})
{\C BDec} routine.  This in turn calls the {\C Clone} routine on
the type in the hash table entry to generate the correct object.  Then
the {\C BDec} method of the newly created object is called.

When the C ANY DEFINED BY decoder allocates a value, it uses the size
information for the identified type.  This is not safe for C++ so the
virtual {\C Clone} routine was added to the {\C AsnType} base
class.  This allows the proper constructor mechanism to be used when
allocating the value.

The virtual {\C Clone} routine simply calls its type's parameterless
constructor via {\C new} (hence every ASN.1 type's class must have a
parameterless constructor).  {\C Clone} is a poor name since
the routine only produces a new instance of the given type without
copying the original's data.

The hash tables are automatically initialized using the C++ constructor
mechanism.  You do not need to call any initialization routines as
described in the C chapter.


\section{\label{buffer-C++-section}Buffer Management}

The C++ buffer management provided with snacc is similar to that of
the C {\C SBuf}s.  The following is from
{\ufn \dots/c++-lib/inc/asn-buf.h}:
\begin{Ccode}
class AsnBuf\\
\{\\
protected:\+\\
  char			\>\>*dataStart;\\
  char			\>\>*dataEnd;\\
  char			\>\>*blkStart;\\
  char			\>\>*blkEnd;\\
  char			\>\>*readLoc;\\
  bool			\>\>writeError;\\
  bool			\>\>readError;\\
\\
\<public:\\
  void			\>\>Init (char *data, size\_t dataLen);\\
  void			\>\>ResetInReadMode();\\
  void			\>\>ResetInWriteRvsMode();\\
  void			\>\>InstallData (char *data, size\_t dataLen);\\
  size\_t		\>\>DataLen();\\
  char			\>\>*DataPtr();\\
  size\_t		\>\>BlkLen();\\
  char			\>\>*BlkPtr();\\
  bool			\>\>Eod();\\
\\
  /* the following 9 methods are required */\\
  bool			\>\>ReadError();\\
  bool			\>\>WriteError();\\
  void			\>\>Skip (size\_t skipLen);\\
  size\_t		\>\>CopyOut (char *dst, size\_t copyLen);\\
  unsigned char		\>\>PeekByte();\\
  char			\>\>*GetSeg (size\_t *lenPtr);\\
  void			\>\>PutSegRvs (char *seg, size\_t segLen);\\
  unsigned char		\>\>GetByte();\\
  void			\>\>PutByteRvs (unsigned char byte);\\
\<\};
\end{Ccode}

This is the only buffer type provided with the C++ library.

\section{\label{mem-C++-section}Dynamic Memory Management}

The built-in C++ memory management system is used by snacc ({\C new}
and {\C delete}).  Better performance might be gained by using a
different management scheme.

To change {\C new} and {\C delete} to use your own memory
management scheme the best way to start is by defining them as virtual
in the {\C AsnType} base class.  More information on providing your
own memory management can be found in Stroustrup \cite{stroustrup}.

\section{\label{error-C++-section}Error Management}

The C++ ASN.1 error management is identical to that of the C ASN.1
model.  C++ exception handling ({\C try} and {\C throw}) were not
used to replace {\C setjmp()} and {\C longjmp()} because they were not
implemented by the C++ compiler we used.


% file: .../doc/meta.tex

% $Header: /usr/app/odstb/CVS/snacc/doc/meta.tex,v 1.1 1997/01/01 22:47:38 rj Exp $
% $Log: meta.tex,v $
% Revision 1.1  1997/01/01 22:47:38  rj
% first check-in
%

\chapter{\label{meta-chapter}The Metacode}

%-----------------------------------------------------------------------------------------------------------------------------------
\section{\label{intro-meta-section}Introduction}

When you call snacc, during its compilation, the text in the {\ufn .asn1} files gets turned into e.g. C++ classes.
Names become identifiers, and after the C++ compilation, the user program has no more access to the original module and type names, only to pointers and the bits and bytes of the classes' contents.

The metacode remedies this.
Using it, a program can access the modules, their types, their subtypes and the named values via strings.
Generic programs do not have to know any of the modules' or types' names---all the information can be traversed starting at a single well-known place.

The metacode has to map strings (``component {\C "name"} in type {\C "Person"} in ASN.1 module {\C "Mail"}'') into the in-core address of the indicated object.
Moreover, using the metacode, a generic program has to be able to traverse all modules, types and their components to learn about their names and types.
Given the name of a type, the metacode must be able to return a newly allocated object instance.

The metacode is an extension to the Snacc compiler's C++ backend and the C++ runtime library.

Since the metacode relies heavyly on the virtual function call mechanism, it is only implemented for the C++ backend.

A number of functions has been added to the C++ runtime library.
All code extensions have been wrapped into preprocessor conditionals.
Currently, only one or two libraries are made in {\ufn \dots/c++-lib/}, one with neither metacode nor Tcl interface, and an optional additional one with both extensions.
If you cannot or do not want to (as stated in {\ufn \dots/policy.h}) use the metacode or the Tcl interface, Snacc will be compiled without it, and only the normal library will get made.

If you want to use the metacode but not the Tcl interface, you have got to change the makefile to compile another set of {\ufn .o} files in a directory you may want to name {\ufn \dots/c++-lib/meta/} and archive the resulting files in {\ufn libasn1meta.a}.
Compile with {\ufn -DMETA=1 -DTCL=0}.

%-----------------------------------------------------------------------------------------------------------------------------------
\section{Implementation}

%leftover:----------------
With the metacode, the strings and the ASN.1 type components' type information are stored in arrays, one per structured ASN.1 type.
The type descriptions are listed in another array, one per ASN.1 module.

In the C++ code generated by snacc, every ASN.1 type is represented by a C++ class.
These C++ classes for ASN.1 simple types are implemented in the runtime library, and for structured types they are generated by the backend.
The metacode is an extension to all of those C++ classes.
The metacode put into every C++ class is very similar:
\begin{itemize}
  \item a static {\C \_desc} member is always present.
  \item simple types with names (ENUMERATED, INTEGER, BIT STRING) get an additional static {\C \_nmdescs[]} member.
    The array is exclusively referenced from {\C \_desc}.
    The array provides the bidirectional mapping of symbolic and numeric values.
  \item structured types with members (SET, SEQUENCE, CHOICE) get an additional static {\C \_mdescs[]} member.
    The array is exclusively referenced from {\C \_desc}.
    The array references the components' type descriptions (their {\C \_desc} data members).
  \item every class gets a virtual {\C \_getref()} function.
    Its only purpose is to return the address of {\C \_desc}\footnote{
      This sounds as if virtual data members were a nice idea, and in fact they are.
      The C++ standards committees are currently discussing this.
    %  Of course, there are not a lot of compilers already implementing this feature.
    }.
  \item structured types with members (SET, SEQUENCE, CHOICE) get an additional virtual {\C \_getref()} function.
    This function provides the member name to member address mapping.
\end{itemize}
The data members are {\C static}, and therefore get instantiated exactly once per executable, not once per class object instance.

To get an impression, let us have a look at an example:
The two ASN.1 types
\begin{ASNcode}
File ::= SET\\
\{\+\\
  name [0] PrintableString,\\
  contents [1] OCTET STRING,\\
  checksum [2] INTEGER OPTIONAL,\\
  read-only [3] BOOLEAN DEFAULT FALSE\-\\
\}\\
\\
Directory ::= SET\\
\{\+\\
  name PrintableString,\\
  files SET OF File\-\\
\}
\end{ASNcode}

get turned into two individual C++ classes:

\begin{Ccode}
class File: public AsnType\\
\{\\
public:\+\\
  PrintableString               \>\>name;\\
  AsnOcts               \>\>contents;\\
  AsnInt                \>\>*checksum;\\
  AsnBool               \>\>*read\_only;\\
\\
\<\#if META\\
  static const AsnSetTypeDesc   \>\>\_desc;\\
  static const AsnSetMemberDesc \>\>\_mdescs[];\\
  const AsnTypeDesc             \>\>*\_getdesc() const;\\
  AsnType                       \>\>*\_getref (const char *membername, bool create = false);\\
\<\#endif // META\\
\\
  // \dots other functions omitted\dots\-\\
\};\\
\\
class Directory: public AsnType\\
\{\\
public:\+\\
  PrintableString               \>\>name;\\
  DirectorySetOf                \>\>files;\\
\\
\<\#if META\\
  static const AsnSetTypeDesc   \>\>\_desc;\\
  static const AsnSetMemberDesc \>\>\_mdescs[];\\
  const AsnTypeDesc             \>\>*\_getdesc() const;\\
  AsnType                       \>\>*\_getref (const char *membername, bool create = false);\\
\<\#endif // META\\
\\
  // \dots other functions omitted\dots\\
\<\};
\end{Ccode}

The above definitions stem from the {\ufn .h} file, the following code is taken from the {\ufn .C} file.
Only the code for the {\ASN Directory} type is shown, because the code for the {\ASN File} type looks very similar.

\begin{Ccode}
\#if META\\
\\
static AsnType *createDirectory()\\
\{\\
  \>return new Directory;\\
\}\\
\\
const AsnSetMemberDesc Directory::\_mdescs[] =\\
\{\+\\
  AsnSetMemberDesc ("name", \&PrintableString::\_desc, false), // `name'\\
  AsnSetMemberDesc ("files", \&DirectorySetOf::\_desc, false), // `files'\\
  AsnSetMemberDesc()\-\\
\};\\
\\
const AsnSetTypeDesc Directory::\_desc\\
(\+\\
  \&EdEx\_StructuredModuleDesc,\\
  "Directory", // `Directory'\\
  true,\\
  AsnTypeDesc::SET,\\
  createDirectory,\\
  \_mdescs\-\\
);\\
\\
const AsnTypeDesc *Directory::\_getdesc() const\\
\{\\
  \>return \&\_desc;\\
\}\\
\\
AsnType *Directory::\_getref (const char *membername, bool create)\\
\{\+\\
  if (!strcmp (membername, "name"))\\
    \>return \&name;\\
  if (!strcmp (membername, "files"))\\
    \>return \&files;\\
  return NULL;\-\\
\}\\
\\
\#endif // META
\end{Ccode}

The two ASN.1 types get turned into two individual C++ classes, but their {\C \_desc} members point to two different instances of the same type.
The C++ backend generates an individual C++ class for every structured ASN.1 type.
The metacode is different: 
The types of the descriptions data members ({\C \_desc}, {\C \_nmdesc} and {\C \_mdesc}) despend on the \emph{general} ASN.1 type (i.e. e.g. {\ASN SET}, not {\ASN File}).







%leftover:----------------
Every {\ufn .asn1} file contains an ASN.1 module and gets translated into a {\ufn .C} file that contains an array that lists all the module's type descriptions.

Snacc generates an additional file {\ufn modules.C} that contains an array that lists all the module descriptions.
This single top level array provides the well known entry point for the whole module and type hierarchy.
%leftover:----------------





The type of the {\C \_desc} member differs depending on the ASN.1 type it describes.
The different {\C \_desc} types mirror the {\C AsnType} class hierarchy.
For example, the ASN.1 BOOLEAN type is mapped into a C++ class called {\C AsnBool} and is described by a C++ class {\C AsnBoolTypeDesc}.
The root of the {\C \_desc} class hierarchy is called {\C AsnTypeDesc} and looks as follows
(taken from {\ufn \dots/c++-lib/inc/meta.h}):
\begin{Ccode}
struct AsnTypeDesc\\
\{\+\\
  const AsnModuleDesc		\>\>*module;\\
  const char			\>\>*const name;	// NULL for basic types\\
  const bool			\>\>pdu;\\
  const enum Type \qquad // NOTE: keep this enum in sync with the typenames[]\\
  \{\+\\
    VOID,\\
    ALIAS,\\
\\
    INTEGER,\\
    REAL,\\
    NUL\_, // sic! (can't fight the ubiquitous NULL \#define)\\
    BOOLEAN,\\
    ENUMERATED,\\
    BIT\_STRING,\\
    OCTET\_STRING,\\
    OBJECT\_IDENTIFIER,\\
\\
    SET,\\
    SEQUENCE,\\
    SET\_OF,\\
    SEQUENCE\_OF,\\
    CHOICE,\\
    ANY,\-\\
  \}				\>\>type;\\
\\
  static const char		\>\>*const typenames[];\\
\\
				\>\>AsnTypeDesc (const AsnModuleDesc *, const char *,\\
					\`bool ispdu, AsnType *(*create)(), Type);\\
\\
  AsnType			\>\>*(*create)();\\
\\
  virtual const AsnModuleDesc	\>\>*getmodule() const;\\
  virtual const char		\>\>*getname() const;\\
  virtual bool			\>\>ispdu() const;\\
  virtual Type			\>\>gettype() const;\\
  virtual const AsnNameDesc	\>\>*getnames() const;\\
  %//virtual const AsnMemberDesc	\>\>*getmembers() const;\\
\\
\<\#if TCL\\
  virtual int			\>\>TclGetDesc (Tcl\_DString *) const;\\
  virtual int			\>\>TclGetDesc2 (Tcl\_DString *) const;\-\\
\#endif\\
\};\\
\\
typedef AsnTypeDesc		\>\>\>AsnRealTypeDesc;\\
typedef AsnTypeDesc		\>\>\>AsnNullTypeDesc;\\
typedef AsnTypeDesc		\>\>\>AsnBoolTypeDesc;\\
\\
typedef AsnTypeDesc		\>\>\>AsnOctsTypeDesc;\\
typedef AsnTypeDesc		\>\>\>AsnOidTypeDesc;
\end{Ccode}

{\C AsnTypeDesc}'s data members provide the following information:
\begin{description}
  \item[\textnormal{The} {\C module}] data member points to the module description described at the end of this chapter in section~\ref{meta-modules-sect}.

  \item[\textnormal{The} {\C name}] is either the type's name as used by the backend code (default, or if snacc was called with the {\ufn -mC} switch) or the type's given names as defined in the {\ufn .asn1} file (if snacc has been called with the {\ufn -mA} command line option).
    Section~\ref{naming-C++-section} on page~\pageref{naming-C++-section} explains the differences.
    (For example, the backend code sometimes has an additional number tacked to the name---you can see the effect in figure~\ref{snacced-example} on page~\pageref{snacced-example}.)
    The generated source code contains the respective counterpart printed in a comment.

  \item[\textnormal{The} {\C pdu}] flag is set to {\C true} iff the type was listed after snacc's {\ufn -meta} or {\ufn -tcl} switch.

  \item[\textnormal{The} {\C type}] member is used as an index into the {\C typenames} array---the virtual function call mechanism obliviates the use for any {\C switch} statements.

  \item[\textnormal{The} {\C create}] data member points to a global function that returns a pointer to a newly allocated object of the description type's mirror type, that is, gives you an instance for the generic description.
    It is the counterpart to the {\C AsnType}'s {\C \_getdesc} function which goes in the opposite direction, from the object instance to its generic description.
    The {\C AsnType}'s {\C Clone} function serves a similar purpose as the {\C AsnTypeDesc}'s {\C create} function.
\end{description}

The {\C AsnTypeDesc} class is the only class in the hierarchy that has got the {\C module}, {\C name}, {\C pdu} and {\C type} data members, and {\C AsnNamesTypeDesc} the only class to implement a {\C names} data member.
Therefore, unlike {\C \_getdesc()} mentioned above, the five virtual functions {\C getmodule}, {\C getname}, {\C ispdu}, {\C gettype} and {\C getnames} are not meant to implement some kind of virtual data members, but help to implement the alias type description functionality described in section~\ref{meta-aliases}.

As you can see looking at the last five code lines with the {\C typedef}s, the five ASN.1 simple types REAL, NULL, BOOLEAN, OCTET STRING and OBJECT IDENTIFIER are directly described by instances of this class.
The other types, having either named values or components, are more demanding and have their own classes derived from {\C AsnTypeDesc}.

%-----------------------------------------------------------------------------------------------------------------------------------
\subsection{Named Values}

Some basic ASN.1 types allow values to be named, namely INTEGER, ENUMERATED and BIT STRING.
The accompanying description types contain arrays listing the names and values.
The virtual function {\C getnames()} returns this array.
The respective C++ classes in the runtime library contain an {\C AsnNameDesc \_nmdescs[]} array, the address of which is given to the type descriptions constructor as last argument.

\begin{Ccode}
struct AsnNameDesc\\
\{\+\\
  const char			\>\>*const name;\\
  const long int		\>\>value;\-\\
\};\\
\\
struct AsnNamesTypeDesc: AsnTypeDesc\\
\{\+\\
  const AsnNameDesc		\>\>*const names;\\
\\
				\>\>AsnNamesTypeDesc (const AsnModuleDesc *, const char *,\\
					\`bool ispdu, AsnType *(*create)(), Type, const AsnNameDesc *);\\
\\
  const AsnNameDesc		\>\>*getnames() const;\\
\\
\<\#if TCL\\
  int				\>\>TclGetDesc (Tcl\_DString *) const;\\
  // for BIT STRING and INTEGER, ENUMERATED has its own:\\
  int				\>\>TclGetDesc2 (Tcl\_DString *) const;\-\\
\#endif\\
\};\\
\\
struct AsnEnumTypeDesc: AsnNamesTypeDesc\\
\{\\
  \>				\>\>AsnEnumTypeDesc (const AsnModuleDesc *, const char *,\\
					\`bool ispdu, Type, AsnType *(*create)(), const AsnNameDesc *);\\
\\
\#if TCL\\
  \>int				\>\>TclGetDesc2 (Tcl\_DString *) const;\\
\#endif\\
\};\\
\\
typedef AsnNamesTypeDesc	\>\>\>AsnIntTypeDesc;\\
typedef AsnNamesTypeDesc	\>\>\>AsnBitsTypeDesc;
\end{Ccode}

The ENUMERATED type gets its own description class because the Tcl interface for ENUMERATED types behaves differently than for the INTEGER and BIT STRING types.

As for {\C AsnTypeDesc::name} above, the content of {\C AsnNameDesc::name} is either the value's name as used by the backend code (default, or if snacc was called with the {\ufn -mC} switch) or the value's name as given in the {\ufn .asn1} file (if snacc has been called with the {\ufn -mA} command line option).
The generated source code contains the respective counterpart printed in a comment.

\subsection{Types with Members}

The ASN.1 types CHOICE, SET and SEQUENCE are defined in terms of other types, their so-called components.
The ASN.1 components map into C++ data members.

The three ASN.1 structured types get mapped into C++ classes that contain an {\C Asn\dots{}MemberDesc \_mdescs[]} array (with the `{\C \dots}' replaced by `{\C Choice}' `{\C Set}' or `{\C Sequence}').
The address of this array is given to the description type's constructor as last argument.
The elements of this array point to the descriptions of the data classes data members.
This is similar to the named values above, only the integral value has been replaced by a pointer to a type description.

\begin{Ccode}
struct AsnMemberDesc // description of CHOICE member; base class for AsnSe\_MemberDesc\\
\{\+\\
  const char			\>\>*const name;\\
  const AsnTypeDesc		\>\>*const desc;\\
\\
				\>\>AsnMemberDesc (const char *, const AsnTypeDesc *);\\
				\>\>AsnMemberDesc();\\
\\
\<\#if TCL\\
  virtual int			\>\>TclGetDesc (Tcl\_DString *) const;\\
  virtual int			\>\>TclGetDesc2 (Tcl\_DString *) const;\-\\
\#endif\\
\};\\
\\
struct AsnSe\_MemberDesc: AsnMemberDesc\qquad// \_ == t/quence; description of SET or SEQUENCE member\\
\{\+\\
  bool				\>\>optional;\\
\\
				\>\>AsnSe\_MemberDesc (const char *, const AsnTypeDesc *, bool);\\
				\>\>AsnSe\_MemberDesc();\\
\-\\
\#if TCL\\
  \>int				\>\>TclGetDesc2 (Tcl\_DString *) const;\\
\#endif\\
\};\\
\\
typedef AsnMemberDesc		\>\>\>AsnChoiceMemberDesc;\\
typedef AsnSe\_MemberDesc	\>\>\>AsnSetMemberDesc;\\
typedef AsnSe\_MemberDesc	\>\>\>AsnSequenceMemberDesc;\\
\\
struct AsnMembersTypeDesc: AsnTypeDesc\\
\{\+\\
				\>\>AsnMembersTypeDesc (const AsnModuleDesc *, const char *,\\
					\`bool ispdu, AsnType *(*create)(), Type);\\
\\
\<\#if TCL\\
  int				\>\>TclGetDesc (Tcl\_DString *) const;\-\\
\#endif\\
\};\\
\\
struct AsnChoiceTypeDesc: AsnMembersTypeDesc\\
\{\+\\
  const AsnChoiceMemberDesc	\>\>*const members;\\
\\
				\>\>AsnChoiceTypeDesc (const AsnModuleDesc *, const char *,\\
					\`bool ispdu, AsnType *(*create)(), Type, const AsnChoiceMemberDesc *);\\
\\
  int				\>\>choicebyname (const char *name) const;\\
  const char			\>\>*choicebyvalue (int value) const;\\
\\
\<\#if TCL\\
  int				\>\>TclGetDesc2 (Tcl\_DString *) const;\-\\
\#endif\\
\};\\
\\
struct AsnSe\_TypeDesc: AsnMembersTypeDesc\qquad// \_ == t/quence\\
\{\+\\
  const AsnSe\_MemberDesc	\>\>*const members;\\
\\
				\>\>AsnSe\_TypeDesc (const AsnModuleDesc *, const char *,\\
					\`bool ispdu, AsnType *(*create)(), Type, const AsnSe\_MemberDesc *);\\
\\
\<\#if TCL\\
  int				\>\>TclGetDesc2 (Tcl\_DString *) const;\-\\
\#endif\\
\};\\
\\
typedef AsnSe\_TypeDesc		\>\>\>AsnSetTypeDesc;\\
typedef AsnSe\_TypeDesc		\>\>\>AsnSequenceTypeDesc;
\end{Ccode}

As for {\C AsnTypeDesc::name} above, the content of {\C AsnMemberDesc::name} is either the member's name as used by the backend code (default, or if snacc was called with the {\ufn -mC} switch) or the component's name as defined in the {\ufn .asn1} file (if snacc has been called with the {\ufn -mA} command line option).
The generated source code contains the respective counterpart printed in a comment.
In case the ASN.1 component was not given a name, the backend's member name is used instead.

The data classes have a member function called {\C \_getref}, that allows the C++ class members to be accessed by their name.
{\C \_getref()} is the second metacode function and it is present in all C++ classes representing composed ASN.1 types.

A class for a SET contains the following code fragment:
\begin{Ccode}
class FooSet: public AsnType\+\\
  AsnInt	\>\>bar; // an example data member\\
  \dots // lots of member functions\\
\<\#if META\\
  static const AsnSetTypeDesc   \>\>\_desc;\\
  static const AsnSetMemberDesc \>\>mdescs[];\\
  const AsnTypeDesc             \>\>*\_getdesc() const;\\
  AsnType                       \>\>*\_getref (const char *membername, bool create = false);\\
\<\#if TCL\\
  int                \>\>TclGetDesc (Tcl\_DString *) const;\\
  int                \>\>TclGetVal (Tcl\_Interp *) const;\\
  int                \>\>TclSetVal (Tcl\_Interp *, const char *valstr);\\
  int                \>\>TclUnsetVal (Tcl\_Interp *, const char *membername);\-\\
\#endif // TCL \\
\#endif // META\\
\};
\end{Ccode}

{\C \_getref()}'s {\C bool} parameter {\C create} determines whether a non-existing member should be returned as a {\C NULL} pointer or whether it should instead be allocated and its address be returned.
This parameter is used by value reading and writing routines to implement their different member access semantics.

The following four assignments are equivalent:
\begin{Ccode}
FooSet foo;\\
foo.bar = 1;\\
*(AsnInt *)foo.\_getref ("bar") = 1;\\
foo.bar.TclSetVal (interp, "1");\\
foo.\_getref ("bar")-->TclSetVal (interp, "1");
\end{Ccode}
{\C TclSetVal()} is a virtual member function and therefore no cast from {\C AsnType~*} to {\C AsnInt~*} is required.
The Tcl interface will be described in chapter~\ref{tcl-if-chapter}.

The C++ classes that represent CHOICE types contain an {\C enum ChoiceIdEnum} that allows {\C \_getref()} to be written using a {\C switch} statement.
The functions {\C choicebyname()} and {\C choicebyvalue()} turn the component's name into its enumeration value and vice versa.
(The enum has not been introduced with the metacode, it is used by Snacc's encoding and printing functions as well.)

\subsection{\label{meta-list}SET OF and SEQUENCE OF}

The list description behaves like an ASN.1 simple type's---the description type is derived directly from the type descriptions' base class and does not redefine any of the metacode functions:

\begin{Ccode}
struct AsnListTypeDesc: AsnTypeDesc\\
\{\+\\
  const AsnTypeDesc		\>\>*const base;\\
\\
  				\>\>AsnListTypeDesc (const AsnModuleDesc *, const char *,\\
					\`bool ispdu, Type, AsnType *(*create)(), const AsnTypeDesc *);\-\\
\\
\#if TCL\\
  \>int				\>\>TclGetDesc (Tcl\_DString *) const;\\
\#endif\\
\};
\end{Ccode}

The {\C TclGetDesc} function merely adds the base type's standard type description (module and type name, pdu flag and type) after its own, so that a programmer may take the base type's name and ask the metacode once again for the base type's full description.

A list type's data class on the other hand has got a {\C \_getref()} function that gives access to the list's elements and it can be used to insert new elements at any desired position.

\subsection{\label{meta-aliases}Aliases}

For ASN.1 types being defined as a direct copy of another type, snacc in normal operation uses a C++ {\C typedef} to define the C++ type.
Since this {\C typedef} makes the two types totally equivalent, the metacode has no chance to preserve the two types' different names and thus, this contruct cannot be used.
A new C++ class has got to be defined instead.

Example: the following ASN.1 code snippet\dots
\begin{ASNcode}
Int1 ::= INTEGER \{ foo(42) \}\\
Int2 ::= Int1
\end{ASNcode}
\dots maps into the following C++ definitions:
\begin{Ccode}
class Int1: public AsnInt\\
\{\\
public:\+\\
                        \>\>Int1(): AsnInt() \{\}\\
                        \>\>Int1 (int i): AsnInt (i) \{\}\\
  enum\\
  \{\\
    \>foo = 42\\
  \};\\
\\
\<\#if META\\
  static const AsnNameDesc      \>\>\_nmdescs[];\\
  static const AsnIntTypeDesc   \>\>\_desc;\\
  const AsnTypeDesc     \>\>*\_getdesc() const;\-\\
\#endif // META\\
\};\\
\\
\#if META\\
struct Int2: public Int1\\
\{\+\\
                        \>\>Int2(): Int1() \{\}\\
                        \>\>Int2 (int i): Int1 (i) \{\}\\
  AsnType               \>\>*Clone() const;\\
\\
  static const AsnAliasTypeDesc \>\>\_desc;\\
  const AsnTypeDesc     \>\>*\_getdesc() const;\-\\
\};\\
\\
\#else // META\\
\\
typedef Int1 \>\>\>Int2;\\
\\
\#endif // META
\end{Ccode}

The descriptor type's definition points to the reference type:

\begin{Ccode}
struct AsnAliasTypeDesc: AsnTypeDesc\\
\{\+\\
  const AsnTypeDesc		\>\>*const alias;\\
\\
				\>\>AsnAliasTypeDesc (const AsnModuleDesc *, const char *,\\
					\`bool ispdu, AsnType *(*create)(), Type, const AsnTypeDesc *);\\
\\
  const AsnModuleDesc		\>\>*getmodule() const;\\
  const char			\>\>*getname() const;\\
  bool				\>\>ispdu() const;\\
  Type				\>\>gettype() const;\\
  const AsnNameDesc		\>\>*getnames() const;\\
  %//const AsnMemberDesc		\>\>*getmembers() const;\\
\\
\<\#if TCL\\
  int				\>\>TclGetDesc (Tcl\_DString *) const;\-\\
\#endif\\
\};
\end{Ccode}

The {\C AsnAliasTypeDesc} is the reason for the five virtual functions from {\C getmodule} to {\C getnames} defined in both {\C AsnTypeDesc} and {\C AsnNamesTypeDesc} on the one hand and {\C AsnAliasTypeDesc} on the other hand.
While the alias type belongs to a different module or has another type name, and it may have another {\C pdu} flag value, its type and names array values are those of its reference type.
Therefore, {\C AsnAliasTypeDesc}'s first three functions of return the description's own values, and the latter two call their reference type's functions.

The {\C getnames} function has to be defined in the hierarchy's base class because the aliases may be defined for any type of type, not only for types with named values.

\subsection{\label{meta-any}ANY (DEFINED BY)}

ANY DEFINED BY is quite problematic.
The ASN.1 Book \cite{ASN.1Book} calls it ``a rather half-baked attempt at solution''.
Since snacc has problems with it---the user has to modify the snacc generated code---and none of our applications requires this construct, no effort has been made to implement it.

ANY itself on the other hand would be quite simple to implement---the virtual function call mechanism that is used to implement the ANY type is the basis for the metacode as well.
But again, since we have no need for the ANY type, it is as far unimplemented.
Besides that, according to the ASN.1 book, the ``use of ANY without the DEFINED BY construct is ``deprecated'' (frowned upon) by the standard''.
The next ASN.1 standard will probably not have the ANY type any more.
In the 1993 draft standard \cite{asn1:1993}, ANY and ANY DEFINED BY can be found in ``Annex I: Superseded features'', Section 3: ``The any type''.
% Macros can be found in this annex as well.

\subsection{\label{meta-modules-sect}Modules}

Every {\ufn .C} file (that corresponds to an {\ufn .asn1} file, or, an ASN.1 module), gets an array that lists all the module's types.
This array contains pointers to all the {\C \_desc} members of all classes of a module.

\begin{Ccode}
struct AsnModuleDesc\\
\{\+\\
  const char			\>\>*const name;\\
  const AsnTypeDesc		\>\>**const types;\-\\
\};\\
\\
extern const AsnModuleDesc	\>\>\>*asnModuleDescs[];
\end{Ccode}

The modules themselves are listed in yet another array, the declaration of which is shown in the preceeding line.
This array has got its own source file named {\ufn modules.C}.
This array allows all modules to be found, and every type that is defined for these modules.

%-----------------------------------------------------------------------------------------------------------------------------------
\section{Efficiency}

The metacode is designed with efficiency in mind.
The metacode is intended for interpreted interfaces and therefore does not need to be highly optimized.
On the other hand, the same object code should be useable for normal (non-metacode) tasks without loss of performance.

\subsection{Normal Operation}

The metacode does not significantly affect the normal mode of operation.
The static data members {\C \_mdescs} and {\C \_desc} do not increase the class instances' size.
The virtual function tables, which have already been present (they are used for the ANY type), get a little longer, but since these tables exist only once for every class, this difference is neglible.
The class instances reference their virtual function table with a pointer, and so the metacode does not introduce any change here.
Except for alias types, the C++ classes generated are exactly the same.
The metacode introduces a new class for alias types, but since no new data members are introduced their size stays the same; only the virtual function table pointer is different.

All normal member functions (constructor, destructor, assignment operator, encode, decode and print functions) are identical---with only one exception: if the metacode is compiled to be usable by the Tcl interface, the constructors initialize their mandatory members.

To sum it up, both code and data grow, but except for a longer loading time from disk and an increased probability for cache misses, the code will run as fast as it does without the metacode.

\subsection{Metacode}

The metacode routines are kept quite simple.
Intended to be used in conjunction with a Tcl interface, speed was not the most important concern.
Consequently, the code is optimized more towards memory usage than run time efficiency.
As an example, name to member resolution uses a linear lookup strategy instead of more elaborated algorithms like binary search or hash tables.
I think for data types that typically have up to a dozen components, more sophisticated algorithms would have been overkill.

A typical object file gets almost 20\% larger due to the metacode (the Tcl interface adds another 25\%).

%-----------------------------------------------------------------------------------------------------------------------------------
\section{\label{meta-ttab-comparison}Metacode Vs. Type Tables}

Here's a list of both the type tables' (see chapter~\ref{ttab-chapter}) and the metacode's (dis)advantages:
\begin{itemize}

  \item source code language:
  \begin{itemize}
    \item[$-$] The type tables are implemented for C only.
    \item[$-$] The metacode works only for C++.
  \end{itemize}

  \item speed:
  \begin{itemize}
    \item[$-$] Encoding and decoding using the type tables is said to be about 4 times slower than using the C routines.
    \item[$+$] The metacode does not (significantly) harm performance.
  \end{itemize}

  \item code size:
  \begin{itemize}
    \item[$+$] The tables are a lot smaller than the compiled routines.
    \item[$-$] The metacode makes the compiled code even larger.
  \end{itemize}

  \item value constants:
  \begin{itemize}
    \item[$-$] The type tables lack the values defined in the {\ufn .asn1} files.
    \item[$+$] The metacode interacts fine with these values.
  \end{itemize}

  \item named values:
  \begin{itemize}
    \item[$-$] The type tables lack the named values defined ENUMERATED, INTEGER and BIT STRING types.
    \item[$+$] The metacode interacts fine with these names.
  \end{itemize}

  \item compatibility to normal snacc code:
  \begin{itemize}
    \item[$-$] The C structures defined by mkchdr and used by the type table encoding and decoding routines and the C structures defined by snacc's C backend are quite different.
    \item[$\pm$] Where the backend's structures generated for SEQUENCE contain madatory members by value, the type table's structures contain only pointer members!
  \end{itemize}

\end{itemize}

%-----------------------------------------------------------------------------------------------------------------------------------
\section{\label{metacode-setup}Setup for the Metacode Generator}

To compile Snacc with the metacode generator, the following condition must be met:
\begin{itemize}
  \item either the configure script must be able to find {\ufn tclsh} and the Tcl/Tk libraries or you have to insert a {\C \#define META 1} into {\ufn \dots/policy.h}
  \item the {\C NO\_META} preprocessor macro in {\ufn \dots/policy.h} must not be set
\end{itemize}

% file: .../doc/tcl.tex

% $Header: /usr/app/odstb/CVS/snacc/doc/tcl.tex,v 1.1 1997/01/01 22:47:40 rj Exp $
% $Log: tcl.tex,v $
% Revision 1.1  1997/01/01 22:47:40  rj
% first check-in
%

\chapter{\label{tcl-if-chapter}Tcl Interface}

\section{\label{intro-tcl-section}Introduction}

This chapter describes the Snacc's Tcl interface, or: the metacode's link to the outside world.

Tcl is a simple scripting language which the author, John K. Ousterhout, describes in his book titled ``Tcl and the TK Toolkit'' \cite{tcl-book}.
Tcl's purpose is to be embedded into other applications, to provide a user interface by extending the language.
Tk, an implementation of the Motif look and feel, is the first and best known extension to Tcl and is described in the same book.

Tcl has got only one data type, the NUL terminated character string.
Tcl supports other data types like integers and lists, but they are represented as strings.
A function operating on an integer first converts the string into an integer, performs its operation, converts the resulting value back into another string and returns it to the Tcl interpreter.
Since lists and even the Tcl procedures are kept as strings, Tcl is rather slow.
Computations in Tcl should best be kept at a minimum, and all intensive work should be wrapped into C or C++ functions and be made available as Tcl commands.

Since procedures and bodies of loops are kept in string form and parsed for every invocation, comments should be put outside code that is executed \emph{very} often.

From Tcl's point of view, Snacc's Tcl interface is nothing but yet another Tcl extension.
The Snacc Tcl interface extends the Tcl language by only one command, {\Tcl snacc}.
The first argument to this command specifies the action to be taken.
This method is very practical for combining Tcl extensions since it avoids collisions with new command names from other extensions.
For example, the Tcl core defines an {\Tcl open} command.
Snacc's Tcl interface wants to offer one as well and has to choose another name.
This could have been done by naming it {\Tcl snacc\_open}, but I think it is better to stick to Tcl's well established convention and so the Tcl interface's open command became {\Tcl snacc open}.
To simplify the wording, I will refer to the `snacc subcommands' simply as `commands'.

The usual (non-metacode) snacc generated functions operate on memory buffers containing BER encoded data; they convert them into hierarchical C++ data structures and vice versa.

The Tcl interface is designed to allow controlled fine grained access to this hierarchical C++ data structure, to read and modify its contents.
While both the C++ code and the Tcl look very similar, for example\dots\\[1ex]
{\C
// this is C++ code\\
x-->foo-->bar = 42;\\[1ex]
}
\dots\ and\dots\\[1ex]
{\Tcl
\# this is Tcl code\\
snacc set \{x foo bar\} 42
}

\dots\ the C++ code gets compiled and the identifiers get turned into pointers and numeric offsets, and the Tcl code gets interpreted and has to mimic the C++ compiler at run time.
This is what the metacode from chapter~\ref{meta-chapter} is for.

To enable snacc's Tcl code generator, you have to give an additional {\ufn -tcl} option, followed by the list of PDU types.
The {\ufn -meta} option can (and should) be omitted.

\section{The {\Tcl snacc} Tcl command}

This section explains the Tcl (sub)commands provided by the Snacc extension.
The commands are grouped in three catagories, commands operating on files (both their external and internal representation), commands accessing the meta information and commands operating on the content itself.

The file commands check the return value from system calls and behave like for example the Tcl {\Tcl open} command, that is, they set the {\Tcl errorCode} variable to {\Tcl POSIX \emph{errno}}, e.g. {\Tcl POSIX ENOENT \{No such file or directory\}}.

The code should be fairly robust, not just against user and programmer errors from `outside' (using the {\Tcl snacc} Tcl command), but against errors from the `inside' as well such as illegal numeric values for enumeration types or illegal choice settings as well.

There are two types of errors:
\begin{enumerate}
  \item programmer errors, where the program has no other choice as to print a regret to the user and exit
  \item user errors, such as trying to write to a read-only file, where the program should tell the user about their mistake and let them try something else.
\end{enumerate}
The Tcl interface code helps the programmer for the second type of error by setting Tcl's {\Tcl errorCode} variable.
The program can {\Tcl catch} any error, and, based on the {\Tcl errorCode}, choose to deal with the mistake or rethrow the error that it is not prepared to handle.

\subsection{File commands}

Most snacc Tcl commands operate on so-called files.
A file is an internal data structure that
\begin{itemize}
  \item references the C++ representation of an ASN.1 data structure as a pointer to {\C AsnType}
  \item may be associated with an external file in the file system
\end{itemize}

The commands operating on these files are as follows:

\begin{description}%{
  \item[{\Tcl snacc create \emph{type}}]
    The command creates a file consisting only of an instance of type \emph{type}.
    \emph{type} has to be denoted as one argument, a Tcl list with two elements, module and type.
    No external filename is associated with this file.
%    The command returns a file handle that consists of letters and digits only and may therefore be used to construct a component in the Tk widget tree.

  \item[{\Tcl snacc open \emph{type filename} ?\emph{flags}? ?\emph{mode}?}]
    Open a file and read and decode its contents.
    \emph{type} has to be denoted as one argument, a list with two elements, module and type.
    The optional \emph{flags} may consist of:
    \begin{description}
      \item[{\Tcl create}] If the file does not exist, create it.
	If this flag is not given and the file does not already exist, an error occurs.
      \item[{\Tcl truncate}] If the file exists, drop its contents.
      \item[{\Tcl \emph{access}}] which may be either {\Tcl ro} or {\Tcl rw}, denoting read only and read/write access.
	If no access mode is specified, the file will be opened read/write if it is writable, and read only otherwise.
    \end{description}
    If the file is created, its mode is set to \emph{mode}, minus umask, of course.
    \emph{mode} may be any value accepted by {\C Tcl\_GetInt(3)} (the function accepts octal values).
    At last, if the file could be opened, its contents is read and BER decoded.
    As for {\Tcl snacc create} above, a file handle is returned.

    If the file cannot be opened, an error is returned identical to Tcl's {\Tcl open} command.

    More errors can be returned, as described under {\Tcl snacc read} below.

  \item[{\Tcl snacc close \emph{file}}]
    closes the file \emph{file} and invalidates the file handle.

  \item[{\Tcl snacc read \emph{file} ?\emph{type filename}?}]
    without the \emph{filename}, rereads the file from its old place; otherwise opens \emph{filename}, reads its contents into \emph{file} and closes it.
    The file's contents gets BER decoded.

    In case no \emph{filename} has been given but the \emph{file} is not associated with a filename, an error is returned and {\Tcl errorCode} is set to {\Tcl SNACC MUSTOPEN}.

    If Snacc's decoding routines detect an error, a Tcl error is returned and {\Tcl errorCode} is set to {\Tcl SNACC DECODE \emph{errval}} where \emph{errval} is the value returned by {\C setjmp()} (see sections~\ref{error-C++-section} and~\ref{lib-err-C-section} on pages~\pageref{error-C++-section} and~\pageref{lib-err-C-section}, respectively).

    If the input file is too short, the buffer will signal a read error and a Tcl error will be returned, with {\Tcl errorCode} set to {\Tcl SNACC DECODE EOBUF}.

  \item[{\Tcl snacc write \emph{file} ?\emph{filename}?}]
    BER encodes the file, then writes the file to its old place in case no \emph{filename} has been given, or opens \emph{filename}, writes \emph{file} into it and closes it.

    In case no \emph{filename} has been given but the \emph{file} is not associated with a filename, an error is returned and {\Tcl errorCode} is set to {\Tcl SNACC MUSTOPEN}.
    If you try to write to a read-only file, an error is returned and {\Tcl errorCode} is set to {\Tcl SNACC WRITE READONLY}.

  \item[{\Tcl snacc finfo \emph{file}}]
    returns a list with two elements, the file name associated with it (the empty string if no external file name is associated with it) and an identifier which may be
    \begin{description}%{
      \item[{\Tcl bad}] the file is not associated with an external file.
      \item[{\Tcl rw}] the external file has been opened read/write.
      \item[{\Tcl ro}] the external file has been opened read only.
    \end{description}%}
\end{description}%}

Since Tcl cannot operate on binary strings (that is, strings containing NUL bytes), but ASN.1 octet strings may contain arbitrary binary data, the binary data has to be converted into a replacement notation that Tcl can work with and that can be converted back to binary without loss of information.
The conversion I chose is fairly simple: NUL is converted into a backslash followed by a zero digit, and every backslash is doubled.

These conversions for the most part take place automatically.
In fact, there is only one point where the binary representation is necessary, when you want to read or write data from or into a file on disk.
Two functions have been written to offer this: the export function converts and writes an octet string to an external file, and the import function reads binary data from a file and converts it to the Tcl compatible representation.
Unlike the functions described above, these two do not operate on ASN.1 files, that is, the contents is not BER decoded/encoded, but may be used for any file in the file system.

\begin{description}%{
  \item[{\Tcl snacc import \emph{filename}}] opens the file named, reads its contents, closes it, performs the above described conversion and returns the resulting Tcl string.
  \item[{\Tcl snacc export \emph{string filename}}] converts the Tcl string into its binary counterpart, opens the file named, writes the binary buffer into it and closes it.
    The file is created and truncated as necessary.
    The command returns the empty string.
\end{description}%}

\subsection{Generic Information Retrieval}

The following functions return information about the modules and their types.
(This information is independent of any file instance, it is the information from the type descriptions in the {\ufn .asn1} files.)

\begin{description}%{
  \item[{\Tcl snacc modules}]
    returns a list of module identifiers.
  \item[{\Tcl snacc types ?\emph{module}?}]
    if a \emph{module} is specified, returns a list of all type names of that module.
    otherwise, a list of all types is returned as a list of pairs, where each pair consists of the module name and the type name.
  \item[{\Tcl snacc type \emph{type}}]
    where \emph{type} is a list with two elements, module and type.
    This command returns a list with the following four elements:
    \begin{enumerate}%{
    \setcounter{enumi}{-1}
      \item the content type as a list consisting of module name and type name
      \item an identifier that is either {\Tcl pdu} or {\Tcl sub} depending on the list of PDUs that had been given after snacc's {\ufn -tcl} option.
      \item the ASN.1 type (e.g. INTEGER or CHOICE)
      \item a list of items that depends on the ASN.1 type:
      \begin{description}%{
	\item[INTEGER] a (possibly empty) list of pairs of name and value for each named value.
	\item[ENUMERATED] a (non-empty) list of names.
	\item[SET, SEQUENCE \textnormal{and} CHOICE] a list of lists of four elements similar to that being described here.
	  Element~0 is the subtypes name, then follow content type (a pair consisting of module name and type name), \emph{pdu} vs. \emph{sub} and finally the ASN.1 type.
	  (The fourth element of the outer list is omitted for obvious reasons: it would explode the type's description.)
      \end{description}%}
    \end{enumerate}%}
\end{description}%}

\subsection{Operations on Content and Structure}

Finally, the last last four functions operate on the file instances itself.
All four commands get a \emph{path} argument that is constructed as follows:
\begin{itemize}
  \item Every \emph{path} starts with a file handle as returned by {\Tcl snacc create} or {\Tcl snacc open}.
  \item All subsequent path elements, except for the last, must indicate elements of composed types.
    For CHOICE, SET and SEQUENCE, these are member names, for SET OF and SEQUENCE OF, these are numeric indices.
  \item The last path element may reference a simple type.
  \item For SET OF and SEQUENCE OF, instead of a numeric index, a pair consisting of the word {\Tcl insert} followed by a numeric index may be specified.
    In this case, a new list element is inserted before that addressed by the index.
    The index must be in the range $0\ldots{}n-1$ to address existing elements and it must be in the range $0\ldots{}n$ for insertion, where in both cases $n$ is the number of elements in the list.
  \item For {\Tcl snacc unset}, the path must point to an optional member of a SET or SEQUENCE or to an element of a SET OF or SEQUENCE OF.
\end{itemize}

The commands are:
\begin{description}%{
  \item[{\Tcl snacc info \emph{path}}] returns information about the value pointed to by \emph{path}.
    The information returned is quite similar to that of {\Tcl snacc type} above, with the following exceptions:
    \begin{itemize}%{
      \item element~0, the content type, contains empty names for types that have not been given a name (e.g. a SET member of type OCTET STRING
	Example: the {\ASN contents} member in type {\ASN File} in file {\ufn edex1.asn1} (page~\pageref{edex1.asn1}) {\Tcl snacc info} returns {\Tcl \{\{\} \{\}\} sub \{OCTET STRING\}}).
      \item the number of elements depends on the ASN.1 type:
	\begin{description}%{
	  \item simple types (\textbf{NULL}, \textbf{BOOLEAN}, \textbf{INTEGER}, \textbf{ENUMERATED}, \textbf{REAL}, \textbf{BIT STRING} and \textbf{OCTET STRING}):
	    no additional elements are returned.
	    For the list of named values for INTEGER, ENUMERATED and BIT STRING, you have to call {\Tcl snacc type [lindex [snacc info \emph{path}]~0]}, unless the content type equals {\Tcl \{\{\}~\{\}\}}.
	  \item[CHOICE]
	    a total of five elements is returned, number~3 is the name of the choice member currently chosen, and the final element number~4 is an identifier that is either {\Tcl void} or {\Tcl valid} depending on whether the pointer representing the choice member is {\C NULL} or pointing to some data.
	  \item[SET \textnormal{and} SEQUENCE]
	    a fourth element, a list of pairs, is returned, where the pairs are built from the member name and an identifier that is either {\Tcl valid} or {\Tcl void}
	  \item[SET OF \textnormal{and} SEQUENCE OF]
	    the number of items is returned as element number~3.
	\end{description}%}
    \end{itemize}%}
  \item[{\Tcl snacc get \emph{path}}]
    returns the value of the subtree pointed to by \emph{path}.
    The value returned is a simple string for simple types, and a hierarchical structure (in Tcl that is a list of lists) otherwise.
    \begin{description}%{
      \item[NULL]
	the empty string is returned.
      \item[BOOLEAN]
	the value is returned as {\Tcl TRUE} or {\Tcl FALSE}.
      \item[INTEGER]
	the numeric value is returned, even if it has been assigned a name.
      \item[ENUMERATED]
	the symbolic value is returned.
	The numeric values are inaccessible through the Tcl interface.
	If the object happens to contain an illegal numeric value, an error is returned and {\Tcl errorCode} is set to {\Tcl SNACC ILLENUM}.
      \item[REAL]
	the value is returned as formatted by {\C sprintf (\dots, "\%g", \dots)}, except for the special values {\ASN PLUS-INFINITY} and {\ASN MINUS-INFINITY} which are returned as {\Tcl +inf} and {\Tcl -inf}, respectively.
      \item[BIT STRING]
	a string, consisting solely of `0' and `1', is returned.
      \item[OCTET STRING]
	the binary string is returned as is, except for the unavoidable NUL-escape described above.
      \item[OBJECT IDENTIFIER]
	the value is returned as a list of numbers.
      \item[CHOICE]
	is returned as a pair, the choice member chosen and its value.
      \item[SET \textnormal{and} SEQUENCE]
	are returned as a list of pairs of member name and value.
	Absent OPTIONAL members are left out from the list.
      \item[SET OF \textnormal{and} SEQUENCE OF]
	are returned as a list of values.
    \end{description}%}
  \item[{\Tcl snacc set \emph{path value}}] sets the subtree identified by \emph{path} to \emph{value}.
    The value must be of the form 
    \begin{description}%{
      \item[NULL]
	the only legal value is the empty string.
	otherwise, an error is returned and {\Tcl errorCode} is set to {\Tcl SNACC ILLNULL}.
      \item[BOOLEAN]
	any value that is accepted by {\C Tcl\_GetBoolean}(3) is fine.
      \item[INTEGER]
	both the numeric (as accepted by {\C Tcl\_GetInt(3)}) and the symbolic values are allowed.
      \item[ENUMERATED]
	any value must be specified by its name.
	If an illegal name is given, an error is returned and {\Tcl errorCode} is set to {\Tcl SNACC ILLENUM}.
      \item[REAL]
	the special values {\ASN PLUS-INFINITY} and {\ASN MINUS-INFINITY} have to be given as {\Tcl +inf} and {\Tcl -inf}, respectively.
	All other values may be specified in any format accepted by {\C Tcl\_GetDouble}(3).
      \item[BIT STRING]
	a string that must consist of `0' and `1' only has to be given.
	otherwise, an error is returned and {\Tcl errorCode} is set to {\Tcl SNACC ILLBIT}.
      \item[OCTET STRING]
	due to the NUL-escapes necessary, any string where a backslash is followed by either another backslash or a `0' digit is legal.
	Improper use of the escape character leads to an error and {\Tcl errorCode} will be set to {\Tcl SNACC ILLESC}.
      \item[OBJECT IDENTIFIER]
	the value has to be specified as a list of numbers.
	if the arc has less than 2 or more than 10 elements, an error is returned and {\Tcl errorCode} is set to {\Tcl SNACC ILLARC <2} or {\Tcl SNACC ILLARC >10}, respectively.
      \item[CHOICE]
	the value expected is a pair, the choice member chosen and its value.
	if an illegal member is specified, an error is returned and {\Tcl errorCode} is set to {\Tcl SNACC ILLCHOICE}.
      \item[SET \textnormal{and} SEQUENCE]
	the value has got to be a list of pairs of member name and value.
	Any member may be specified at most once.
	All mandatory members must be present.
	Failure to do so will result in an error and {\Tcl errorCode} to be set to {\Tcl SNACC DUPMEMB} or {\Tcl SNACC MISSMAND}, respectively.
	All optional members not listed in the value will be deallocated.
      \item[SET OF \textnormal{and} SEQUENCE OF]
	the whole list is replaced with the specified value that has to be a proper Tcl list.
    \end{description}%}
  \item[{\Tcl snacc unset \emph{path}}]
    unsets the subtree pointed to by \emph{path}.
    Only OPTIONAL members of SET and SEQUENCE types and list elements of SEQ OF and SEQUENCE OF may be unset.
    If you try to unset a mandatory SET or SEQUENCE member, an error is returned and {\Tcl errorCode} is set to {\Tcl SNACC MANDMEMB}.
\end{description}%}

I did not follow Tk's example where one has to set widget commands to {\Tcl \{\}} to delete them.
This method would have the drawback that one could not distinguish between an empty and a non-existing octet string (in C that would be {\C ""} vs. {\C NULL}).

The value returned by {\Tcl snacc get} may be very long, {\Tcl snacc get file0} returns the contents of the whole file!

\section{\label{snacc-examples}Examples}

The following example session shall illustrate the {\Tcl snacc} commands usage.
It assumes that the editor example files {\ufn edex0.asn1} and {\ufn edex1.asn1} (see appendix~\ref{edex-files} on page~\pageref{edex-files}) have been compiled into a binary that has been linked with the necessary libraries.

The notation used is as in the Tcl book \cite{tcl-book}, i.e. `$\Rightarrow$' indicates a normal return value and `$\varnothing$' indicates an error with the error message set in {\Tcl \emph{oblique typeface}}.

A look at the types available:

\begin{Tclex}
   & snacc types\\
\R & \{EdEx-Simple Hand\} \{EdEx-Structured StructuredChoice\} \{EdEx-Structured Coordinate\} \{EdEx-Structured CoordinateSeq\} \{EdEx-Structured RGBColor\} \{EdEx-Structured Simple\} \{EdEx-Simple File\} \{EdEx-Simple RainbowColor\} \{EdEx-Structured DirectorySetOf\} \{EdEx-Structured Various\} \{EdEx-Structured File1\} \{EdEx-Structured CoordinateSeq1\} \{EdEx-Structured Directory\} \{EdEx-Structured Structured\} \{EdEx-Simple DayOfTheWeek\}
\end{Tclex}

Create a file (without filename):

\begin{Tclex}
   & set file [snacc create \{EdEx-Structured Structured\}]\\
\R & file0
\end{Tclex}

The string returned is the file handle. It is used as the first snaccpath component in successive calls.

Look at the file's type:

\begin{Tclex}
   & snacc info \$file\\
\R & \{EdEx-Structured Structured\} sub SET \{\{coord valid\} \{color valid\}\}
\end{Tclex}

The file's type is a SET with the name `Structured' in module `EdEx-Structured' (it is defined in file {\ufn edex1.asn1} (see page~\pageref{edex1.asn1})).
The `sub' tells us that the type has not been marked as a PDU.
The SET has the components `coord' and `color', both are present (they are not OPTIONAL, i.e. mandatory).

Look at a component's type:

\begin{Tclex}
   & snacc info "\$file color"\\
\R & \{EdEx-Structured StructuredChoice\} sub CHOICE rainbow valid
\end{Tclex}

Snacc has generated the type name `StructuredChoice' for this type, this name was not defined in the {\ufn .asn1} file.
The CHOICE object currently is set to `rainbow'.
A CHOICE component is always present (CHOICE components may not be OPTIONAL), the `valid' is just for completeness.

Ask for the CHOICE's generic type information:

\begin{Tclex}
   & snacc type \{EdEx-Structured StructuredChoice\}\\
\R & \{EdEx-Structured StructuredChoice\} sub CHOICE \{\{rainbow \{EdEx-Simple RainbowColor\} sub INTEGER\} \{rgb \{EdEx-Structured RGBColor\} sub SEQUENCE\}\}
\end{Tclex}

The CHOICE type has two possible components, `rainbow', an INTEGER and `rgb', a SEQUENCE.

Look at the INTEGER's type information:

\begin{Tclex}
   & snacc type \{EdEx-Simple RainbowColor\}\\
\R & \{EdEx-Simple RainbowColor\} sub INTEGER \{\{red~0\} \{orange~1\} \{yellow~2\} \{green~3\} \{blue~4\} \{indigo~5\} \{violet~6\}\}
\end{Tclex}

The type has got named values.

Access the file contents:

\begin{Tclex}
   & snacc get \$file\\
\R & \{coord \{cartesian \{\{x 0\} \{y 0\}\}\}\} \{color \{rainbow 977768\}\}
\end{Tclex}

The color component contains garbage.
Change that:

\begin{Tclex}
   & snacc set "\$file color rainbow" green\\
\R &\\
   & snacc get "\$file color"\\
\R & rainbow 3
\end{Tclex}

Change it again, select the CHOICE's other component type, `rgb', and set its `red' component:

\begin{Tclex}
   & snacc set "\$file color rgb red" 256\\
\R &
\end{Tclex}

Changing a CHOICE component selection work only for write access, on read access this is not possible:

\begin{Tclex}
   & snacc get "\$file color rainbow"\\
\E & \emph{snacc get: illegal component "rainbow" in path}\\
   & snacc get "\$file color rgb"\\
\R & \{red 256\} \{green 544501616\} \{blue 1814045815\}
\end{Tclex}

Upon setting a SET or SEQUENCE type, all mandatory members have to be specified:

\begin{Tclex}
   & snacc set "\$file color rgb" \{\{green 0\} \{blue 0\}\}\\
\E & \emph{mandatory member "red" is missing in list}\\
   & snacc set "\$file color rgb" \{\{red 0\} \{green 256\} \{blue 0\}\}\\
\R &\\
   & snacc get "\$file color"\\
\R & rgb \{\{red 0\} \{green 256\} \{blue 0\}\}
\end{Tclex}

Finish up:

\begin{Tclex}
   & snacc close \$file\\
\R &\\
   & snacc get \$file\\
\E & \emph{snacc get: no file named "file0"}
\end{Tclex}

\section{\label{tcl-if-impl}Implementation}

The Tcl interface is implemented in {\ufn \dots/c++-lib/inc/tcl-if.h} and {\ufn \dots/c++-lib/src/tcl-if.C}.
It gets initialized with the help of {\ufn \dots/c++-lib/inc/init.h} and {\ufn \dots/c++-lib/src/tkAppInit.c}.

The {\Tcl snacc} commands implementation is pretty straight forward: check the arguments, call a metacode function to perform an action and return the result, which may indicate success or an error.

Care has been taken to check the return codes of all system calls and to set Tcl's {\Tcl errorCode} variable in case any system call returns an error.

The file {\ufn tkAppInit.c} contains the function that introduces the {\Tcl snacc} Tcl command to the Tcl interpreter.
The path that leads to the function's invocation is a little tricky and is described in section~\ref{editor-building}, ``Building Your Own Editor''.

\section{Setup for the Tcl Code Generator}

To compile Snacc with the Tcl interface code generator, you have got to fulfill the following conditions:
\begin{itemize}
  \item the configure script must be able to find {\ufn tclsh} and the Tcl/Tk libraries
  \item the preprocessor switches {\C NO\_META} and {\C NO\_TCL} in {\ufn \dots/policy.h} must not be set
\end{itemize}

\section{\label{tcl-if-deficiencies}Deficiencies}

\begin{itemize}
  \item
    Values defined in the ASN.1 files currently are inaccessible.
    Adding access functions to the metacode and Tcl interface is rather trivial: build an array of elements that hold a variable's name as a character string and an {\C AsnType~*} that points to the C++ variable.
    {\C a[i].val-->\_getdesc()} would return a pointer to the variable's type description.

    (First you should fix snacc's value parser as currently it lets some values silently vanish, for example the {\ASN victory} in {\ufn edex0.asn1} that you can find in appendix~\ref{edex0.asn1} on page~\pageref{edex0.asn1}.)
  \item
    The Tcl interface does not provide symbolic object identifiers.
    Mapping numeric to symbolic oids is a task that is difficult to get right since snacc translates\\
    {\ASN anOidVal OBJECT IDENTIFIER ::= \{ 1 2 foo(3) \}}\\
    and\\
    {\ASN anOidVal OBJECT IDENTIFIER ::= \{ 1 2 3 \}\\
    foo INTEGER ::= 3}\\
    into identical C++ code, but translating the second {\ASN anOidVal} into {\ASN \{ 1 2 foo \}} may in fact violate {\ASN foo}'s semantics.
\end{itemize}

% file: .../doc/editor.tex

% $Header: /usr/app/odstb/CVS/snacc/doc/editor.tex,v 1.1 1997/01/01 22:47:41 rj Exp $
% $Log: editor.tex,v $
% Revision 1.1  1997/01/01 22:47:41  rj
% first check-in
%

\chapter{\label{snacced-chapter}SnaccEd, the Snacc Editor}

SnaccEd is a simple graphical editor for BER encoded files.
A set of ASN.1 files describes one or more hierarchical datastructures that can be displayed as an n-ary non-circular graph, in other words: a tree.

\begin{figure}[ht!]
\begin{center}
\includegraphics[scale=0.7]{example}
\caption[SnaccEd: an example screen shot]{An example screen shot}
\label{snacced-example}
\end{center}
\end{figure}

SnaccEd is combined from
\begin{itemize}
  \item the usual Snacc BER encode and decode functions
  \item the metacode (described in chapter~\ref{meta-chapter} starting on page~\pageref{meta-chapter})
  \item the Tcl library
  \item the Snacc Tcl interface (described in chapter~\ref{tcl-if-chapter} starting on page~\pageref{tcl-if-chapter})
  \item the Tk widget set
  \item a freeware tree widget (another Tcl extension, implemented in C++)
  \item a Tcl script that glues all those parts together
\end{itemize}

All items except for the Tcl script are compiled into an executable, the \emph{snaccwish}.
The Tk interpreter has the name {\ufn wish}, for ``windowing shell'', and consequently, I name the program that results from linking Snacc with Tk {\ufn snaccwish}.
For every individual set of ASN.1 files, a different snaccwish has to be made, because every snaccwish contains the specialized encode and decode routines for the ASN.1 files' types.

This {\ufn snaccwish} is a Tcl interpreter that has the additional commands of the Tk widget set, the tree widget and the Snacc interface built in.
This interpreter reads the {\ufn snacced} script that implements the graphical Snacc editor.
I will henceforth refer to the script as {\ufn snacced} and to the interpreter as {\ufn snaccwish}.
You may name your script and the shell binary differently, just make sure that the script calls the correct binary!.

Snacced can be called with various arguments, see the manual page for details.

The {\ufn snacced} script is only the most visible entry point, other scripts will be read using Tcl's autoloading mechanism.

The Tcl script is (or can be) always the same.
It uses the {\Tcl snacc} command to learn about the ASN.1 modules, types and PDUs.

Since the BER format has not got any magic number or similar concept, the Snacc routines in general cannot identify the ASN.1 type contained in a BER encoded string of octets.
As a consequence, one has to choose not only the file name but the ASN.1 type as well when one opens or creates a file (see figure~\ref{selbox-pic} \sthisnextafter{selbox-pic} for an example).

\begin{figure}[ht]
\begin{center}
\includegraphics[scale=0.7]{selbox}
\caption{The file and content type selection box}
\label{selbox-pic}
\end{center}
\end{figure}

One can then examine and manipulate the file's structure and contents.

\section{Manipulating the Display}

This section describes the pointer\footnote{
  My pointer device is a mouse, but yours may be a trackball, a tablet, a joystick or something else.
} operations that change the amount of information to be shown.
(To change the file's contents, the node's content window has got be opened.)

The file is displayed by means of a tree widget.
Only a part of the full hierarchy is shown.
The subtree's root is at the left side.
The function of the pointer buttons\footnote{
  I will refer to the buttons but their number, not their position.
  I could refer to button~1 as the right button, but this might confuse you as your button~1 may in fact be on the left hand side.
} when clicking on \emph{node names} is as follows:
\begin{description}
  \item[button~1]
    adds or deletes the node's subnodes to or from the display, respectively.
    (Except for SET OF and SEQUENCE OF types, where with button 2 you have got to open the node content editor, a list widget, and have to toggle the display of individual elements by clicking on their index numbers. This is explained at the end of section~\ref{cont-ed-list-sect} \sthisnextafter{cont-ed-list-sect}.)
    \begin{itemize}
      \item For nodes that have subnodes being shown, the subtree gets hidden.
      \item Otherwise, the node's immediate descendents are added to the display.
    \end{itemize}
  \item[button~2]
    opens or closes the node, where ``closed'' means that only the nodes name is being shown, and ``open'' means that an additional window showing the node's contents it put under the node's name.
    This content window is explained in the next section.
  \item[button~3]
    adds or deletes the node's parent to or from the display, respectively.
    \begin{itemize}
      \item For nodes where the parent is displayed, all parents and all siblings with their subtrees will get hidden.
      \item Otherwise, the parent is added to the display.
    \end{itemize}
\end{description}

Pressing and holding button 2 on a free space, the display can be dragged by moving the pointer.

\section{The Content Window}

\begin{figure}
\begin{center}
\includegraphics[scale=0.65]{simple}
\caption{Content editors for ASN.1 simple types}
\label{simple-content}
\end{center}
\end{figure}

\begin{figure}
\begin{center}
\includegraphics[scale=0.7]{struct}
\caption{Content editors for ASN.1 structured types}
\label{struct-content}
\end{center}
\end{figure}

\def\exref{ (based on the example displayed as figure~\ref{snacced-example} on page~\pageref{snacced-example})}

\begin{figure}
\begin{center}
\includegraphics[scale=0.7]{str-popup}
\caption[Popup for import/export of OCTET STRING contents]{Popup for import/export of OCTET STRING contents\exref}
\label{str-popup}
\end{center}
\end{figure}

\begin{figure}
\begin{center}
\includegraphics[scale=0.7]{list-popup}
\caption[Popup for action selection for SET OF and SEQUENCE OF types]{Popup for action selection for SET OF and SEQUENCE OF types\exref}
\label{list-popup}
\end{center}
\end{figure}

The content window that may be opened beneath the node's name looks and behaves different for every content type.
An example for every ASN.1 simple type is shown in figure~\ref{simple-content} \sthisnextafter{simple-content}.
The ASN.1 input for the example can be found in appendix~\ref{edex0.asn1} on page~\pageref{edex0.asn1}.
\begin{itemize}
  \item
    The NULL type has only one value that cannot be changed.
  \item
    Values of BOOLEAN type are displayed as a toggle button.
  \item
    For the ENUMERATED type, SnaccEd displays a list of radio buttons listing the values' names.
    (The numerical values are not shown.)
  \item
    INTEGER values are displayed using an entry widget where the numeric value can be seen and changed.
    The entry widget's binding have been changed to allow the input of ``--'' and decimal digits only in addition to the usual control functions (procedure {\Tcl int\_entry\_bindings}).
    Similar to the ENUMERATED type, values can be given a name; the list is displayed as above.
  \item
    Individial names in a BIT STRING may be named.
    SnaccEd displays a list of buttons identifying those bits along with their name.
    Clicking on one of those buttons toggles the bit's value.
    \newline
    The bit string is displayed and can be edited in its binary representation in an entry widget below the names.
    The entry widget's binding have been changed to allow the input of ``0'' and ``1'' only in addition to the usual control functions (procedure {\Tcl bit\_string\_entry\_bindings}).
  \item
    OCTET STRINGs and derived types are displayed in a text widget.
    Since Tcl cannot handle strings containing NUL bytes, NUL bytes are displayed as the two character combination ``\char`\\0'' and backslashes are duplicated, ``\char`\\\char`\\''.
    Button 3 pops up a small menu that allows you to load or save the octet string from or to an external file, respectively (figure~\ref{str-popup} \sthisnextafter{str-popup}).
    The X text selection to copy text between the text widget and e.g. an xterm can be used as well.
  \item
    CHOICE types allow exactly one of their subtypes to be valid and therefore are displayed as a list of radio buttons.
    Clicking on a button deletes the old choice and allocates the new one.
    See the ``{\C color}'' in figure~\ref{struct-content} on page~\ref{struct-content}.
  \item
    The SET and SEQUENCE types' elements are displayed in a list of buttons, and optional elements may be added and deleted by clicking on their buttons.
    Mandatory members do not respond to button clicks and are greyed out.
    Examples: the list element \#1 right in the middle of figure~\ref{snacced-example} on page~\pageref{snacced-example} or the ``{\C rgb}'' in figure~\ref{struct-content} on page~\pageref{struct-content}.
  \item \label{cont-ed-list-sect}
    SnaccEd visualizes the types SET OF and SEQUENCE OF in a list widget.
    The widget shows the elements' ordinal numbers, the elements themselves are shown in individual widgets to the right.
    Button~3 in the list widget brings up a small menu where you can choose the action to perform when (with button 1) you click on a list element (figure~\ref{list-popup} \sthisnextafter{list-popup}):
    \begin{itemize}
      \item toggle the display of an element
      \item insert a new element
      \item append a new element
      \item delete an element
    \end{itemize}
    The cursor shape changes and reminds you of your chosen action.
    An example of a list widget is ``{\C files}'' in figure~\ref{snacced-example} on page~\pageref{snacced-example}.
\end{itemize}
Some content editors can be resized: move the pointer to the content editor's frame.
Where the cursor shape changes to a ``bottom\_right\_corner'', press button~1 and drag the frame.

\section{\label{editor-building}Building Your Own Editor}

There isn't much to be done to get an editor that understands your BER encoded files.
This chapter's example can be found in {\ufn \dots/tcl-example/}.

\begin{itemize}
  \item
    Make sure your Snacc compiler has be configured to support the Tcl code.
    This can be verified by calling {\ufn `snacc -h | grep tcl`}; if the output is empty, the configuration script was unable to find Tcl/Tk.
  \item
    The Tcl/Tk libraries must have been compiled with with gcc in order to use its {\C main()} function.
    Otherwise, the constructors and destructors of static variables may not be called.
    I have added some code at the end of {\ufn \dots/.../c++-lib/src/tcl-if.C} that checks for this condition.
    \newline
    Tcl's default is not to use gcc but cc where present, but you can compile Tcl/Tk by calling {\ufn env CC='gcc -traditional' ./configure} instead of {\ufn ./configure}.
  \item
    Install the tree widget.
    SnaccEd works with tree-3.6.
    (tree-3.6.2 requires another Tcl extension, itcl, but to avoid complicating matters any further, SnaccEd sticks to the simpler version.)
    When {\ufn \dots/configure} found the tree widget libraries, the macro definition for {\Make TREELIBS} in {\ufn \dots/makehead} will have been set to {\ufn -ltktree -lOS}.
  \item
    In your makefile, extend {\ufn snacc}'s list of arguments with the {\ufn -tcl} option followed by the list of PDUs (that is an additional \emph{two} arguments).
  \item
    Put {\ufn modules.C} into the list of files to be compiled.
  \item
    Compile with {\ufn -DTCL}.
  \item
    To link, replace {\ufn -lasn1c++} against {\ufn -lasn1tcl} and add {\ufn \$(TREELIBS)} to the list of libraries.
    (You may have to add a {\ufn -L} option as well.)
  \item
    Call {\ufn \$(TCL\_LIB)/make-snacced \$(SNACCED) \$(WISH) \$(TCL\_LIB)}, where {\ufn \$(TCL\_LIB)} is either {\ufn \$\{prefix\}/lib/snacc/tcl/} (the place where the Tcl library files got installed by {\ufn make install}) or {\ufn \dots/tcl-lib/} (in case you haven't called {\ufn make install} yet), and {\ufn \$(WISH)} is the name of the wish executable you just compiled.
    The {\ufn make-snacced} script generates a small Tcl script, {\ufn \$(SNACCED)}, that executes your wish executable, extends the Tcl variable {\Tcl auto\_path} to include your {\ufn \$(TCL\_LIB)} und through the auto loading mechanism calls the Tcl library routine {\Tcl snacced} to start the Snacc editor.
\end{itemize}

Make sure you get the {\ufn \dots/tcl-example/} working before you despair of your own set of files.
Take {\ufn \dots/tcl-example/makefile} as a guide.

Your can use the {\ufn .h} and {\ufn .C} file resulting from calling {\ufn snacc -tcl \dots} for both the editor and your other uses.
Just remember: if you compile with {\ufn -DTCL}, link against {\ufn libasn1tcl.a}; if you want to disable the metacode and tcl interface, compile with {\ufn -DMETA=0} and link against {\ufn libasn1c++.a}.

Make sure that you don't link with your old {\C main()}\footnote{Or make sure your {\C main()} behaves similar to the Tk libraries'}.
The generated file {\ufn modules.C} contains the line\\
{\C static int (*dummy)(Tcl\_Interp~*) = Tcl\_AppInit;}\\
that forces {\ufn libasn1tcl.a(tkAppInit.o)} to be linked.
{\ufn \dots/c++-lib/src/tkAppInit.c} in turn contains the lines\\
{\C extern int main();\\
int *tclDummyMainPtr = (int *)main;}\\
that force the {\C main} function in the Tk library to be linked.

{\C Tcl\_AppInit()} calls {\C Snacc\_Init()} that is defined in {\ufn \dots/c++-lib/src/tcl-if.C}.
{\C Snacc\_Init()} installs the {\Tcl snacc} command.

\section{\label{snacced-impl}Implementation}

The Tcl scripts that implement the editor can be found in {\ufn \dots/tcl-lib/} and, after installation, in {\ufn \$\{prefix\}/lib/snacc/tcl/}.

You are free to change the Tcl script(s), for example to display some data types in a more appropiate manner.
Octet strings may be user readable but often are not, pictures and audio data come to mind.

If add procedures or {\ufn .tcl} files, you have got to rebuild Tcl's autoloading index.
This best done by adding the files to the {\Make TCLFILES.dist} list in {\ufn \dots/tcl-lib/makefile} and running {\ufn make} again.

%!!! nicht in doc/:
The first SnaccEd was able to handle only one file at a time.
To enable the editor to handle several files simultaneously required the following steps:
\begin{itemize}
  \item instead of using the default toplevel widget {\Tcl .} (dot), open a toplevel widget for every file.
    The toplevel widgets get names {\Tcl .file0}, {\Tcl .file1}, \dots
    The same name without the leading dot is used as a global array variable to hold miscellaneous pieces of information about the file.
    You can see how the names are generated in the code example on page~\pageref{tcl-name-space-example-code}.
    The name of this variable is given to many procedures in the {\Tcl fileref} parameter.
  \item identify global variables. Those were
    \begin{itemize}
      \item the name of the file handle
      \item the names of some widgets, namely the toplevel, the menubar, the canvas and the tree widget.
    \end{itemize}
    Other global variables can be left untouched: the help text, the list of PDU types.
    This information is the same for all the files a snaccwish can handle.
\end{itemize}

Since the file and type selection box, the help text and the dialog boxes are modal, only one instance is needed and they can have the same names for every file opened by the editor.

The editor displays only a portion of the ASN.1 file.
The Snacc editor keeps the displayed portions of the ASN.1 file in two similar data structures.

The contents of an ASN.1 file is accessed by calling the {\Tcl snacc} command with a {\Tcl \emph{path}} that identifies the requested data portion.

Every ASN.1 file is displayed using one toplevel widget.
This toplevel widget is a frame for a number of subwidgets:
\begin{itemize}
  \item a menubar
  \item a canvas
  \item two scrollbars, one vertical, one horizontal, to select the visible part of a canvas that has grown too large for the frame.
\end{itemize}

The menubar contains two buttons, one for the usual file related commands, and a help button.

The canvas is the main arena.
Its subwidgets are the tree widget and all the canvas items that make up the nodes and edges.
The tree widget computes the positions of the canvas items and moves them in place.

The contents of an ASN.1 file can be seen as a tree (the data structure may by recursive using CHOICE types or OPTIONAL components, and a PDU may contain instances of a type that contain other instances of the same type (see figure~\ref{recursion} on page~\pageref{recursion} for an example), but as ASN.1 has no pointers, cycles are impossible).
To display this tree, it is mirrored in a number of Tcl data structures:
\begin{itemize}
  \item
    The \emph{snaccpath} is the 1:1 representation of the PDU's structure.
    This is what in chapter~\ref{tcl-if-chapter} is always referred to as ``\emph{path}'' argument to most {\Tcl snacc} subcommands.
    The snaccpath is a proper Tcl list.
  \item
    The \emph{treepath} is very similar to the snaccpath.
    Its structure is the same as the snaccpath's, but its syntax and a few elements are different:
    \begin{itemize}
      \item
	The components in a snaccpath are separated by ``~'' (space), in a treepath they are separated by ``/'' (slash).
	This difference is not strictly necessary, but it helps to detect errors in argument passing as the {\Tcl snacc} commands will never accept any treepath for their path arguments.
      \item
	In a snaccpath, the elements of SET OF and SEQUENCE OF types are identified by their index.
	In a treepath, another numeric id is used instead.
	The reason for this becomes clear when we have a look at where the treepath is used and what would have to be done if the elements' list indices were used in the treepaths.

	The treepath is used in a number of places, for widget and variable names and for canvas item tags, all detailed in the below bulleted items.

	When an element of a SET OF or SEQUENCE OF type is deleted, the snaccpath's indices for the deleted element's successors have to be decremented to point to the same item; when an element is inserted, those indices need to be incremented.
	As a consequence, the widget and variable names and the canvas item tags of all elements that follow the one element that has been deleted or inserted would have to be adjusted and all the names and tags of their descendants.
	Even if these names and tags could easily be changed (they cannot), it would still be an enormous amount of work and the slow Tcl interpreter could need some seconds to complete this task.
	This enormous labour can be avoided by introducing a table lookup:

	Every node of a SET OF or SEQUENCE OF type gets an idlist (identifier list).
	This idlist is a Tcl list, its length is the same as there are elements in the ASN.1 data object.
	Every idlist element corresponds to an element of the data object.
	Whenever an element is deleted from the data object, the corresponding id from the idlist is removed as well; insertions are likewise performed in both the data object and the idlist.
	The idlist contains numbers, zero for data objects that are not visually displayed on the canvas and locally unique non-zero numbers otherwise.

	When a data object is identified through its treepath, the id is extracted and the id's position is sought in the idlist.
	The id's position in the idlist is the element's index for the snaccpath.
    \end{itemize}

    The treepath is used to build the names of widgets that display a PDU's structure and content portions.
  \item
    The node labels and lines for the edges are canvas items, no full fledged widgets.
    Canvas items can be given tags for identification purposes; the tags of an item are an ordered Tcl list.
    Canvas items have a locally unique id, but as different items can have the same tag, item groups can be identified.

    Since all tags form an ordered Tcl list, individual items can be addressed:\\
    {\Tcl [lindex [\$canvas gettags \$id] \$index]}\\
    SnaccEd uses this mechanism to translate button clicks into paths: when a canvas item is clicked at, the canvas makes this item ``current'' and\\
    {\Tcl [\$canvas find withtag current]}\\
    returns the item's id.
    The id is then used as described above to retrieve the tag list.

    The canvas line items that are used as edges get no tags.

    The canvas text items that make up the node labels and the canvas window items that contain the content editors get three tags.
    The three tags are ordered from most general to most specific:
    \begin{enumerate}\setcounter{enumi}{-1}%
      \item
	For node labels this tag has the form \emph{validity}-label.
	The validity is either ``valid'' or ``void''.
	Absent OPTIONAL components are ``void''.
	Active node labels get the tag ``valid-label''.
	In the procedure {\Tcl new\_file} this tag is used to bind the three pointer button events to the callback procedures {\Tcl prune\_or\_add\_children}, {\Tcl toggle\_editor} and {\Tcl set\_or\_add\_root}, respectively.

	For content editors this tag is simply ``edit'', because content editors can only be opened for valid nodes and therefore the validity would be redundant.
      \item
	This tag is the treepath.
	It is the same for all canvas items for this node: the label and possibly the content editor.
	This is the tag that is given to the tree widget.
	The tree widget handles all canvas items with the same tag as a group: it uses their bounding box to calculate the tree layout and it keeps the relative distances of the group's items so that their internal layout persists any change in the tree's layout.
      \item
	This tag is a combination of the other two tags: it is the treepath, a colon and either ``label'' or ``edit''.
	This tag is the most specific and it is used to address the individual canvas item, for example to check for a content editors existence.
	No two items have the same value of this tag.
    \end{enumerate}
  \item
    Content editors are not simple canvas items.
    They are build from one or more widgets and this widget tree is put into a canvas window item.
    The widgets have names of the form {\Tcl \$canvas.edit\$treepath}.
    The leading \$canvas is the name of the canvas widget.
    Widget names starting with that name are descendants of the canvas, here they are children.
    The trailing \$treepath does not contain any dots and therefore Tk understands edit\$treepath as a single node in the widget tree.
  \item
    Most of the content editors modify a global variable, for example the entry widgets for INTEGER types or the radiobuttons for ENUMERATED types.
    The variable's name is the simple composition {\Tcl var:\$treepath} that guarantees its uniqueness.

    SET and SEQUENCE types need a variable for each of their components: the component's name gets tacked to the end which yields {\Tcl var:\$treepath:\$name}.

    Named bits of BIT STRING types get similar variables, the bit value is put after the second colon.
    The bits' toggle buttons operate on these variables.
\end{itemize}

\begin{figure}
\begin{center}
\includegraphics[scale=.6]{recur}
\caption{`Recursive' data structures}
\label{recursion}
\end{center}
\end{figure}

When button~2 is clicked on a node label, the procedure {\Tcl toggle\_editor} gets called:
\begin{itemize}
  \item
    checks whether the editor for the current node already exists.

    If it does, the editor is deleted.

    Otherwise, the editor is opened by creating a frame widget that is filled with an appropiate set of subwidgets.
    The editor is supplied with the corresponding content portion from the ASN.1 file and a number of event bindings is installed that let the user modify the contents.
    The frame is placed at the right place below the node name and the tree widget is called to adjust the layout.
    Most of the changes to nodes containing simple ASN.1 types are detected using Tcl's trace mechanism.
    For example, the entry widget for an INTEGER modifies a global variable.
    The trace procedure that gets called upon every modification computes the snaccpath from the variable's name and modifies the ASN.1 file accordingly.
    The only ASN.1 simple type that makes an exception to the rule is OCTET STRING.
    Text widgets do not modify global variables and even if they did, copying the string back to the Snacc object for every key stroke is a waste of CPU time.
    The text widget's contents is written back to the octet string when the widget receives a leave window event or, since Tk uses explicit focus (click-to-focus instead of focus-follows-pointer) and thus the widget's contents may be changed even after the pointer has left the window, it receives a focus out event.

    The structured ASN.1 types are modified explicitly, i.e. through button clicks.
    The components are modified using their natural GUI counterparts.
    \begin{itemize}
      \item OPTIONAL members of SET and SEQUENCE types are allocated and deallcated by clicking on a checkbutton.
      \item CHOICE members are selected by clicking on a radiobutton.
      \item SET OF and SEQUENCE OF types are implemented as list and SnaccEd displays them using a list widget.
	(This is not actually true.
	Since the listbox widget allows only single selection or selection of multiple but contiguous entries, I replaced it with a text widget and with some event bindings and a tag for the selected items made it behave like a listbox widget with multiple selection.
	A patch to the Tk~3.6 code that implements non-contiguous selections exists, but I didn't want to enforce the trouble of getting, patching and installing the Tk library again on anyone.)
    \end{itemize}
\end{itemize}

set\_or\_add\_root:

When button~3 has been clicked on a node label, the callback procedure {\Tcl set\_or\_add\_root} will be called.

\begin{itemize}
  \item
    The procedure checks whether the node that has been clicked at is the file's toplevel node.
    In this case nothing can be done because there is no parent that could be shown or hidden.
  \item
    Otherwise, the routine checks whether the clicked node is currently displayed as the subtree's root.
    In this case both the node's parent and grand parent are calculated.
    The grand parent is not displayed, it merely corresponds to the tree widgets invisible root.
    The parent is displayed by adding the node to the invisible grand parent.
    The old root node and its siblings are aded by calling the {\Tcl ed\_expand} procedure.
    {\Tcl ed\_expand} in turn calls {\Tcl ed\_addnode} for all the parent's children.
    {\Tcl ed\_addnode} checks whether the node is already present on the canvas.
    For the old root node this check yields true and the routine simply moves the node and all its descendants to below the new root node.
    All other nodes are created as usually done when clicking on button~1.
  \item
    Otherwise, the node that has been clicked is to be shown as root.
    This is simple: just tell the tree widget to display the node as root.
    The tree widget removes everything else that does not belong to the subtree and calls the remove callback procedures for all these items.
    The remove callback procedure is installed when a content editor is opened and is used to destroy the content editor frame widget and its subwidgets.
\end{itemize}

selbox:

The file-and-content-type-selection-box (short: selbox) serves different purposes.
The selbox contains three parts: a file name selection, a content type selection and a button row.
One of the two selection parts can be disabled (it will not even been shown).
In figure~\ref{selbox-pic} on page~\pageref{selbox-pic} both parts are visible.

Both selections are necessary if a user wants to open a file.
The user is the only one to know which of the PDU types is contained in the file.

Only the file selection is necessary to implement the usual ``Save As\dots'' functionality where the content type is already known.

Only the content type selection is needed when the program wants the user to create a new file without giving it an external file name.
SnaccEd currently has no such function.
Instead, if a user upon opening a file does not select a file name, an internal file without an external file name gets created.

The selection box is implemented in the file {\ufn \dots/tcl-lib/selbox.tcl}.
In this file, every procedure name starts with the prefix {\Tcl selbox\_} (except for the main entry point, {\Tcl selbox}).

If the selbox was made non-modal, it would not break, because each individual selbox widget gets its own widget tree and all its status variables are put into a uniquely named array variable.
The code that generates the names is similar to the example on page~\pageref{tcl-name-space-example-code}.


% file: .../doc/idl-gen.tex

% $Header: /usr/app/odstb/CVS/snacc/doc/idl-gen.tex,v 1.1 1997/01/01 22:47:42 rj Exp $
% $Log: idl-gen.tex,v $
% Revision 1.1  1997/01/01 22:47:42  rj
% first check-in
%

\chapter{\label{idl-code-gen-chapter}IDL Code Generation}

\section{\label{intro-IDL-section}Introduction}

\emph{Under construction}


% file: .../doc/ttab.tex

% $Header: /usr/app/odstb/CVS/snacc/doc/ttab.tex,v 1.1 1997/01/01 22:47:43 rj Exp $
% $Log: ttab.tex,v $
% Revision 1.1  1997/01/01 22:47:43  rj
% first check-in
%

\chapter{\label{ttab-chapter}Type Tables}

Type tables are a flexible and compact way of dealing with ASN.1.
The type table data structure is included in the appendix.  It was
defined in ASN.1 to provide a good storage format for the tables.

When snacc produces a type table it includes the useful types module
as well, if one was specified.  If you are really trying to limit the
size of your type tables, put the only useful types that you need in
your ASN.1 module and compile it without using the useful types
module.

A generic buffer type (� la ISODE and XDR) was defined to allow type
table driven routines to read from a wide variety of buffer formats.
Currently slightly modified versions of the {\C ExpBuf} and the
{\C SBuf} are provided.  It shouldn't be too hard for you to add
support for your own buffer formats.  The generic buffers,
{\C GenBuf}s are described in more detail in a following section.

The general procedure for using type tables is to:
\begin{enumerate}
  \item Use snacc to compile your ASN.1 modules into a type table.
  \item Use {\ufn mkchdr} (make C header, not make cheddar) with the
    type table to produce a friendly C type description of the types in
    the type table.
  \item Load the type table during runtime and use it to configure the table
    encode, decode and other routines.
\end{enumerate}

Step two, making the C header file is not necessary but will make
dealing with the value easier and more type safe.  Internally the
table driven encoders and decoders know nothing of these header file
and treat the types in a uniform, generic manner.  This requires the
encoders and decoders to make assumptions about the way C represents
strucutures and values.  Look in the {\ufn \dots/c-lib/src/tbl-enc.c} and
{\ufn \dots/c-lib/src/tbl-dec.c} files to see how this generic data
structure is manipulated.

On the down side, the compiler directives do not affect the data
structures generated by mkchdr and the generated C type definitions
will generally be different from those generated by the C backend.
This can be fixed, but time was lacking. Type tables also do not
support ANY DEFINED BY types.  Someone could fix this without too much
difficulty.  Only a C type table library is provided.  I didn't have
time to deal with the complexities of creating C++ objects in a
generic way.

Currently the type tables are lacking subtyping information.  It is
available in snacc's main parse tree but I didn't have time to add it
to the tables.  If you want to add it, take the subtype related data
structures (in ASN.1) from asn1module.asn1 (quite a few), remove all
the cruft pertaining to linking and error checking etc, and add it to
the type table type definitions.  Then change the
{\ufn \dots/compiler/core/gen-tbls.c} file to take the subtype information from
the parse tree and put it into the type table.  See the appendix or
{\ufn \dots/asn1specs/} for the ASN.1 definitions of the parse tree
and type tables.

The parse tree itself was defined in ASN.1 so it could be the table
format.  The extra complexity required for linking and error checking
made this very difficult.  Cycles in the data structure and the many
links between the data elements made encoding in BER difficult.
[Maybe ASN.1 needs a type refernce type (i.e. pointer)].


\section{How Type Table See Values}

As mentioned in the last section, table driven encoding, decoding,
printing etc. routines see your values in a generic way.  They do not
have abstract syntax specific header files like those created by
{\ufn mkchdr}.

The basic idea is that all of the standard primitive and list
(SEQUENCE OF and SET OF) types are used and some regular rules are
used for allocating and laying out structs for SEQUENCE, SET and
CHOICE types.

\begin{small}
\begin{verbatim}
/* values from table driver routines' point of view */
typedef void AVal;

/* SEQUENCE and SET type use AStructVal */
typedef AVal *AStructVal;  /* an array of AVal ptrs */
\end{verbatim}
\end{small}

For SETs and SEQUENCEs, the {\C AStructVal} type is used. Its is
basically an array of pointers, one for each component of the
SET/SEQUENCE\@.  Every component is referenced by pointer to simplify
allocations.  OPTIONAL or DEFAULT components may be NULL\@.  For
example, the type:

\begin{small}
\begin{verbatim}
Foo ::= SEQUENCE { a INTEGER, b BOOLEAN, c OCTET STRING }
\end{verbatim}
\end{small}

would be represented as an array of three pointers.  The first pointer
would point to an {\C AsnInt} value, the second would point to an
{\C AsnBool} value, and the third would point to an {\C AsnOcts}
value. {\ufn mkchdr} would give you the following C typedef for Foo:

\begin{small}
\begin{verbatim}
typedef struct Foo
{
    AsnInt *a;
    AsnBool *b;
    AsnOcts *c;
} Foo;
\end{verbatim}
\end{small}

\begin{small}
\begin{verbatim}
/* Internal representation for a CHOICE type */
typedef struct AChoiceVal
{
   enum { achoiceval_notused } choiceId;
   AVal *val;
} AChoiceVal;
\end{verbatim}
\end{small}


A CHOICE type is represented in a way similar to the C backend's
output.  That is, an enum to hold the id of the CHOICE component
that is present and a pointer to the component itself.  For example,
the type:

\begin{small}
\begin{verbatim}
Bar ::= CHOICE { a INTEGER, b BOOLEAN, c OCTET STRING }
\end{verbatim}
\end{small}

would internally be represented as AChoiceVal type.  However,
{\ufn mkchdr} would give you the following:

\begin{small}
\begin{verbatim}
typedef struct Bar
{
    enum
    {
        a_ID = 0,
        b_ID = 1,
        c_ID = 2
    } choiceId;
    union
    {
        AsnInt *a;
        AsnBool *b;
        AsnOcts *c;
    } a;
} Bar;
\end{verbatim}
\end{small}


\section{Type Table Utilities}

There are a bunch of useful routines in {\ufn \dots/c-lib/src/tbl*.c}.  Look
at the source code in {\ufn \dots/tbl-tools/*/} and
{\ufn \dots/tbl-example/} to see how to use some of them.

The {\C LoadTblFile} will decode a type table from a given file.
Notice that its definition of the {\C TBL} data structure has been
augmented to simplify encoding and decoding operations. (Look at the
patch in {\ufn \dots/c-lib/tbl.h.patch} that is applied through the makefile automatically.)
% The additional fields do not screw up the compiled encoder/decoder (in {\ufn \dots/c-lib/src/tbl.c}) provided they are recompiled.
The compiler uses unmodified {\ufn tbl.h} and {\ufn tbl.c} files.

I don't have time to document these routines.  Look through the table
tools, examples and library code.  Their usage should be fairly
obvious.


\section{Type Table Tools}

The {\ufn \dots/tbl-tools/} directory contains three tools, {\ufn mkchdr},
{\ufn pval} and {\ufn ptbl}.  These are described in the following
sections.

\subsection{Making C Header Files with mkchdr}

{\ufn mkchdr} produces a C header file from a type table.  This header
file shows the representation of the types that the table tools will
expect or return for the types in the given type table.

The main use is to provide you with an easy to use definition of the
ASN.1 types C representation.  You do not need to use {\ufn mkchdr}
but it is definately recommended.  Note that the table routines could
have used an even more generic data structure to represent values
(e.g. ISODE's Presentation Elements).  If you have worked with these,
you know that they are cumbersome.

Its synopsis is:
\begin{small}
\begin{verbatim}
mkchdr <tbl-file> [output-file]
\end{verbatim}
\end{small}

If the output file is omitted, the header file is printed to {\C stdout}.

Here is an example of the output.  Given the table that has the
following ASN.1 module in it:

\begin{small}
\begin{verbatim}
P-REC DEFINITIONS ::=
BEGIN
PersonnelRecord ::=  [APPLICATION 0] IMPLICIT SET
{
                     Name,
    title        [0] IA5String,
                     EmployeeNumber,
    dateOfHire   [1] Date,
    nameOfSpouse [2] Name,
    children     [3] IMPLICIT SEQUENCE OF ChildInformation DEFAULT {}
}

ChildInformation ::= SET
{
                    Name,
    dateOfBirth [0] Date
}

Name ::= [APPLICATION 1] IMPLICIT SEQUENCE
{
    givenName  IA5String,
    initial    IA5String,
    familyName IA5String
}

EmployeeNumber ::= [APPLICATION 2] IMPLICIT INTEGER

Date ::= [APPLICATION 3] IMPLICIT IA5String -- YYYYMMDD

END
\end{verbatim}
\end{small}

{\ufn mkchdr} will produce:
\begin{small}
\begin{verbatim}
typedef AsnInt EmployeeNumber;

typedef struct Name
{
    IA5String *givenName;
    IA5String *initial;
    IA5String *familyName;
} Name;

typedef IA5String Date;

typedef struct ChildInformation
{
    Name *field0;
    Date *dateOfBirth;
} ChildInformation;

typedef AsnList PersonnelRecordSeqOf;

typedef struct PersonnelRecord
{
    Name *field0;
    IA5String *title;
    EmployeeNumber *field1;
    Date *dateOfHire;
    Name *nameOfSpouse;
    PersonnelRecordSeqOf *children;
} PersonnelRecord;

\end{verbatim}
\end{small}


\subsection{Printing Tables with ptbl}

{\ufn ptbl} is a program that will show you the contents of a type
table.  It can print a table in two modes:
\begin{itemize}
  \item The value notation for the TBL ASN.1 data structure (see the appendix).
  \item The ASN.1 text version
\end{itemize}

Its synopsis is:

\begin{small}
\begin{verbatim}
ptbl [-a] <tbl-file>
\end{verbatim}
\end{small}

For example, using {\ufn ptbl -a p-rec.tt} to print the
PersonnelRecord module used in the last section would yield:

\begin{small}
\begin{verbatim}
P-REC  DEFINITIONS ::=
BEGIN
EmployeeNumber ::= [APPLICATION 2] IMPLICIT INTEGER
Name ::= [APPLICATION 1] IMPLICIT SEQUENCE
{
    givenName IA5String,
    initial IA5String,
    familyName IA5String
}
Date ::= [APPLICATION 3] IMPLICIT IA5String
ChildInformation ::= SET
{
    Name,
    dateOfBirth [0] Date
}
PersonnelRecordSeqOf ::= SEQUENCE OF ChildInformation
PersonnelRecord ::= [APPLICATION 0] IMPLICIT SET
{
    Name,
    title [0] IA5String,
    EmployeeNumber,
    dateOfHire [1] Date,
    nameOfSpouse [2] Name,
    children [3] IMPLICIT PersonnelRecordSeqOf
}
END
--  Definitions for  ASN-USEFUL
ASN-USEFUL  DEFINITIONS ::=
BEGIN
ObjectDescriptor ::= [UNIVERSAL 7] IMPLICIT OCTET STRING
NumericString ::= [UNIVERSAL 18] IMPLICIT OCTET STRING
PrintableString ::= [UNIVERSAL 19] IMPLICIT OCTET STRING
TeletexString ::= [UNIVERSAL 20] IMPLICIT OCTET STRING
T61String ::= [UNIVERSAL 20] IMPLICIT OCTET STRING
VideotexString ::= [UNIVERSAL 21] IMPLICIT OCTET STRING
IA5String ::= [UNIVERSAL 22] IMPLICIT OCTET STRING
GraphicString ::= [UNIVERSAL 25] IMPLICIT OCTET STRING
VisibleString ::= [UNIVERSAL 26] IMPLICIT OCTET STRING
ISO646String ::= [UNIVERSAL 26] IMPLICIT OCTET STRING
GeneralString ::= [UNIVERSAL 27] IMPLICIT OCTET STRING
UTCTime ::= [UNIVERSAL 23] IMPLICIT OCTET STRING
GeneralizedTime ::= [UNIVERSAL 24] IMPLICIT OCTET STRING
EXTERNAL ::= [UNIVERSAL 8] IMPLICIT SEQUENCE
{
    direct-reference OBJECT IDENTIFIER,
    indirect-reference INTEGER,
    data-value-descriptor ObjectDescriptor,
    encoding CHOICE
    {
        single-ASN1-type [0] OCTET STRING,
        octet-aligned [1] IMPLICIT OCTET STRING,
        arbitrary [2] IMPLICIT BIT STRING
    }
}
END
\end{verbatim}
\end{small}

Note that the useful type module is included in the table.  As
mentioned before, to minimize the size of your tables, put the
definitions of the useful types (from {\ufn \dots/asn1specs/asn-useful.asn1})
into your ASN.1 module and do not compile with useful types module
(i.e. don't use the -u option).  If your module doesn't use any
useful types, just don't use the -u option.

The other mode of {\ufn ptbl}, value notation for the TBL data
structure, produces a lot of output.  It may be useful if you are
debugging one of the table drivers.


\subsection{Printing Values with pval}

The {\ufn pval} program will convert BER values into their value
notation.  Its arguments are a type table file, a module and type name
and a list of BER files of that type.

Its synopsis is:

\begin{small}
\begin{verbatim}
pval -T <tt file name> [-m <module name>] -n <type name> <ber value file list>
\end{verbatim}
\end{small}

Running {\ufn pval} on a PersonnelRecord value produced the following:

\begin{small}
\begin{verbatim}
-- Contents of file "demo-tbls/p-rec.ber"--
-- module = ???, type = PersonnelRecord --

value P-REC.PersonnelRecord ::=
{
  {
    givenName '4a6f686e'H  -- "John" --,
    initial '45'H  -- "E" --,
    familyName '536d697468'H  -- "Smith" --
  },
  title '5468652042696720436865657365'H  -- "The Big Cheese" --,
  99999,
  dateOfHire '3139383230313034'H  -- "19820104" --,
  nameOfSpouse {
    givenName '4d617279'H  -- "Mary" --,
    initial '4c'H  -- "L" --,
    familyName '536d697468'H  -- "Smith" --
  },
  children {
    {
      {
        givenName '4a616d6573'H  -- "James" --,
        initial '52'H  -- "R" --,
        familyName '536d697468'H  -- "Smith" --
      },
      dateOfBirth '3139353730333130'H  -- "19570310" --
    },
    {
      {
        givenName '4c697361'H  -- "Lisa" --,
        initial '4d'H  -- "M" --,
        familyName '536d697468'H  -- "Smith" --
      },
      dateOfBirth '3139363130363231'H  -- "19610621" --
    }
  }
}

 -- decoded 143 bytes for the above value --
\end{verbatim}
\end{small}


\section{Using Tables in Your Own Applications}

The best way to get a handle on using tables is to look at the example
in {\ufn \dots/tbl-example/}.  The general idea is to compile your
ASN.1 into a type table (use the snacc {\ufn -T} option).  If you desire a
livable definition of the C data structures for the types in the type
table, run {\ufn mkchdr} and compile the generated header file with
your C code.  During runtime, simply load your table file with
{\C LoadTblFile} (I use the {\ufn .tt} suffix naming convention for type table
files but it doesn't matter) and then use the {\C TblEncode},
{\C TblDecode}, {\C TblPrint} and {\C TblFree} routines with
your table.  Quite simple.  Seriously.

\section{Using GenBufs}

The {\C GenBuf} generic buffers are really a way of encapsulating
other buffer formats.  A {\C GenBuf} contains a table of pointers to
the buffer functions (the standardized ones (see the buffer section)
plus ``Peek'' routines that the table drivers needed).  They are
defined in {\ufn \dots/c-lib/inc/gen-buf.h}.

{\C GenBuf}s require functions for the standard buffer routines,
macros will not do since you cannot have a pointer to macro.

The benefit of the {\C GenBuf}s is that since they can support other
buffer types, only one set of library routines is needed.  (Note that
there are 3 libraries in {\ufn \dots/c-lib/} for the backend model and only one for the type table model.

Here is most of {\ufn gen-buf.h} to give you an idea of how things work:

\begin{small}
\begin{verbatim}
typedef unsigned char (*BufGetByteFcn) PROTO ((void *b));
typedef unsigned char *(*BufGetSegFcn) PROTO ((void *b,
                                               unsigned long int *lenPtr));
typedef long int (*BufCopyFcn) PROTO ((char *dst, void *b,
                                   unsigned long int len));
typedef void (*BufSkipFcn) PROTO ((void *b, unsigned long int len));
typedef unsigned char (*BufPeekByteFcn) PROTO ((void *b));
typedef unsigned char *(*BufPeekSegFcn) PROTO ((void *b,
                                               unsigned long int lenPtr));
typedef long int (*BufPeekCopyFcn) PROTO ((char *dst, void *b,
                                   unsigned long int len));
typedef void (*BufPutByteRvsFcn) PROTO ((void *b, unsigned char byte));
typedef void (*BufPutSegRvsFcn) PROTO ((void *b, char *data,
                                    unsigned long int len));
typedef int (*BufReadErrorFcn) PROTO ((void *b));
typedef int (*BufWriteErrorFcn) PROTO ((void *b));

typedef struct GenBuf
{
    BufGetByteFcn    getByte;
    BufGetSegFcn     getSeg;
    BufCopyFcn       copy;
    BufSkipFcn       skip;
    BufPeekByteFcn   peekByte;
    BufPeekSegFcn    peekSeg;
    BufPeekCopyFcn   peekCopy;
    BufPutByteRvsFcn putByteRvs;
    BufPutSegRvsFcn  putSegRvs;
    BufReadErrorFcn  readError;
    BufWriteErrorFcn writeError;
    void             *bufInfo;
    void             *spare; /* hack to save space for ExpBuf ** type */
} GenBuf;

#define GenBufGetByte( b)              ((b)->getByte (b->bufInfo))
#define GenBufGetSeg( b, lenPtr)       ((b)->getSeg (b->bufInfo, lenPtr))
#define GenBufCopy( dst, b, len)       ((b)->copy (dst, b->bufInfo, len))
#define GenBufSkip( b, len)            ((b)->skip (b->bufInfo,len))
#define GenBufPeekByte( b)             ((b)->peekByte (b->bufInfo))
#define GenBufPeekSeg( b, lenPtr)      ((b)->peekSeg (b->bufInfo, lenPtr))
#define GenBufPeekCopy( dst, b, len)   ((b)->peekCopy (dst, b->bufInfo, len))
#define GenBufPutByteRvs( b, byte)     ((b)->putByteRvs (b->bufInfo, byte))
#define GenBufPutSegRvs( b, data, len) ((b)->putSegRvs (b->bufInfo, data, len))
#define GenBufReadError( b)            ((b)->readError (b->bufInfo))
#define GenBufWriteError( b)           ((b)->writeError (b->bufInfo))
\end{verbatim}
\end{small}

\section{Type Tables Vs. Metacode}

Please refer to section~\ref{meta-ttab-comparison} on page~\pageref{meta-ttab-comparison}.


% file: .../doc/modifying.tex

% $Header: /usr/app/odstb/CVS/snacc/doc/modifying.tex,v 1.1 1997/01/01 22:47:44 rj Exp $
% $Log: modifying.tex,v $
% Revision 1.1  1997/01/01 22:47:44  rj
% first check-in
%

\chapter{\label{modifying-chapter}Modifying the Compiler}

The compiler consists of about 30,000 lines of yacc, lex and C code
(another 7,000+ for the runtime library routines).  The best way to
understand the compiler internals is to understand the module data
structure ({\ufn \dots/compiler/core/asn1module.h}) and to read the compiler
chapter in this document to gain a conceptual understanding of each
pass of the compiler.

The most common form of modification will likely be for macro
handling.  To understand this, look at the way the OBJECT-TYPE macro is
treated in:
\begin{description}
\item[lex-asn1.l] {add any new keywords}
\item[parse-asn1.y] { parse the macro into the desired data structure.
Use the existing productions as much as possible.}
\item[link-type.c] { link any type defined or referenced in the
macro}
\item[link-values.c] { link any value defined or referenced in the
macro}
\item[do-macros.c] { perform any semantic action for the macro }

\item[normalize.c] { move any type and value definitions in the macro
to the top level so the code generator can generate code for them
(without looking in the macro).}

\item[code generators] { to convert any special semantics into useful
C or C++.  This phase is likely to be dependent on the generated
code's target environment.}
\end{description}

In general I have tried to put comments where funky things happen and
to use function and variable names that are meaningful.  However,
things may get ugly in certain places.  Thesis writing is harmful to
your coding style!


% file: .../doc/future-work.tex

% $Header: /usr/app/odstb/CVS/snacc/doc/future-work.tex,v 1.1 1997/01/01 22:47:45 rj Exp $
% $Log: future-work.tex,v $
% Revision 1.1  1997/01/01 22:47:45  rj
% first check-in
%

\chapter{\label{concl-chapter}Future Work}

There are still many interesting and useful things that can be done to
Snacc.  To name a few, Snacc could be improved to:
\begin{itemize}
  \item{support the new features of 1993 ASN.1}
  \item{generate forward encoders that use only the indefinite length form for constructed BER values}
  \item{support new encoding rules such as the variants of Packed Encoding Rules (PER),
    Canonical Encoding Rules (CER) and Distinguished Encoding Rules (DER).}
  \item{parse complex values.}
\end{itemize}

1993 ASN.1 has many improvements such as eliminating macros and adding
stronger mechanisms for adding type semantics.  With stronger
semantics, the compiler can do more for the user.

New encoding rules such as PER offer improved compression and
possibly better performance in certain cases.  They can be added to
the compiler by adding a new backend to the compiler and providing a
runtime library.

The ability to parse complex ASN.1 value notation values is useful for
protocol testing.  It allows PDUs to be defined in a human readable
form and converted into their encoded representation.  The combination
of an interpreted encoder and decoder with complex value parsing and
support for newer encoding rules would very useful for protocol
testing.


Here is a list of the things that I didn't get time to do:

\begin{enumerate}

  \item {Add contexts to nibble memory.  It's a pain if you decode
    a table and then a value---you can't free the value without freeing
    the table.  This one should be relatively easy and only requires
    modifications to the libraries and the example/user code.}

  \item Spiff up the table stuff.
    \begin{enumerate}
      \item {add subtype info (available in big parse tree) to tbl data struct}
      \item {add enumerated types name/value info to tbl data struct}
      \item {Add ANY DEFINED BY id to type mappings to tbl data struct (available in parse tree if using OBJECT-TYPE macro)}
      \item {Support for C++ table drivers.}
      \item {I'm sure there's more}
    \end{enumerate}

  \item {existing bugs!}
    \begin{enumerate}
      \item {The hash stuff for ANY DEFINED BY mappings will bomb if you go more than 4 levels deep (unlikely, but\dots). Add linear chaining at the bottom level.}
      \item {add the ability to handle CMIP style ANY DEFINED BY ids.  Currently snacc gripes about this and stops.}
	\begin{verbatim}
	   Eg ::= SEQUENCE { id Foo, ANY DEFINED BY id }
	   Foo::= CHOICE { a OBJECT IDENTIFIER, b INTEGER }
	\end{verbatim}
    \end{enumerate}

  \item {Bigger changes}
    \begin{enumerate}
      \item {punt yacc and lex and use PCCTS (Purdue Compiler Construction\dots) (better error reporting, easier to deal with (top down), etc.)}
      \item {make an ASN.1 '93 version (no more macros!!!)}
      \item {add PER.}
      \item {make forwards encoders (use only indefinite lengths, though) for C, C++ and tables.  Allows simpler buffer writing.}
    \end{enumerate}

\end{enumerate}


\appendix

% file: .../doc/asn1-defs.tex

% $Header: /usr/app/odstb/CVS/snacc/doc/asn1-defs.tex,v 1.1 1997/01/01 22:47:47 rj Exp $
% $Log: asn1-defs.tex,v $
% Revision 1.1  1997/01/01 22:47:47  rj
% first check-in
%

\chapter{The Module Data Structure ASN.1 Definition}

The ASN.1 modules are parsed into an internal data structure.  The
data structure was initially defined in ASN.1 with the idea that if
we needed to write a parsed module to disk, the ASN.1 encoding
routines could be used.  No file format was needed so the ASN.1 was
merely an exercise.

This procedure highlighted the problem with circular links and index
like links in ASN.1 data structures.  BER does not support this type
of linking; to handle it, the offending links can be made optional
and not encoded.  After decoding, these links need to be
re-established.  See the type table data structure for a format
suitable for writing to files.

The following is the ASN.1 definition of the Module data structure.
The C translation (as generated by an early version of Snacc) can be
found in {\ufn \dots/compiler/core/asn1module.h}\footnote{
  The {\ufn asn1module.h} that is produced by a current version of Snacc cannot be compiled because its type definitions are in the wrong order.
  This may be caused by the {\ASN --\,--snacc cTypeName} compiler directives, since one of the affected types is BasicTypeChoiceId, but i'm not really sure.
  ---rj
}.
\begin{small}
\begin{verbatim}
-- .../asn1specs/asn1module.asn1
--
-- This module describes the data structure used to represent the
-- compiled ASN.1.
-- Using ASN.1 for the internal data structure allows writing
-- (encoding) to disk for storage (not done yet due to recursive
--  refs back to the module)
--
-- Mike Sample 91/08/29
-- Modifed 92/05  MS
--

Asn1Module DEFINITIONS IMPLICIT TAGS ::=
BEGIN

-- exports everything
-- imports nothing

Modules ::= [APPLICATION 0] IMPLICIT SEQUENCE
{
    creationTime INTEGER,
    modules ModuleList
}

ModuleList ::= SEQUENCE OF Module


Module ::= SEQUENCE
{
    status ENUMERATED { mod-ok(0), mod-not-linked(1), mod-error(2) },
    modId      ModuleId,
    tagDefault ENUMERATED { explicit-tags(0), implicit-tags(1) },
    exportStatus  ENUMERATED { exports-all(0), exports-nothing(1),
                               exports-some(2) },
    imports    ImportModuleList,
    typeDefs   TypeDefList,
    valueDefs  ValueDefList,
    hasAnys    BOOLEAN,

    asn1SrcFileName MyString,
    cHdrFileName    MyString,
    cSrcFileName    MyString,
    cxxHdrFileName  MyString,
    cxxSrcFileName  MyString

    cxxname		MyString, -- META

    idlFileName		MyString, -- IDL
    idlname		MyString -- IDL
}


ModuleId ::= SEQUENCE
{
    name MyString,
    oid  OBJECT IDENTIFIER OPTIONAL --snacc cTypeName:"OID" isPtr:"TRUE"
}


ImportModuleList ::= SEQUENCE OF ImportModule


ImportModule ::= SEQUENCE
{
    modId       ModuleId,
    importElmts ImportElmtList,
    moduleRef   Module, --snacc isEncDec:"FALSE"
    lineNo INTEGER
}


ImportElmtList ::= SEQUENCE OF ImportElmt


ImportElmt ::= SEQUENCE
{
    resolvedRef CHOICE
    {
        type  [0] TypeDef,  -- not encoded
        value [1] ValueDef  -- not encoded
     } OPTIONAL,
    name MyString,
    privateScope BOOLEAN, -- true if from MODNAME.TYPE ref
    lineNo INTEGER
}

TypeDefList ::= SEQUENCE OF TypeDef

OidOrInt ::= CHOICE
{
    oid OBJECT IDENTIFIER,
    intId INTEGER
}

AnyRef ::= SEQUENCE
{
    anyIdName MyString,
    id OidOrInt
}

AnyRefList ::= SEQUENCE OF AnyRef

TypeDef ::= SEQUENCE
{
    exported     BOOLEAN,
    recursive    BOOLEAN,
    isPdu    BOOLEAN,
    localRefCount     INTEGER,
    importRefCount     INTEGER,
    tmpRefCount    INTEGER,
    visited      BOOLEAN,
    definedName  MyString,
    type         Type,
    cTypeDefInfo CTDI,
    cxxTypeDefInfo CxxTDI,
    attrList     AttributeList,
    refList      TypeDefList,
    anyRefs      AnyRefList
}


Tag ::= SEQUENCE
{
    tclass INTEGER, -- swap this for the BER_CLASS enum from basetypes.h
    form  INTEGER,  -- swap this for the BER_FORM enum
    code  INTEGER,
    explicit BOOLEAN,
    valueRef Value
}

Type ::= SEQUENCE
{
    optional   BOOLEAN,
    implicit   BOOLEAN,
    tags       TagList,
    defaultVal [0] IMPLICIT NamedValue OPTIONAL,
    subtypes   [1] Subtype OPTIONAL,
    basicType  [2] BasicType,
    lineNo     INTEGER,
    cTypeRefInfo CTRI,
    cxxTypeRefInfo CxxTRI,
    attrList   AttributeList
}

TagList ::= SEQUENCE OF Tag

AttributeList ::= SEQUENCE OF MyString

NamedNumberList ::= ValueDefList


-- BasicTypes with NULL need no more info that which type it is
-- (this is known from the choice id)

BasicType ::= CHOICE
{
    unknown         [0]  IMPLICIT NULL,
    boolean         [1]  IMPLICIT NULL,
    integer         [2]  IMPLICIT NamedNumberList,
    bitString       [3]  IMPLICIT NamedNumberList,
    octetString     [4]  IMPLICIT NULL,
    null            [5]  IMPLICIT NULL,
    oid             [6]  IMPLICIT NULL,
    real            [7]  IMPLICIT NULL,
    enumerated      [8]  IMPLICIT NamedNumberList,
    sequence        [9]  IMPLICIT NamedTypeList,
    sequenceOf      [10] IMPLICIT Type,
    set             [11] IMPLICIT NamedTypeList,
    setOf           [12] IMPLICIT Type,
    choice          [13] IMPLICIT NamedTypeList,
    selection       [14] IMPLICIT SelectionType,
    componentsOf    [15] IMPLICIT Type, --  [Resolved](local/import) type ref
    any             [16] IMPLICIT NULL,
    anyDefinedBy    [17] IMPLICIT AnyDefinedByType,
    localTypeRef    [19] IMPLICIT TypeRef,
    importTypeRef   [20] IMPLICIT TypeRef,
    macroType       [21] MacroType,
    macroDef        [22] IMPLICIT MacroDef --snacc isPtr:"FALSE"
}

MacroDef ::= MyString -- just keep the text for now

MacroType ::= CHOICE
{
    rosOperation    [0] IMPLICIT RosOperationMacroType,
    rosError        [1] IMPLICIT RosErrorMacroType,
    rosBind         [2] IMPLICIT RosBindMacroType,
    rosUnbind       [3] IMPLICIT RosBindMacroType,
    rosAse          [4] IMPLICIT RosAseMacroType,
    rosAc           [5] IMPLICIT RosAcMacroType,
    mtsasExtension  [6] IMPLICIT MtsasExtensionMacroType,
    mtsasExtensions [7] IMPLICIT MtsasExtensionsMacroType,
    mtsasExtensionAttribute [8] IMPLICIT MtsasExtensionAttributeMacroType,
    mtsasToken      [9] IMPLICIT MtsasTokenMacroType,
    mtsasTokenData [10] IMPLICIT MtsasTokenDataMacroType,
    mtsasSecurityCategory [11] IMPLICIT MtsasSecurityCategoryMacroType,
    asnObject       [12] IMPLICIT AsnObjectMacroType,
    asnPort         [13] IMPLICIT AsnPortMacroType,
    asnRefine       [14] IMPLICIT AsnRefineMacroType,
    asnAbstractBind      [15] IMPLICIT AsnAbstractBindMacroType,
    asnAbstractUnbind    [16] IMPLICIT AsnAbstractBindMacroType,
    asnAbstractOperation [17] IMPLICIT RosOperationMacroType,
    asnAbstractError     [18] IMPLICIT RosErrorMacroType,
    afAlgorithm     [19] IMPLICIT Type,
    afEncrypted     [20] IMPLICIT Type,
    afProtected     [21] IMPLICIT Type,
    afSignature     [22] IMPLICIT Type,
    afSigned        [23] IMPLICIT Type,
    snmpObjectType  [24] IMPLICIT SnmpObjectTypeMacroType
}


AnyDefinedByType ::= SEQUENCE
{
    fieldName MyString,   -- name of field that its defined by
    link NamedType OPTIONAL  -- REFERENCE not encoded
}


SelectionType ::= SEQUENCE
{
    fieldName MyString,   -- name of field in choice
    typeRef   Type,      -- [Resolved](local/import) type ref
    link NamedType OPTIONAL -- REFERENCE not encoded
}

NamedTypeList ::= SEQUENCE OF NamedType

NamedType ::= SEQUENCE
{
    fieldName  MyString, -- may be empty or NULL str
    type       Type
}


TypeRef ::= SEQUENCE
{
    typeName    MyString,
    moduleName  MyString, -- used for "modname.type" refs (may be null)
    module      Module,   --snacc isEncDec:"FALSE"
    link        TypeDef   --snacc isEncDec:"FALSE"
}



RosOperationMacroType ::= SEQUENCE
{
    arguments NamedType,
    result    NamedType,
    errors    [0] IMPLICIT TypeOrValueList OPTIONAL,
    linkedOps [1] IMPLICIT TypeOrValueList OPTIONAL
}

ValueList ::= SEQUENCE OF Value

TypeOrValueList ::= SEQUENCE OF TypeOrValue

TypeOrValue ::= CHOICE
{
    type  [0] IMPLICIT Type,
    value [1] IMPLICIT Value
}

OidList ::= SEQUENCE OF OBJECT IDENTIFIER


RosErrorMacroType ::= SEQUENCE
{
    parameter NamedType
}

RosBindMacroType ::= SEQUENCE
{
    argument NamedType,
    result   NamedType,
    error    NamedType
}


RosAseMacroType  ::= SEQUENCE
{
    operations       ValueList,
    consumerInvokes  ValueList,
    supplierInvokes  ValueList
}

RosAcMacroType ::= SEQUENCE
{
    nonRoElements       ValueList,
    bindMacroType       Type,
    unbindMacroType     Type,
    remoteOperations    Value,
    operationsOf        ValueList,
    initiatorConsumerOf ValueList,
    responderConsumerOf ValueList,
    abstractSyntaxes    OidList
}


MtsasExtensionMacroType ::=  SEQUENCE
{
    elmtType              [0] IMPLICIT NamedType OPTIONAL,
    defaultValue          [1] IMPLICIT Value    OPTIONAL,
    criticalForSubmission [2] IMPLICIT BOOLEAN  OPTIONAL,
    criticalForTransfer   [3] IMPLICIT BOOLEAN  OPTIONAL,
    criticalForDelivery   [4] IMPLICIT BOOLEAN  OPTIONAL
}


MtsasExtensionsMacroType ::= SEQUENCE
{
    extensions ValueList
}

MtsasExtensionAttributeMacroType ::= SEQUENCE
{
    type Type OPTIONAL
}

MtsasTokenMacroType ::= SEQUENCE
{
    type Type OPTIONAL
}

MtsasTokenDataMacroType ::= SEQUENCE
{
    type Type OPTIONAL
}

MtsasSecurityCategoryMacroType ::= SEQUENCE
{
    type Type OPTIONAL
}

AsnObjectMacroType ::= SEQUENCE
{
   ports  AsnPortList OPTIONAL
}

AsnPortList ::= SEQUENCE OF AsnPort

AsnPort ::= SEQUENCE
{
    portValue Value,
    portType  ENUMERATED
    {
         consumer-port (0),
         supplier-port (1),
         symmetric-port (2)
    }
}


AsnPortMacroType ::= SEQUENCE
{
    abstractOps     [0] IMPLICIT TypeOrValueList OPTIONAL,
    consumerInvokes [1] IMPLICIT TypeOrValueList OPTIONAL,
    supplierInvokes [2] IMPLICIT TypeOrValueList OPTIONAL
}


AsnRefineMacroType ::= INTEGER

AsnAbstractBindMacroType ::= SEQUENCE
{
    ports [0] IMPLICIT AsnPortList OPTIONAL,
    type  [1] IMPLICIT Type OPTIONAL
}


SnmpObjectTypeMacroType ::= SEQUENCE
{
   syntax Type,
   access ENUMERATED
       { snmp-read-only (0), snmp-read-write (1),
         snmp-write-only (2), snmp-not-accessible (3)},
   status ENUMERATED
       { snmp-mandatory (0), snmp-optional (1),
         snmp-obsolete (2), snmp-deprecated (3)},
   description [0] IMPLICIT Value OPTIONAL,
   reference   [1] IMPLICIT Value OPTIONAL,
   index       [2] IMPLICIT TypeOrValueList OPTIONAL,
   defVal      [3] IMPLICIT Value OPTIONAL
}


Subtype ::= CHOICE
{
    single [0] SubtypeValue,
    and    [1] IMPLICIT SubtypeList,
    or     [2] IMPLICIT SubtypeList,
    not    [3] Subtype
}

SubtypeList ::= SEQUENCE OF Subtype

SubtypeValue ::= CHOICE
{
    singleValue       [0] IMPLICIT Value,
    contained         [1] IMPLICIT Type,
    valueRange        [2] IMPLICIT ValueRangeSubtype,
    permittedAlphabet [3] Subtype, -- only valuerange or singleval
    sizeConstraint    [4] Subtype, -- only single value ints or val range
    innerSubtype      [5] IMPLICIT InnerSubtype
}


ValueRangeSubtype ::= SEQUENCE
{
    lowerEndInclusive BOOLEAN,
    upperEndInclusive BOOLEAN,
    lowerEndValue Value,
    upperEndValue Value
}


InnerSubtype ::= SEQUENCE
{
    constraintType ENUMERATED { full-ct (0), partial-ct (1), single-ct (2) },
    constraints ConstraintList
}

ConstraintList ::= SEQUENCE OF Constraint

Constraint ::= SEQUENCE
{
    fieldRef MyString,  -- not used if in single-ct, may be null
    presenceConstraint ENUMERATED
    {
        present-ct (0),
        absent-ct (1),
        empty-ct (2),
        optional-ct (3)
    },
    valueConstraints Subtype
}


ValueDefList ::= SEQUENCE OF ValueDef


ValueDef ::= SEQUENCE
{
   exported    BOOLEAN,
   definedName MyString,
   value       Value
}

Value ::= SEQUENCE
{
    type    Type OPTIONAL,
    valueType  INTEGER,  -- holds one of choiceId's def'd for BasicType
    basicValue BasicValue,
    lineNo     INTEGER
}

BasicValue ::= CHOICE
{
    unknown          [0]  IMPLICIT NULL,
    empty            [1]  IMPLICIT NULL,
    integer          [2]  IMPLICIT INTEGER,
    specialInteger   [3]  IMPLICIT SpecialIntegerValue,
    longInteger      [4]  IMPLICIT INTEGER,  -- put LONG before INTGEGER
    boolean          [5]  IMPLICIT BOOLEAN,
    real             [6]  IMPLICIT REAL,
    specialReal      [7]  IMPLICIT SpecialRealValue,
    asciiText        [8]  IMPLICIT OCTET STRING,
    asciiHex         [9]  IMPLICIT OCTET STRING,
    asciiBitString   [10] IMPLICIT OCTET STRING,
    oid              [11] IMPLICIT OBJECT IDENTIFIER,
    linkedOid        [12] IMPLICIT OBJECT IDENTIFIER, --snacc cTypeName:"OID"
    berValue         [13] IMPLICIT OCTET STRING,
    perValue         [14] IMPLICIT OCTET STRING,
    namedValue       [15] IMPLICIT NamedValue,
    null             [16] IMPLICIT NULL,
    localValueRef    [17] IMPLICIT ValueRef,
    importValueRef   [18] IMPLICIT ValueRef,
    valueNotation    [19] IMPLICIT OCTET STRING
}



SpecialIntegerValue ::= ENUMERATED { min-int (0), max-int (1) }
SpecialRealValue ::= ENUMERATED { minus-infinity-real (0), plus-infinity-real (1) }


ValueRef ::= SEQUENCE
{
    valueName  MyString,
    moduleName MyString, -- used for "modname.value" refs (may be null)
    link       ValueDef, --snacc isEncDec:"FALSE"
    module     Module    --snacc isEncDec:"FALSE"
}

NamedValue ::= SEQUENCE
{
   fieldName MyString, -- may be null
   value     Value
}

NamedValueList ::= SEQUENCE OF NamedValue

CTypeId  ::= ENUMERATED { c-choice (0), c-list (1), c-any (2), c-anydefinedby (3),
                          c-lib (4), c-struct (5), c-typeref (6), c-no-type (7),
                         c-typedef (8) }

-- C Type Def Info - info used for routine naming
-- and referencing from other types
CTDI ::= SEQUENCE
{
    asn1TypeId         INTEGER, --snacc cTypeName:"enum BasicTypeChoiceId"
    cTypeId            CTypeId,
    cTypeName          MyString,
    isPdu              BOOLEAN,
    isEncDec           BOOLEAN, -- if false, no routines are gen
                                -- and not included in encodings
    isPtrForTypeDef    BOOLEAN,
    isPtrForTypeRef    BOOLEAN,
    isPtrInChoice      BOOLEAN,
    isPtrForOpt        BOOLEAN,

               -- defines these names, used by references
    optTestRoutineName MyString, -- routine/macro to check whether
                                 -- opt type is present
    defaultFieldName   MyString, -- base for generating field names

    printRoutineName   MyString,
    encodeRoutineName  MyString,
    decodeRoutineName  MyString,
    freeRoutineName    MyString,

    genPrintRoutine    BOOLEAN,
    genEncodeRoutine   BOOLEAN,
    genDecodeRoutine   BOOLEAN,
    genFreeRoutine     BOOLEAN,
    genTypeDef         BOOLEAN
}


--
-- CTRI (C Type Ref Info) is used for generating C typedefinitions
-- from the ASN.1 types info
CTRI ::= SEQUENCE
{
    cTypeId CTypeId,
    cFieldName MyString,
    cTypeName  MyString,
    isPtr BOOLEAN,
--    isEndCType BOOLEAN,        -- false for struct/union def
    cNamedElmts CNamedElmts OPTIONAL,      -- for C_LIB bits/int/enums
    choiceIdValue INTEGER,     -- enum value of this c field
    choiceIdSymbol MyString,   -- this fields sym in choiceId enum
    choiceIdEnumName MyString,
    choiceIdEnumFieldName MyString,
    optTestRoutineName MyString, -- these names are gained from refd type def
    printRoutineName   MyString,   -- or are over-ridden snacc attribute comment
    encodeRoutineName  MyString,
    decodeRoutineName  MyString,
    freeRoutineName    MyString,
    isEncDec           BOOLEAN -- whether part of enc value
}

CNamedElmts ::= SEQUENCE OF CNamedElmt

CNamedElmt ::= SEQUENCE
{
    name MyString,
    value INTEGER
}


CxxTDI ::= SEQUENCE
{
    asn1TypeId         INTEGER, --snacc cTypeName:"enum BasicTypeChoiceId"
    className          MyString,
    isPdu              BOOLEAN,
    isEnc              BOOLEAN,
    isPtrForTypeDef    BOOLEAN,
    isPtrForOpt        BOOLEAN,
    isPtrInChoice      BOOLEAN,
    isPtrInSetAndSeq   BOOLEAN,
    isPtrInList        BOOLEAN,
    optTestRoutineName MyString,
    defaultFieldName   MyString -- base for generating field names
}



CxxTRI ::= SEQUENCE
{
    isEnc BOOLEAN,
    className MyString,
    fieldName MyString,
    isPtr BOOLEAN,
    namedElmts CNamedElmts,
    choiceIdSymbol MyString,
    choiceIdValue INTEGER,
    optTestRoutineName MyString
}

IDLTDI ::= SEQUENCE
{
    asn1TypeId         INTEGER, --snacc cTypeName:"enum BasicTypeChoiceId"
    typeName           MyString,
    isPdu              BOOLEAN,
    isEnc              BOOLEAN,
    isPtrForTypeDef    BOOLEAN,
    isPtrForOpt        BOOLEAN,
    isPtrInChoice      BOOLEAN,
    isPtrInSetAndSeq   BOOLEAN,
    isPtrInList        BOOLEAN,
    optTestRoutineName MyString,
    defaultFieldName   MyString -- base for generating field names
}

IDLTRI ::= SEQUENCE
{
    isEnc BOOLEAN,
    typeName MyString,
    fieldName MyString,
    isPtr BOOLEAN,
    namedElmts CNamedElmts,
    choiceIdSymbol MyString,
    choiceIdValue INTEGER,
    optTestRoutineName MyString
}

-- use snacc compiler directives to overide the builtin types.
--
-- All strings used in module data struct are null terminated so
-- can just use a char *
-- Note the snacc comments before the PrintableString
-- bind with the MyString TypeDef and the ones after PrintableString
-- bind with the PrintableString Type ref.


MyString ::= --snacc isPtrForTypeDef:"FALSE"
             --snacc isPtrForTypeRef:"FALSE"
             --snacc isPtrInChoice:"FALSE"
             --snacc isPtrForOpt:"FALSE"
             --snacc optTestRoutineName:"MYSTRING_NON_NULL"
             --snacc genPrintRoutine:"FALSE"
             --snacc genEncodeRoutine:"FALSE"
             --snacc genDecodeRoutine:"FALSE"
             --snacc genFreeRoutine:"FALSE"
             --snacc printRoutineName:"printMyString"
             --snacc encodeRoutineName:"EncMyString"
             --snacc decodeRoutineName:"DecMyString"
             --snacc freeRoutineName:"FreeMyString"
             PrintableString --snacc cTypeName:"char *"

END
\end{verbatim}
\end{small}


\chapter{The Type Table (TBL) Data Structure ASN.1 Definition}

The following is the type table data structure that Snacc uses for
type table values.  Using ASN.1 gives a representation suitable for
saving tables to files or sending them over a network to reconfigure a
device (e.g. SNMP mib).

This file is actually compiled by Snacc to compile itself.
For bootstrapping purposes, an initial version is included in the distribution.

\begin{small}
\begin{verbatim}
-- .../asn1specs/tbl.asn1
--
--  TBL types describe ASN.1 data structures.
--  These can be used in generic, interpretive encoders/decoders.
--  Interpretive decoders are typically slower, but don't eat memory
--  with type-specific encoding and decoding code.
--  The tbl types can also be sent over the network
--  and allow dynamic re-configuration of encoders/decoders.
--
--  This definition is fairly small so it should be reasonable easy
--  to understand.  To learn more about semantics of this data
--  struct, look in snacc/tbl-tools/print-tbl/pasn1.c.
--
--  Copyright Mike Sample and UBC, 1992, 1993
--

TBL DEFINITIONS ::=
BEGIN


-- imports nothing
-- exports nothing

TBL ::= --snacc isPdu:"TRUE" -- SEQUENCE
{
   totalNumModules  INTEGER,  -- these totals can help allocation
   totalNumTypeDefs INTEGER,  -- when decoding (ie use arrays)
   totalNumTypes    INTEGER,
   totalNumTags     INTEGER,
   totalNumStrings  INTEGER,
   totalLenStrings  INTEGER,
   modules SEQUENCE OF TBLModule
}

TBLModule ::= SEQUENCE
{
    name     [0] IMPLICIT PrintableString,
    id       [1] IMPLICIT OBJECT IDENTIFIER OPTIONAL,
    isUseful [2] IMPLICIT BOOLEAN, -- true if useful types module
    typeDefs [3] IMPLICIT SEQUENCE OF TBLTypeDef
}

TBLTypeDef ::= SEQUENCE
{
   typeDefId TBLTypeDefId,
   typeName  PrintableString OPTIONAL, -- I have forgotten why this is opt!
   type      TBLType
}

TBLType ::= SEQUENCE
{
   typeId    [0] IMPLICIT TBLTypeId,
   optional  [1] IMPLICIT BOOLEAN,
   tagList   [2] IMPLICIT SEQUENCE OF TBLTag OPTIONAL,
   content   [3] TBLTypeContent,
   fieldName [4] IMPLICIT PrintableString OPTIONAL
}

TBLTypeContent ::= CHOICE
{
   primType [0] IMPLICIT NULL,
   elmts    [1] IMPLICIT SEQUENCE OF TBLType,
   typeRef  [2] IMPLICIT TBLTypeRef
}

TBLTypeRef ::= SEQUENCE
{
    typeDef TBLTypeDefId,
    implicit BOOLEAN
}

TBLTypeId ::= ENUMERATED
{
    tbl-boolean (0),
    tbl-integer (1),
    tbl-bitstring (2),
    tbl-octetstring (3),
    tbl-null (4),
    tbl-oid (5),
    tbl-real (6),
    tbl-enumerated (7),
    tbl-sequence (8),
    tbl-set (9),
    tbl-sequenceof (10),
    tbl-setof (11),
    tbl-choice (12),
    tbl-typeref (13)
}

TBLTypeDefId ::= INTEGER

TBLTag ::= SEQUENCE
{
    tclass TBLTagClass,
    code INTEGER (0..MAX)
}

TBLTagClass ::=  ENUMERATED { universal (0), application (1),
                             context (2), private (3)}

END

\end{verbatim}
\end{small}

\chapter{\label{edex-files}ASN.1 Files for the Editor Example}

The files can be found in {\ufn \dots/tcl-example/}.

\begin{ASNcode}
\label{edex0.asn1}%
--\,-- file: edex0.asn1\\
--\,--\\
--\,-- SnaccEd example, simple types module\\
\\
EdEx-Simple DEFINITIONS ::=\\
BEGIN\\
\\
RainbowColor ::= INTEGER\\
\{\\
  \>red(0), orange(1), yellow(2), green(3), blue(4), indigo(5), violet(6)\\
\}\\
\\
DayOfTheWeek ::= ENUMERATED\\
\{\\
  \>sunday(0), monday(1), tuesday(2), wednesday(3), thursday(4), friday(5), saturday(6)\\
\}\\
\\
Hand ::= BIT STRING\\
\{\\
  \>thumb(0), forefinger(1), middle-finger(2), ring-finger(3), little-finger(4)\\
\}\\
\\
victory Hand ::= \{ forefinger, middle-finger \}\\
\\
END
\end{ASNcode}

\begin{ASNcode}
\label{edex1.asn1}%
--\,-- file: edex1.asn1\\
--\,--\\
--\,-- SnaccEd example, structured types module\\
\\
EdEx-Structured DEFINITIONS ::=\\
BEGIN\\
\\
IMPORTS RainbowColor, DayOfTheWeek, Hand FROM EdEx-Simple;\\
\\
RGBColor ::= SEQUENCE\\
\{\+\\
  red INTEGER,\\
  green INTEGER,\\
  blue INTEGER\-\\
\}\\
\\
Coordinate ::= CHOICE\\
\{\+\\
  cartesian [0] SEQUENCE \{ x REAL, y REAL \},\\
  polar [1] SEQUENCE \{ angle REAL, distance REAL \}\-\\
\}\\
\\
File ::= SET\\
\{\+\\
  name [0] PrintableString,\\
  contents [1] OCTET STRING,\\
  checksum [2] INTEGER OPTIONAL,\\
  read-only [3] BOOLEAN DEFAULT FALSE\-\\
\}\\
\\
Directory ::= SET\\
\{\+\\
  name PrintableString,\\
  files SET OF File\-\\
\}\\
\\
Simple ::= SET\\
\{\+\\
  null   [0] NULL,\\
  bool   [1] BOOLEAN,\\
  day    [2] DayOfTheWeek,\\
  int    [3] INTEGER,\\
  color  [4] RainbowColor,\\
  real   [5] REAL,\\
  bits   [6] Hand,\\
  str    [7] OCTET STRING,\\
  optstr [8] OCTET STRING OPTIONAL\-\\
\}\\
\\
Structured ::= SET\\
\{\+\\
  coord [0] Coordinate,\\
  color [1] CHOICE \{ rainbow RainbowColor, rgb RGBColor \}\-\\
\}\\
\\
Various ::= SET\\
\{\+\\
  simple [0] Simple,\\
  struct [1] Structured,\\
  recursion [2] Various OPTIONAL\-\\
\}\\
\\
END
\end{ASNcode}


% file: .../doc/coding.tex

% $Header: /usr/app/odstb/CVS/snacc/doc/coding.tex,v 1.1 1997/01/01 22:47:48 rj Exp $
% $Log: coding.tex,v $
% Revision 1.1  1997/01/01 22:47:48  rj
% first check-in
%

\chapter{\label{coding-tricks}Coding Tricks For Readability}

\index{ANSI C}
\index{K&R C}
\index{PROTO}
\index{PARAM}
\index{COMMA}

One of our project partners needed some additional function arguments and so they duplicated the function declarator and put a preprocessor switch around it.
The metacode and the Tcl interface added some additional compilation conditions.
Since the Tcl interface is only useful on top of the metacode, there are six different combinations instead of eight.
Even these six combinations would have made the code very ugly.
Please decide for yourself, here is an example with an example function with only up to four arguments, {\C PrintCxxCode()} has up to 17 arguments.

\begin{Ccode}
static void\\
\#if A\\
FunctionName PARAMS ((a, b),\\
  \>TypeA a \_AND\_\\
  \>TypeB b)\\
\#if B\\
FunctionName PARAMS ((a, b, c),\\
  \>TypeA a \_AND\_\\
  \>TypeB b \_AND\_\\
  \>TypeC c)\\
\#if C\\
FunctionName PARAMS ((a, b, c, d),\\
  \>TypeA a \_AND\_\\
  \>TypeB b \_AND\_\\
  \>TypeC c \_AND\_\\
  \>TypeD d)\\
\#endif\\
\#endif\\
\#else\\
FunctionName PARAMS ((b),\\
  \>TypeB b)\\
\#if B\\
FunctionName PARAMS ((b, c),\\
  \>TypeB b \_AND\_\\
  \>TypeC c)\\
\#if C\\
FunctionName PARAMS ((b, c, d),\\
  \>TypeB b \_AND\_\\
  \>TypeC c \_AND\_\\
  \>TypeD d)\\
\#endif\\
\#endif\\
\#endif
\end{Ccode}

Here is the code after introduction of my shorthand:

\begin{Ccode}
static void\\
FunctionName PARAMS ((if\_A (a COMMA) b if\_C (COMMA c) if\_D (COMMA d)),\\
  \>if\_A (TypeA a \_AND\_)\\
  \>TypeB b\\
  \>if\_C (\_AND\_ TypeC c)\\
  \>if\_D (\_AND\_ TypeD d))
\end{Ccode}

The tricks are very simple.
One is the {\C if\_\dots} macro

\begin{Ccode}
\#if A\\
\#define if\_A( code)			\>\>\>code\\
\#else\\
\#define if\_A( code)
\#endif
\end{Ccode}

that lets us get rid of at least four lines of code for every invocation, and the other trick is the {\C COMMA} macro

\begin{Ccode}
\#define COMMA \>\>\>,
\end{Ccode}

that makes the arguments to the {\C if\_\dots{}} macros look like a single argument.
Without this trick, 

The other trick, of course is the {\C if\_\dots{}} macro itself.
The {\C if\_\dots{}} macros have to expand into code without brackets, for example {\C if\_A (a COMMA b)} exands into {\C a, b}.
The {\C COMMA} is not my invention, snacc's {\C \_AND\_} macro is exactly the same.
Both {\C \_AND\_} and {\C COMMA} serve the purpose of being a comma (``,'') as the final result (well, only for ANSI C, for K\&R C, the {\C \_AND\_} becomes a semicolon), but without being an argument separator to the C preprocessor.
The {\C PROTO} macro that was already present in snacc 1.1 gets a single argument as well, but by means of additional parenthesis, inside which commas can savely be used.
It expands into code with brackets around it: {\C PROTO ((int a, char~*b))} expands into {\C (int a, char~*b)}.
The first argument to the {\C PARAM} macro is bracketed list as well, and for the arguments purpose, to be a function argument list, this is fine.

To have both an ANSI C and a K\&R C version, without {\C PROTO}, {\C PARAMS} and supporting macros, twelve conditional code compilations would have to be written out instead of one!
And what a tedious job to maintain all twelve versions!

% file: .../doc/makefile.tex

% $Header: /usr/app/odstb/CVS/snacc/doc/makefile.tex,v 1.1 1997/01/01 22:47:49 rj Exp $
% $Log: makefile.tex,v $
% Revision 1.1  1997/01/01 22:47:49  rj
% first check-in
%

\chapter{\label{makefile-sect}Makefiles}

Some of Snacc's makefiles look rather sophisticated.
This section explains some of the tricks.

\section{CVS, Dependencies and Make's Include Statement}

The makefiles take advantage of the file inclusion feature.
Since this has already been supported by UNIX System III\footnote{yes, System III, not System V R3} make (somewhen around 1980), I consider it to be pretty portable.
If your make is crippled, either use a newer one (e.g. GNU make), or as a last resort, remove (better: comment out) the include statements and call make with the additional arguments {\ufn -f .../makehead -f makefile -f dependencies -f .../maketail}.

Snacc's configuration script generates the file {\ufn makehead} which gets included by all makefiles.
It contains a lot of definitions used by make.

The dependencies have been moved out of each makefile into a separate file called {\ufn dependencies} that is not under cvs control---otherwise, the makefiles would inflate the repository unnecessarily.
The makefiles have an include statement for their dependencies file.
GNU make automatically makes the dependencies if the file does not exist, but other versions of {\ufn make} simply give up.
In that case, an initial (empty) file has to be generated.
Snacc's top level makefile does this for you if you call {\ufn make depend}.

A third file that is included by almost every makefile is {\ufn \dots/maketail}.
It holds the rules that are common to all makefiles where C/C++ code is compiled.

\section{Circular Dependencies}

In a normal makefile rule, a file depends upon other files.
If any of a file's dependencies is newer, the file is remade.
This goes well as long as the dependency graph is non-circular, but snacc is compiled from some files it has generated itself.
This recursion can lead to one of two results: in the worse case, {\ufn make} builds the compiler because its source files are newer, builds the source files because the compiler is newer, builds the compiler because some source files are newer, and so on ad infinitum\dots{}
Even if this endless recursion does not happen, one or two of the above steps will be made every time {\ufn make} is called.
To avoid this waste of time, one lets the compiler generate a new source file, but when the new and the old version are identical, the old file is kept and {\ufn make} sees that the compiler is up-to-date, and the recursion is terminated.
Of course, if the source file's contents did change, it is replaced with the new version.

This is a simplified example of a normal makefile:

\begin{Makefile}
snacc:		\>tbl.h\\
		\>\emph{compile} snacc\\
\\
tbl.h:		\>snacc tbl.asn1\\
		\>./snacc \dots{} tbl.asn1\\
\end{Makefile}

Most {\ufn make} versions will complain and print a warning about this `infinite loop' or `circular dependency'.
The first approach towards a solution could be:

\begin{Makefile}
snacc:		\>tbl.h\\
		\>\emph{compile} snacc\\
\\
tbl.h:		\>snacc tbl.asn1\\
		\>mv tbl.h tbl.h.prev\\
		\>./snacc \dots{} tbl.asn1\\
		\>if cmp tbl.h.prev tbl.h; then\char`\\\\
		\>\>  echo "tbl.h hasn't changed";\char`\\\\
		\>\>  mv tbl.h.prev tbl.h;\char`\\\\
		\>else\char`\\\\
		\>\>  \$(RM) tbl.h.prev;\char`\\\\
		\>fi\\
\end{Makefile}

The effect is that you keep snacc from being remade if the contents of tbl.h did not change, but the two steps to create tbl.h and to test whether it is different from tbl.h.prev will be made every time snacc or tbl.asn1 are newer than tbl.h, which they most often will be since few of the changes to snacc will affect tbl.h's contents.
And {\ufn make} will still complain about the recursion.
To solve all this, another file, a stamp file is introduced.
It separates the file's contents from its modification time:

\begin{Makefile}
snacc:		\>tbl.h\\
		\>\emph{compile} snacc\\
\\
stamp-tbl:	\>snacc tbl.asn1\\
		\>mv tbl.h tbl.h.prev\\
		\>./snacc \dots{} tbl.asn1\\
		\>if cmp tbl.h.prev tbl.h; then\char`\\\\
		\>\>  echo "tbl.h hasn't changed";\char`\\\\
		\>\>  mv tbl.h.prev tbl.h;\char`\\\\
		\>else\char`\\\\
		\>\>  \$(RM) tbl.h.prev;\char`\\\\
		\>fi\\
		\>date > \$@\\
\\
tbl.h:		\>stamp-tbl\\
		\>@true\\
\end{Makefile}

The dummy command in the rule for tbl.h is necessary, since otherwise, despite stamp-tbl commands having modified tbl.h, many versions of make think that tbl.h has not been modified.

If you want {\ufn tbl.h} to be remade (e.g. you have changed an option to snacc), you must delete {\ufn stamp-tbl}---{\ufn tbl.h} may (and should) be left in place.

The rules in {\ufn \dots/compiler/makefile}, {\ufn \dots/c-lib/makefile} and {\ufn \dots/c++-lib/makefile} are further complicated by the fact that
\begin{enumerate}
  \item snacc prints the current time into the file which the comparison must take into account
  \item if snacc has not been built it cannot be used to generate its source files---a bootstrapping version of snacc's source files has got to be supplied.
\end{enumerate}

\section{Compiling Different Libraries From One Set Of Source Files}

The different libraries in {\ufn \dots/c-lib/} and {\ufn \dots/c++-lib/} get made by means of recursive calls to make with different macro settings.
This keeps the makefiles short as it avoids a lot of duplication of file lists and rules which would be a hassle to maintain.
The different libraries get compiled from the same set of source files, the code to be compiled is determined through {\ufn cpp} (C preprocessor) macro switches.

\section{Configuration, Optional Code and Makefiles}

The {\ufn \dots/configure} script looks for Tcl/Tk.
If they are absent, there is no use in trying to compile Snacc's Tcl interface.
For makefiles to detect whether the Tcl interface should be compiled or not, there is a file {\ufn \dots/tcl-p.c} that, after being compiled into {\ufn tcl-p}, exits with 0 (the shells' `true' value) if Tcl/Tk is present and the user has not disabled this option by setting {\C NO\_TCL} in {\ufn \dots/policy.h} to {\C 1}.
{\ufn tcl-p} gets made automatically.


\bibliography{snacc,corba,tcl}
\bibliographystyle{plain}

\end{document}
